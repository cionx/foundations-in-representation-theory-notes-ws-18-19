\section{Long Exact Sequence}


\begin{theorem}[Long exact (co)homology sequence]
  \leavevmode
  \begin{enumerate}
    \item
      For every short exact sequence of chain complexes
      \[
        \xi
        \colon
        0
        \to
        \Ccc'
        \to
        \Ccc
        \to
        \Ccc''
        \to
        0
      \]
      and every~$n \in \Integer$ there exists a connecting morphism
      \[
        \del_n
        =
        \del_n^\xi
        \colon
        \Hl_n(\Ccc'')
        \to
        \Hl_{n-1}(\Ccc')
      \]
      with the following two properties:
      \begin{enumerate}
        \item
          The resulting sequence
          \[
            \begin{tikzcd}
                {}
              & \dotsb
                \arrow{r}
                \arrow[d, phantom, ""{coordinate, name=Y}]
              & \Hl_{n+1}(\Ccc'')
                \arrow[ dll,
                        "\del_{n+1}",
                        rounded corners,
                        to path={ -- ([xshift=2ex]\tikztostart.east)
                                  |- (Y) \tikztonodes
                                  -| ([xshift=-2ex]\tikztotarget.west)
                                  -- (\tikztotarget)}
                      ]
              \\
                \Hl_n(\Ccc')
                \arrow{r}
              & \Hl_n(\Ccc)
                \arrow{r}
                \arrow[d, phantom, ""{coordinate, name=Z}]
              & \Hl_n(\Ccc'')
                \arrow[ dll,
                        "\del_n",
                        rounded corners,
                        to path={ -- ([xshift=2ex]\tikztostart.east)
                                  |- (Z) \tikztonodes
                                  -| ([xshift=-2ex]\tikztotarget.west)
                                  -- (\tikztotarget)}
                      ]
              \\
                \Hl_{n-1}(\Ccc')
                \arrow{r}
              & \dotsb
              & {}
            \end{tikzcd}
          \]
          is exact.
        \item
          \label{fuctoriality of long exact homology sequence}
          If
          \[
            \begin{tikzcd}
                0
                \arrow{r}
              & \Ccc'
                \arrow{r}
                \arrow{d}
              & \Ccc
                \arrow{r}
                \arrow{d}
              & \Ccc''
                \arrow{r}
                \arrow{d}
              & 0
              \\
                0
                \arrow{r}
              & \Dcc'
                \arrow{r}
              & \Dcc
                \arrow{r}
              & \Dcc''
                \arrow{r}
              & 0
            \end{tikzcd}
          \]
          is a commutative diagram of chain complexes with (short) exact rows~$\xi$ and~$\zeta$ then the following square commutes for every~$n \in \Integer$:
          \[
            \begin{tikzcd}
                \Hl_n(\Ccc'')
                \arrow{r}[above]{\del_n^\xi}
                \arrow{d}
              & \Hl_{n-1}(\Ccc')
                \arrow{d}
              \\
                \Hl_n(\Dcc'')
                \arrow{r}[above]{\del_n^\zeta}
              & \Hl_{n-1}(\Dcc')
            \end{tikzcd}
          \]
      \end{enumerate}
    \item
      Dually, there exists for every short exact sequence of cochain complexes
      \[
        \xi
        \colon
        0
        \to
        \bp\Cccc
        \to
        \Cccc
        \to
        \bpp\Cccc
        \to
        0
      \]
      and every~$n \in \Integer$ a connecting morphism
      \[
        \del^n
        =
        \del^n_\xi
        \colon
        \Hl^n(\,\bpp\Cccc)
        \to
        \Hl^{n+1}(\,\bp\Cccc)
      \]
      with the following two properties:
      \begin{enumerate}
        \item
          The resulting sequence
          \[
            \begin{tikzcd}
                {}
              & \dotsb
                \arrow{r}
                \arrow[d, phantom, ""{coordinate, name=Y}]
              & \Hl^{n-1}(\,\bpp\Cccc)
                \arrow[ dll,
                        "\del^{n-1}",
                        rounded corners,
                        to path={ -- ([xshift=2ex]\tikztostart.east)
                                  |- (Y) \tikztonodes
                                  -| ([xshift=-2ex]\tikztotarget.west)
                                  -- (\tikztotarget)}
                      ]
              \\
                \Hl^n(\,\bp\Cccc)
                \arrow{r}
              & \Hl^n(\Cccc)
                \arrow{r}
                \arrow[d, phantom, ""{coordinate, name=Z}]
              & \Hl^n(\,\bpp\Cccc)
                \arrow[ dll,
                        "\del^n",
                        rounded corners,
                        to path={ -- ([xshift=2ex]\tikztostart.east)
                                  |- (Z) \tikztonodes
                                  -| ([xshift=-2ex]\tikztotarget.west)
                                  -- (\tikztotarget)}
                      ]
              \\
                \Hl^{n+1}(\,\bp\Ccc)
                \arrow{r}
              & \dotsb
              & {}
            \end{tikzcd}
          \]
          is exact.
        \item
          If
          \[
            \begin{tikzcd}
                0
                \arrow{r}
              & \bp\Cccc
                \arrow{r}
                \arrow{d}
              & \Cccc
                \arrow{r}
                \arrow{d}
              & \bpp\Cccc
                \arrow{r}
                \arrow{d}
              & 0
              \\
                0
                \arrow{r}
              & \bp\Dccc
                \arrow{r}
              & \Dccc
                \arrow{r}
              & \bpp\Dccc
                \arrow{r}
              & 0
            \end{tikzcd}
          \]
          is a commutative diagram of cochain complexes with (short) exact rows~$\xi$ and~$\zeta$ then the following square commutes for every~$n \in \Integer$:
          \[
            \begin{tikzcd}
                \Hl^n(\,\bpp\Cccc)
                \arrow{r}[above]{\del^n_\xi}
                \arrow{d}
              & \Hl^{n+1}(\,\bp\Cccc)
                \arrow{d}
              \\
                \Hl^n(\,\bpp\Dccc)
                \arrow{r}[above]{\del^n_\zeta}
              & \Hl^{n+1}(\,\bp\Dccc)
            \end{tikzcd}
          \]
      \end{enumerate}
  \end{enumerate}
\end{theorem}


\begin{proof}
  This proof is currently missing from these notes and will be added later.
% TODO: Add this proof.
\end{proof}


\begin{remark*}
  If
  \[
    \begin{tikzcd}
        0
        \arrow{r}
      & \Ccc'
        \arrow{r}
        \arrow{d}
      & \Ccc
        \arrow{r}
        \arrow{d}
      & \Ccc''
        \arrow{r}
        \arrow{d}
      & 0
      \\
        0
        \arrow{r}
      & \Dcc'
        \arrow{r}
      & \Dcc
        \arrow{r}
      & \Dcc''
        \arrow{r}
      & 0
    \end{tikzcd}
  \]
  is a short exact sequence of chain complexes in~$\Acat$ then property~\ref*{fuctoriality of long exact homology sequence} states that the induced ladder diagram
  \[
    \begin{tikzcd}[column sep = 1em]
        \dotsb
        \arrow{r}
      & \Hl_{n+1}(\Ccc'')
        \arrow{rr}[above]{\del_{n+1}^\xi}
        \arrow{d}
      & {}
      & \Hl_{n}(\Ccc')
        \arrow{r}
        \arrow{d}
      & \Hl_{n}(\Ccc)
        \arrow{r}
        \arrow{d}
      & \Hl_{n}(\Ccc'')
        \arrow{rr}[above]{\del_n^\xi}
        \arrow{d}
      & {}
      & \Hl_{n-1}(\Ccc')
        \arrow{r}
        \arrow{d}
      & \dotsb
      \\
        \dotsb
        \arrow{r}
      & \Hl_{n+1}(\Dcc'')
        \arrow{rr}[above]{\del_{n+1}^\zeta}
      & {}
      & \Hl_{n}(\Dcc')
        \arrow{r}
      & \Hl_{n}(\Dcc)
        \arrow{r}
      & \Hl_{n}(\Dcc'')
        \arrow{rr}[above]{\del_n^\zeta}
      & {}
      & \Hl_{n-1}(\Dcc')
        \arrow{r}
      & \dotsb
    \end{tikzcd}
  \]
  with (long) exact rows is again commutative.
  For cochain complexes the analogous statement holds.
\end{remark*}


\begin{lemma*}
  \leavevmode
  \begin{enumerate}
    \item
      Let~$0 \to \Ccc' \to \Ccc \to \Ccc'' \to 0$ be a short exact sequence of chain complexes in~$\Acat$.
      If any two of the chain complexes,~$\Ccc$,~$\Ccc$,~$\Ccc''$ are acyclic then the third one is also acyclic.
    \item
      Let~$f \colon \Ccc \to \Dcc$ be a morphism of chain complexes.
      If~$\ker(f)$ and~$\coker(f)$ are acyclic then~$f$ is a~{\qim}.
  \end{enumerate}
\end{lemma*}


\begin{proof}
  This is Exercise~2 of Exercise~sheet~8.
\end{proof}


\begin{example}
  Let~$X$ be a topological space.
  \begin{enumerate}
    \item
      The singular chain complex~$\Ccc^\sing(X)$ is covariantly functorial in~$X$, in the following sense:
      If~$Y$ is another topological space and~$f \colon X \to Y$ is a continuous map, then for every~{\simplex{$n$}}~$\sigma \colon \Delta^n \to X$ in~$X$ we get an~{\simplex{$n$}}~$f \sigma \colon \Delta^n \to Y$ in~$Y$.
      We hence get for every~$n \geq 0$ an induced group homomorphism
      \[
        f_n
        \colon
        C^\sing_n(X)
        \to
        C^\sing_n(Y)  \,,
      \]
      that is on basis elements given by~$f_n(\sigma) = f \sigma$.
      By also setting~$f_n = 0$ for every~$n < 0$ we get a family of group homomorphisms~$f_* \defined (f_n)_{n \in \Integer}$ that is a morphism of chain complexes
      \[
        f_*
        \colon
        \Ccc^\sing(X)
        \to
        \Ccc^\sing(Y)
        \,.
      \]
      
      If~$A \subseteq X$ is a subspace then the chain complex~$\Ccc^\sing(A)$ is a subcomplex of the chain complex~$\Ccc^\sing(X)$, i.e.\ it holds that~$C^\sing_n(A) \subseteq C^\sing_n(X)$ is a subgroup for every~$n \in \Integer$, and the differential of~$\Ccc^\sing(A)$ is the ({\welldef}) restriction of the differential of~$\Ccc^\sing(X)$.
      We can hence form the quotient chain complex
      \[
                  \Ccc^\sing(X,A)
        \defined  \Ccc^\sing(X)/\Ccc^\sing(A) \,,
      \]
      that is given in components by
      \[
          C^\sing_n(X,A)
        = C^\sing_n(X)/C^\sing_n(A)
      \]
      for every~$n \in \Integer$, and where the differential~$C^\sing_n(X,A) \to C^\sing_{n-1}(X,A)$ is for every~$n \in \Integer$ induced by the differential~$C^\sing_n(X) \to C^\sing_{n-1}(X)$.
      The homology
      \[
                  \Hl^\sing_n(X,A)
        \defined  \Hl_n( \Ccc^\sing(X,A) )
      \]
      is the~\dash{$n$}{th} \emph{relative singular homology}\index{relative singular homology}\index{singular homology!relative} of~$X$ with respect to~$A$.
      
      The inclusion~$i \colon A \to X$ is an injective continuous map.
      It follows at every position~$n \in \Integer$ that the induced group homomorphism~$i_n \colon C^\sing_n(A) \to C^\sing_n(X)$ is a monomorphism, and hence overall that the induced morphism of chain complexes~$i_* \colon \Ccc^\sing(A) \to \Ccc^\sing(X)$ is a monomorphism.
      The above chain complex~$\Ccc(X,A)$ is precisely the cokernel of this morphism~$i_*$, and we therefore have a short exact sequence of chain complexes
      \begin{equation}
        \label{ses of singular cc}
        0
        \to
        \Ccc^\sing(A)
        \xlongto{i_*}
        \Ccc^\sing(X)
        \to 
        \Ccc^\sing(X,A)
        \to
        0 \,.
      \end{equation}
      We hence get the folloing long exact sequence:
      \[
        \begin{tikzcd}[row sep = small]
            {}
          & \dotsb
            \arrow{r}
            \arrow[d, phantom, ""{coordinate, name=Y}]
          & \Hl_{n+1}^\sing(X,A)
            \arrow[ dll,
                    "\del_{n+1}",
                    rounded corners,
                    to path={ -- ([xshift=2ex]\tikztostart.east)
                              |- (Y) \tikztonodes
                              -| ([xshift=-2ex]\tikztotarget.west)
                              -- (\tikztotarget)}
                  ]
          \\
            \Hl_n^\sing(A)
            \arrow{r}
          & \Hl_n^\sing(X)
            \arrow{r}
            \arrow[d, phantom, ""{coordinate, name=Z}]
          & \Hl_n^\sing(X,A)
            \arrow[ dll,
                    "\del_n",
                    rounded corners,
                    to path={ -- ([xshift=2ex]\tikztostart.east)
                              |- (Z) \tikztonodes
                              -| ([xshift=-2ex]\tikztotarget.west)
                              -- (\tikztotarget)}
                  ]
          \\
            \Hl_{n-1}^\sing(A)
            \arrow{r}
          & \dotsb
          & {}
        \end{tikzcd}
      \]
    \item
      The singular cochain complex~$\Cccc_\sing(X)$ is the dual of the singular chain complex~$\Cccc_\sing(X)$, and is hence contravariantly functorial~$X$:
      If~$Y$ is another topological space and~$f \colon X \to Y$ is a continuous map then we get an induced morphism of cochain complexes~$f^* \colon \Cccc_\sing(Y) \to \Cccc_\sing(X)$, where for every~$n \in \Integer$ the group homomorphism~$f^n \colon C^n_\sing(Y) \to C^n_\sing(X)$ is the dual homomorphism to the homomorphism~$f_n \colon C_n^\sing(X) \to C_n^\sing(Y)$ as constructed above.
      
      If~$A \subseteq X$ is a subspace with inclusion~$i \colon A \to X$ then the three chain complexes~$\Ccc^\sing(X)$,~$\Ccc^\sing(A)$ and~$\Ccc^\sing(X,A)$ consist of free abelian groups.
      We hence get by dualizing the short exact sequence~\eqref{ses of singular cc} again a short exact sequence%
      \footnote{Every short exact sequence~$0 \to X' \to X \to X'' \to 0$ of free abelian groups is already a split short exact sequence.
      Its dual is then again a split short exact sequence.}
      \begin{equation}
        \label{ses of singular ccc}
        0
        \to
        \Cccc_\sing(X,A)
        \to
        \Cccc_\sing(X)
        \xlongto{i^*}
        \Cccc_\sing(A)
        \to
        0 \,.
      \end{equation}
      The cochain complex~$\Cccc_\sing(X,A)$ is the dual of the chain complex~$\Ccc^\sing(X,A)$, i.e.\ it is given in components by
      \[
                  C^n_\sing(X,A)
        \defined  \Hom_\Integer(C_n^\sing(X,A), \Integer)
      \]
      for every~$n \in \Integer$, and its differential~$C^n_\sing(X,A) \to C^{n+1}_\sing(X,A)$ is the dual homomorphism to the differential~$C_{n+1}^\sing(X,A) \to C_n^\sing(X,A)$.%
      \footnote{It follows from the exactness of the sequence~\eqref{ses of singular ccc} that~$\Cccc_\sing(X,A)$ is a kernel of the morphism~$f^*$, hence we could have also defined~$\Cccc_\sing(X,A)$ in this way.}
      
      We now get from the short exact sequence~\eqref{ses of singular ccc} the following long exact sequence:
      \[
        \begin{tikzcd}[row sep = small]
            {}
          & \dotsb
            \arrow{r}
            \arrow[d, phantom, ""{coordinate, name=Y}]
          & \Hl^{n-1}_\sing(A)
            \arrow[ dll,
                    "\del^{n-1}",
                    rounded corners,
                    to path={ -- ([xshift=2ex]\tikztostart.east)
                              |- (Y) \tikztonodes
                              -| ([xshift=-2ex]\tikztotarget.west)
                              -- (\tikztotarget)}
                  ]
          \\
            \Hl^n_\sing(X,A)
            \arrow{r}
          & \Hl^n_\sing(X)
            \arrow{r}
            \arrow[d, phantom, ""{coordinate, name=Z}]
          & \Hl^n_\sing(A)
            \arrow[ dll,
                    "\del^n",
                    rounded corners,
                    to path={ -- ([xshift=2ex]\tikztostart.east)
                              |- (Z) \tikztonodes
                              -| ([xshift=-2ex]\tikztotarget.west)
                              -- (\tikztotarget)}
                  ]
          \\
            \Hl^{n+1}_\sing(X,A)
            \arrow{r}
          & \dotsb
          & {}
        \end{tikzcd}
      \]
  \end{enumerate}
\end{example}




