\section{Injectives in Module Categories}


\begin{conventionnonum}
  Let~$\kf$ be a commutative ring and let~$A$ be a~{\kalg}.
\end{conventionnonum}


\begin{theorem}[Baer’s criterion]
  \label{baers criterion}
  For an~{\module{$A$}}~$M$ the following conditions are equivalent:
  \begin{enumerate}
    \item
      The module~$M$ is injective.
    \item
      \label{Baer condition}
      For every left ideal~$J \subseteq A$, every module homomorphism~$J \to M$ extends to a module homomorphism~$A \to M$.
      \[
        \begin{tikzcd}
            J
            \arrow[hook]{r}
            \arrow{d}
          & A
            \arrow[dashed]{dl}
          \\
            M
          & {}
        \end{tikzcd}
      \]
  \end{enumerate}
\end{theorem}


\begin{remark*}
  Baer’s criterion states that it suffices to prove the defining condition of an injective module for inclusions~$J \to A$ of left ideals~$J \subseteq A$.
\end{remark*}


\begin{proof}[Proof of Baer’s criterion]
  If~$M$ is injective then condition~\ref*{Baer condition} holds because the inclusion~$J \to A$ is a monomorphism.
  
  Suppose on the other hand that condition~\ref*{Baer condition} holds.
  Let~$N$ be any~{\module{$A$}},~$N' \subseteq N$ a submodule and~$f' \colon N' \to M$ a module homomorphism.
  We need to show that there exists a module homomorphism~$f \colon N \to M$ with~$\restrict{f}{N'} = f'$.
  
  It follows from Zorn’s lemma that there exists a maximal extension~$\overline{f}$ of~$f$.
  More explicitly, there exists a submodule~$\overline{N} \subseteq M$ and a module homomorphism~$\overline{f} \colon \overline{N} \to M$ with~$N' \subseteq \overline{N}$ and~$\restrict{\overline{f}}{N'} = f'$, such that~$\overline{N}$ is maximal with this property (with respect to the inclusion of submodules).
  
  Suppose that~$\overline{N} \neq N$.
  Then let~$y \in N$ with~$y \notin \overline{N}$ and consider the left ideal
  \[
    J
    \defined
    \{
      a \in A
    \suchthat
      a y \in \overline{N}
    \}  \,.
  \]
  The map
  \[
    g
    \colon
    J
    \to
    M \,,
    \quad
    a
    \mapsto
    \overline{f}(a y)
  \]
  is a module homomorphism and can therefore (by assumption) be extended to a module homomorphism
  \[
    \overline{g}
    \colon
    A
    \to
    M \,.
  \]
  For~$N'' \defined \overline{N} + A y$ the map
  \[
    f''
    \colon
    N''
    \to
    M \,,
    \quad
    x + a y
    \mapsto
    \overline{f}(x) + \overline{g}(a)
  \]
  where ~$x \in \overline{N}$ and~$a \in A$ is a~{\welldef} module homomorphism that extends~$f$.
  To see that~$f''$ is~{\welldef}, suppose that~$x + ay = x' + a'y$ for some~$x, x' \in \overline{N}$ and~$a, a' \in A$.
  Then
  \[
    \overline{N}
    \ni
    x - x'
    =
    (a' - a)y
  \]
  and therefore~$a' - a \in J$;
  then
  \[
      \overline{f}(x) - \overline{f}(x')
    =
    \overline{f}(x - x')
    =
    \overline{f}( (a' - a) y )
    =
    \overline{g}(a' - a)
    =
    \overline{g}(a') - \overline{g}(a)
  \]
  and therefore
  \[
    \overline{f}(x) + \overline{g}(a)
    =
    \overline{f}(x') + \overline{g}(a') \,.
  \]
  But~$\overline{N}$ is a proper submodule of~$N''$, which contradicts the maximality of~$\overline{N}$.
\end{proof}


\begin{definition*}
  An~{\module{$A$}}~$M$ is \emph{divisible}\index{divisible} if there exists for every~$m \in M$ and~$a \in A$ with~$a \neq 0$ some~$m' \in M$ such that~$m = a m'$.
\end{definition*}


\begin{corollary}
  \label{injective iff divisible}
  An abelian group~$M$ is injective in~$\Ab$ if and only if it is divisible.
\end{corollary}


\begin{proof}
  If~$M$ is injective then for every~$n \in \Integer$ with~$n \neq 0$ the linear map
  \[
    f
    \colon
    n\Integer
    \to
    M \,,
    \quad
    k
    \mapsto
    \frac{k}{n} a
  \]
  extends to a linear map~$g \colon \Integer \to M$.
  Then
  \[
    n g(1)
    =
    g(n)
    =
    f(n)
    =
    a \,.
  \]
  
  Suppose on the other hand that~$M$ is divisible.
  We use \hyperref[baers criterion]{Baer’s criterion} to show that~$M$ is injective:
  Let~$J \subseteq \Integer$ be an ideal and let~$f \colon J \to M$ be a group homomorphism;
  we need to show that~$f$ extends to a group homomorphism~$\Integer \to M$.
  The ideal~$J$ is of the form~$J = (n)$ for some~$n \in \Integer$.
  If~$J = 0$ then the zero morphism~$\Integer \to M$ is such an extension.
  Otherwise,~$n \neq 0$ and it follows for the element~$f(n) \in M$ from the divisibility of~$M$ that there exists some~$x \in M$ with~$f(n) = n x$.
  The group homomorphism
  \[
    g
    \colon
    \Integer
    \to
    M \,,
    \quad
    k
    \mapsto
    k x
  \]
  then satisfies
  \[
    g(n)
    =
    n x
    =
    f(n)
  \]
  and is therefore an extension on~$f$.
\end{proof}


\begin{remark}
  \Cref{injective iff divisible} holds more generally when~$\Integer$ is replaced by any~PID.
\end{remark}


\begin{example}
  \leavevmode
  \begin{enumerate}
    \item
      The abelian group~$\Integer$ is no injective.
    \item
      The abelian group~$\Rational$ is injective.
    \item
      The abelian group~$\Rational/\Integer$ is injective.
  \end{enumerate}
\end{example}


\begin{lemma}
  \label{product of injectives}
  If~$(I_\beta)_{\beta \in B}$ is a family of injective objects in a category~$\Ccat$ then the product~$\prod_{\beta \in B} I_\beta$ (if it exists) is again injective.
\end{lemma}


\begin{corollary}
  \label{Ab has enough injectives}
  The category~$\Ab$ has enough injectives.
\end{corollary}


\begin{proof}
  Let~$M$ be an abelian group and let
  \[
    I
    \defined
    \prod_{f \in \Hom_\Integer(M, \Rational/\Integer)} \Rational/\Integer \,.
  \]
  The abelian group~$I$ again injective by \cref{product of injectives}.
  Let~$i \colon M \to I$ be the group homomorphism that is given in the~\dash{$f$}{th} coordinate (for~$f \in \Hom_\Integer(M, \Rational/\Integer)$) by~$f$, i.e.,
  \[
    i(x)
    =
    ( f(x) )_{f \in \Hom_\Integer(M, \Rational/\Integer)} \,.
  \]
  There exists by the upcoming \cref{nonzero homomorphism into QZ} for every~$x \in M$ with~$x \neq 0$ a group homomorphism~$f \in \Hom_\Integer(M, \Rational/\Integer)$ with~$f(x) \neq 0$.
  This shows that~$i(x) \neq 0$, which shows that~$i$ is injective.
\end{proof}


\begin{lemma}
  \label{nonzero homomorphism into QZ}
  Let~$M$ be an abelian group.
  There exists for every~$x \in M$ with~$x \neq 0$ a group homomorphism~$f \in \Hom_\Integer(M, \Rational/\Integer)$ with~$f(x) \neq 0$.
\end{lemma}


\begin{proof}
  We may assume that~$M = \Integer x$ because~$\Rational/\Integer$ is injective, and every such homomorphism~$\Integer x \to \Rational/\Integer$ therefore extends to a homomorphism~$M \to \Rational/\Integer$.
  The group~$\Integer x$ is cyclic, and so we may assume that~$M = \Integer/n$ for some~$n \geq 0$ and that~$x = \class{1}$.
  In the case~$n = 0$ we can use the projection~$\Integer \to \Integer/2$  to assume that~$n > 0$.
  We can now use the embedding~$\Integer/n \to \Rational/\Integer$ given by~$\class{1} \to 1/n$ as a suitable morphism.
\end{proof}


\begin{theorem}
  \label{adjunctions between abelian categories}
  Let~$\Acat$ and~$\Bcat$ be abelian categories and let~$(F, G, \varphi)$ be an adjunction from~$\Acat$ to~$\Bcat$
  (i.e.,~$F \colon \Acat \to \Bcat$ and~$G \colon \Bcat \to \Acat$ and~$F$ is left adjoint to~$G$).
  \begin{enumerate}
    \item
      The functors~$F$ and~$G$ are additive, and the map~$\varphi_{X,Y}$ is for all objects~$X \in \Ob(\Acat)$ and~$Y \in \Ob(\Bcat)$ an isomorphism of abelian groups.
    \item
      The functor~$F$ is right exact, whereas the functor~$G$ is left exact.
    \item
      \label{exact preserve inj and proj}
      If~$F$ is exact and~$I$ is an injective object of~$\Bcat$ then~$G(I)$ is an injective object of~$\Acat$.
      If~$G$ is exact and~$P$ is a projective object of~$\Acat$ then~$F(P)$ is a projective object of~$\Bcat$.
  \end{enumerate}
\end{theorem}


\begin{proof}
  \leavevmode
  \begin{enumerate}
    \item
      This is part~(ii) of Exercise~4 of Exercise sheet~13.
    \item
      This is part~(iii) of Exercise~4 of Exercise sheet~13.
    \item
      We show the first assertion, the second assertion then follows by duality.
      We show that~$\Hom_\Acat(-,G(I))$ maps monomorphisms to surjections.
      So let~$f \colon X' \to X$ be a monomorphism in~$\Acat$.
      We get the following commutative square:
      \[
        \begin{tikzcd}[column sep = large]
            \Hom_\Acat(X, G(I))
            \arrow{r}[above]{f^*}
          & \Hom_\Acat(X', G(I))
          \\
            \Hom_\Bcat(F(X), I)
            \arrow{u}[left]{\varphi_{X,Y}}
            \arrow{r}[below]{F(f)^*}
          & \Hom_\Bcat(F(X'), I)
            \arrow{u}[right]{\varphi_{X',Y}}
        \end{tikzcd}
      \]
      The functor~$F$ respects kernels because it is left exact, and hence respect monomorphisms.
      We therefore find that the morphism~$F(f)$ is again a monomorphism.
      It follows from the injectivity of the object~$I$ that~$F(f)^*$ is surjective.
      Therefore,~$f^* = \varphi_{X', Y} \circ F(f)^* \circ \varphi_{X,Y}^{-1}$ is surjective.
    \qedhere
  \end{enumerate}
\end{proof}


\begin{corollary}
  If~$I$ is an injective abelian group then~$\Hom_\Integer(A,I)$ is an injective left~{\module{$A$}}, where~$A$ acts on~$\Hom_\Integer(A,I)$ via
  \[
    (a.f)(a')
    =
    f(a' a)
  \]
  for all~$a \in A$ and all~$f \in \Hom_\Integer(A,I)$ and~$a' \in A$.
  Similarly for right~{\modules{$A$}}.
\end{corollary}


\begin{proof}
  The forgetful functor~$F \colon \Modl{A} \to \Ab$ is exact and has a right adjoint
  \[
    G
    \colon
    \Ab
    \to
    \Modl{A} \,,
    \quad
    M
    \mapsto
    \Hom_\Integer(A,M)  \,.
  \]
  It follows from part~\ref*{exact preserve inj and proj} of \cref{adjunctions between abelian categories} that~$G$ respects injectives.
  It therefore follows from the injectivity of~$I$ that~$G(I) = \Hom_\Integer(A,I)$ is again injective.
\end{proof}


\begin{example*}
  The~{\module{$A$}}~$\Hom_\Integer(A, \Rational/\Integer)$ is injective.
\end{example*}


\begin{lemma*}
  Let~$M$ be an~{\module{$A$}}.
  Then there exists for every~$x \in M$ with~$x \neq 0$ a module homomorphism~$f \colon M \to \Hom_\Integer(A, \Rational/\Integer)$ with~$f(x) \neq 0$.
\end{lemma*}


\begin{proof}
  We may assume that~$M = A x$ because by the injectivity of~$\Hom_\Integer(M, \Rational/\Integer)$ every such module homomorphism~$Ax \to \Hom_\Integer(A, \Rational/\Integer)$ can be extended to a module homomorphism~$M \to \Hom_\Integer(A, \Rational/\Integer)$.
  We may further assume that~$M = A/J$ for a left ideal~$J \subseteq A$ and that~$x = \class{1}$.
  It follows from~$x \neq 0$ by \cref{nonzero homomorphism into QZ} that there exists some~$h \in \Hom_\Integer(A/J, \Rational/\Integer)$ with~$h(\class{1}) \neq 0$.
  Let~$g \in \Hom_\Integer(A, \Rational/\Integer)$ be the composition
  \[
    g
    \colon
    A
    \xlongto{\pi}
    A/J
    \xlongto{h}
  \]
  where~$\pi \colon A \to A/J$ is the canonical projection.
  Let~$\tilde{f} \colon A \to \Hom_\Integer(A, \Rational/\Integer)$ be the unique module homomorphism with~$\tilde{f}(1) = g$.
  It holds for every~$y \in J$ that
  \[
    \tilde{f}(y)(a)
    =
    (y g)(a)
    =
    g(ay)
    =
    h(\class{ay})
    =
    h(0)
    =
    0
  \]
  for every~$a \in A$ because~$ay \in J$;
  hence~$\tilde{f}(y) = 0$.
  This shows that~$\tilde{f}$ factors through a {\welldef} module homomorphism~$f \colon A/J \to \Hom_\Integer(A, \Rational/\Integer)$.
  This homomorphism satisfies
  \[
    f(\class{1})
    =
    \tilde{f}(1)
    =
    g
  \]
  with~$g \neq 0$ (because~$h \neq 0$).
\end{proof}


\begin{corollary}
  The category~$\Modl{A}$ has enough injectives (and so has~$\Modr{A}$).
\end{corollary}


\begin{proof}
    The proof proceeds analogous to \cref{Ab has enough injectives}, with the role of~$\Rational/\Integer$ replaced by~$\Hom_\Integer(A, \Rational/\Integer)$.
\end{proof}





\lecturend{25}


\begin{remark}
  We can for every~{\module{$A$}}~$\indmodule[A]{M}$ form the right derived functors of the left exact functor~$\Hom_A(M, -) \colon \Modl{A} \to \Modl{\kf}$.
\end{remark}




