\section{\texorpdfstring{$\Ext^1$}{Ext 1}}


\begin{conventionnonum}
  Let~$\Acat$ be an abelian category for which the class of isomorphism classes~$\Ob(\Acat)/{\cong}$ is actually a set.
\end{conventionnonum}


\begin{remarkdefinition}
  Let~$X$ and~$Y$ be two objects in~$\Acat$.
  \begin{enumerate}
    \item
      We denote by~$\Extensions(X,Y)$ the class
      \[
        \Extensions(X,Y)
        \defined
        \left\{
          \xi
          =
          (a, E, b)
        \suchthat*
          \begin{array}{c}
            E \in \Ob(\Acat),
            \\
            a \colon Y \to E, \;
            b \colon E \to X,
            \\
            \text{$0 \to Y \xto{a} E \xto{b} X \to 0$ is exact}
          \end{array}
        \right\}  \,.
      \]
    \item
      Two such sequences~$\xi, \xi' \in \Extensions(X,Y)$ given by~$\xi = (a,E,b)$ and~$\xi' = (a', E', b')$  are \emph{equivalent} if there exists a morphism~$\varphi \colon E \to E'$ that makes the resulting diagram
      \[
        \begin{tikzcd}
            0
            \arrow{r}
          & Y
            \arrow{r}[above]{a}
            \arrow[equal]{d}
          & E
            \arrow{r}[above]{b}
            \arrow[dashed]{d}[right]{\varphi}
          & X
            \arrow{r}
            \arrow[equal]{d}
          & 0
          \\
            0
            \arrow{r}
          & Y
            \arrow{r}[above]{a'}
          & E'
            \arrow{r}[above]{b'}
          & X
            \arrow{r}
          & 0
        \end{tikzcd}
      \]
      commute.
      That~$\xi$ is equivalent to~$\xi'$ is denoted by~$\xi \sim \xi'$.
      Note that it follows from the \hyperref[5 lemma]{5-lemma} that~$\varphi$ is an isomorphism, which shows tells us that~$\sim$ is symmetric.
      We also observe that~$\sim$ is reflexive and transitive.
      We thus find that~$\sim$ is an equivalence relation on the class~$\Extensions(X,Y)$.
    \item
      The quotient~$\Ext^1_\Acat(X,Y) \defined \Extensions(X,Y)/{\sim}$ is by assumption a set.
      An equivalence class~$[\xi] \in \Ext^1(X,Y)$ is a \emph{Yoneda extension}\index{Yoneda!extension}\index{extension!Yoneda}.
  \end{enumerate}
\end{remarkdefinition}


\begin{remark*}
   If there exists a short exact sequence~$0 \to Y \to E \to X \to 0$ then~$E$ is an \emph{extension}\index{extension} of~$X$ by~$Y$.
   The class~$\Extensions(X,Y)$ can therefore be though of as the class of extensions of~$Y$ by~$X$.
   (Hence the letter~$\Extensions$.)
\end{remark*}


\begin{remark}
  Let~$X$ and~$Y$ be objects in~$\Acat$ and let~$\xi = (a,E,b) \in \Extensions(X,Y)$.
  \begin{enumerate}
    \item
      Every morphism~$f \colon X' \to X'$ in~$\Acat$ induces a map~$f^* \colon \Ext^1(X,Y) \to \Ext^1(X',Y)$ as follows:
      
      We start with the following pullback square:
      \[
        \begin{tikzcd}
            E'
            \arrow{r}[above]{b'}
            \arrow{d}[left]{f'}
            \arrow[phantom]{dr}[description]{\pb}
          & X'
            \arrow{d}[right]{f}
          \\
            E
            \arrow{r}[below]{b}
          & X
        \end{tikzcd}
      \]
      We know from \cref{kernels of pullbacks} that there exists a unique morphism~$a' \colon Y \to E'$ that makes the resulting diagram
      \[
        \begin{tikzcd}
            0
            \arrow{r}
          & Y
            \arrow{r}[above]{a'}
            \arrow[equal]{d}
          & E'
            \arrow{r}[above]{b'}
            \arrow{d}[right]{f'}
            \arrow[phantom]{dr}[description]{\pb}
          & X'
            \arrow{r}
            \arrow{d}[right]{f}
          & 0
          \\
            0
            \arrow{r}
          & Y
            \arrow{r}[below]{a}
          & E
            \arrow{r}[below]{b}
          & X
            \arrow{r}
          & 0
        \end{tikzcd}
      \]
      commute, and such that the rows of this diagram are (short) exact.
      Observe that~$(a', E', b') \in \Extensions(X',Y)$.
      
      We claim that this construction is compatible with equivalence.
      More explicitely, let~$\xi_1, \xi_2 \in \Extensions(X,Y)$ with~$\xi_1 \sim \xi_2$
      Then for~$\xi'_1, \xi'_2 \in \Extensions(X', Y)$ resulting from the above construction, also~$\xi'_1 \sim \xi'_2$.
      
      Indeed, let~$\xi_i = (a_i, E_i, b_i)$ and~$\xi'_i = (a'_i, E'_i, b'_i)$.
      Let~$\varphi \colon E_1 \to E_2$ be a morphism that makes the resulting diagram
      \[
        \begin{tikzcd}
            0
            \arrow{r}
          & Y
            \arrow{r}[above]{a_1}
            \arrow[equal]{d}
          & E_1
            \arrow{r}[above]{b_1}
            \arrow[dashed]{d}[right]{\varphi}
          & X
            \arrow{r}
            \arrow[equal]{d}
          & 0
          \\
            0
            \arrow{r}
          & Y
            \arrow{r}[below]{a_2}
          & E_2
            \arrow{r}[below]{b_2}
          & X
            \arrow{r}
          & 0
        \end{tikzcd}
      \]
      commute.
      We get the following commutative diagram:
      \[
        \begin{tikzcd}[column sep =  2em, cramped]
            {}
          & 0
            \arrow{rr}
          & {}
          & Y
            \arrow{rr}[above]{a'_1}
            \arrow[equal]{dd}
            \arrow[equal]{dl}
          & {}
          & E'_1
            \arrow{rr}[above]{b'_1}
            \arrow{dd}[right, very near start]{f'_1}
          & {}
          & X'
            \arrow{rr}
            \arrow{dd}[right, near start]{f}
            \arrow[equal]{dl}
          & {}
          & 0
          \\
            0
            \arrow{rr}
          & {}
          & Y
          & {}
          & E'_2
            \arrow[from=ll, crossing over, "a'_2", near end]
          & {}
          & X'
            \arrow[from=ll, crossing over, "b'_2", near end]
          & {}
          & 0
            \arrow[from=ll, crossing over]
          & {}
          \\
            {}
          & 0
            \arrow{rr}
          & {}
          & Y
            \arrow{rr}[below, near start]{a_1}
            \arrow[equal]{dl}
          & {}
          & E_1
            \arrow{rr}[below, near start]{b_1}
            \arrow{dl}[below right]{\varphi}
          & {}
          & X
            \arrow{rr}
            \arrow[equal]{dl}
          & {}
          & 0
          \\
            0
            \arrow{rr}
          & {}
          & Y
            \arrow[from=uu, crossing over, equal]
            \arrow{rr}[below]{a_2}
          & {}
          & E_2
            \arrow[from=uu, crossing over, "f'_2", near start]
            \arrow{rr}[below]{b_2}
          & {}
          & X
            \arrow[from=uu, crossing over, "f", near start]
            \arrow{rr}
          & {}
          & 0
          & {}
        \end{tikzcd}
      \]
      It follows from the \hyperref[functoriality of pullback and pushout]{functoriality of the pullback} that there exists a unique morphism~$\varphi' \colon E'_1 \to E'_2$ that makes the resulting cube
      \[
        \begin{tikzcd}[cramped]
            {}
          & E'_1
            \arrow{rr}[above]{b'_1}
            \arrow{dd}[right, very near start]{f'_1}
            \arrow[dashed]{dl}[above left]{\varphi'}
          & {}
          & X'
            \arrow{dd}[right, near start]{f}
            \arrow[equal]{dl}
          \\
            E'_2
            \arrow{dd}[right, near start]{f'_2}
          & {}
          & X'
            \arrow[from=ll, crossing over, "b'_2", near end]
          & {}
          \\
            {}
          & E_1
            \arrow{rr}[above, near start]{b_1}
            \arrow{dl}[below right]{\varphi}
          & {}
          & X
            \arrow[equal]{dl}
          \\
            E_2
            \arrow{rr}[below]{b_2}
          & {}
          & X
            \arrow[from=uu, crossing over, "f", near start]
          & {}
          \\
        \end{tikzcd}
      \]
      commute.
      It then follows that the complete diagram
      \[
        \begin{tikzcd}[column sep =  2em, cramped]
            {}
          & 0
            \arrow{rr}
          & {}
          & Y
            \arrow{rr}[above]{a'_1}
            \arrow[equal]{dd}
            \arrow[equal]{dl}
          & {}
          & E'_1
            \arrow{rr}[above]{b'_1}
            \arrow{dd}[right, very near start]{f'_1}
            \arrow[dashed]{dl}[above left]{\varphi'}
          & {}
          & X'
            \arrow{rr}
            \arrow{dd}[right, near start]{f}
            \arrow[equal]{dl}
          & {}
          & 0
          \\
            0
            \arrow{rr}
          & {}
          & Y
          & {}
          & E'_2
            \arrow[from=ll, crossing over, "a'_2", near end]
          & {}
          & X'
            \arrow[from=ll, crossing over, "b'_2", near end]
          & {}
          & 0
            \arrow[from=ll, crossing over]
          & {}
          \\
            {}
          & 0
            \arrow{rr}
          & {}
          & Y
            \arrow{rr}[below, near start]{a_1}
            \arrow[equal]{dl}
          & {}
          & E_1
            \arrow{rr}[below, near start]{b_1}
            \arrow{dl}[below right]{\varphi}
          & {}
          & X
            \arrow{rr}
            \arrow[equal]{dl}
          & {}
          & 0
          \\
            0
            \arrow{rr}
          & {}
          & Y
            \arrow[from=uu, crossing over, equal]
            \arrow{rr}[below]{a_2}
          & {}
          & E_2
            \arrow[from=uu, crossing over, "f'_2", near start]
            \arrow{rr}[below]{b_2}
          & {}
          & X
            \arrow[from=uu, crossing over, "f", near start]
            \arrow{rr}
          & {}
          & 0
          & {}
        \end{tikzcd}
      \]
      commutes:
      It remains to show that the square
      \[
        \begin{tikzcd}
            Y
            \arrow{r}[above]{a'_1}
            \arrow[equal]{d}
          & E'_1
            \arrow{d}[right]{\varphi'}
          \\
            Y
            \arrow{r}[below]{a'_2}
          & E'_2
        \end{tikzcd}
      \]
      commutes.
      We have that
      \[
        f'_2 \varphi' a'_1
        =
        \varphi f'_1 a'_1
        =
        \varphi a_1 \id_Y
        =
        a_2 \id_Y \id_Y
        =
        a_2 \id_Y \id_Y
        =
        f'_2 a'_2 \id_Y \,.
      \]
      and also that
      \[
        b'_2 \varphi' a'_1
        =
        \id_{X'} \underbrace{b'_1 a'_1}_{=0}
        =
        0
        =
        \underbrace{b'_2 a'_2}_{=0} \id_Y  \,.
      \]
      It follows from the universal property of the pullback (applied to~$E'_2$) that indeed~$\varphi' a'_1 = a'_2 \id_Y$.
      
      This shows that~$\xi'_1$ and~$\xi'_2$ are again equivalent.
      We hence get a~{\welldef}%
      \footnote{We use, without proof, that different choices of pullback give equivalent sequences.}
      map
      \[
        f^*
        \colon
        \Ext^1(X, Y)
        \to
        \Ext^1(X', Y) \,.
      \]
      For~$\class{\xi} \in \Ext^1(X,Y)$ we also write
      \[
        \class{\xi} \cdot f
        \defined
        f^*( \class{\xi} )  \,.
      \]

    \item
      Let~$g \colon Y \to Y'$ be a morphism in~$\Acat$.
      We find dually to above discussion that the morphism~$g$ induces a~{\welldef} map
      \[
        g_*
        \colon
        \Ext^1(X,Y)
        \to
        \Ext^1(X,Y')  \,,
      \]
      and we denote for~$\class{\xi} \in \Ext^1(X,Y)$ the Yoneda extension~$g_*(\class{\xi}) \in \Ext^1(X,Y')$ by~$g \cdot \class{\xi}$.
      If~$\xi = (a,E,b)$ and~$g \cdot \class{\xi} = \class{\xi'}$ then one such a representative~$\xi' = (a',E',b')$ is given by the commutative diagram
      \[
        \begin{tikzcd}
            0
            \arrow{r}
          & Y
            \arrow{r}[above]{a}
            \arrow{d}[left]{g}
            \arrow[phantom]{dr}[description]{\po}
          & E
            \arrow{r}[above]{b}
            \arrow{d}
          & X
            \arrow{r}
            \arrow[equal]{d}
          & 0
          \\
            0
            \arrow{r}
          & Y'
            \arrow{r}[below]{a'}
          & E'
            \arrow{r}[below]{b'}
          & X
            \arrow{r}
          & 0
        \end{tikzcd}
      \]
      where the left square is a pushout.
    \item
      If~$f_2 \colon X' \to X$ and~$f_1 \colon X'' \to X'$ are morphisms in~$\Acat$ then
      \[
        (f_2 \circ f_1)^*
        =
        f_1^* \circ f_2^*
      \]
      for the induced maps
      \begin{align*}
        f_2^*
        &\colon
        \Ext^1(X, Y)
        \to
        \Ext^1(X', Y) \,,
        \\
        f_1^*
        &\colon
        \Ext^1(X', Y)
        \to
        \Ext^1(X'', Y)  \,.
      \end{align*}
      Indeed, let~$\class{\xi} \in \Ext^1(X,Y)$ with~$\xi = (a,E,b)$.
      Then~$f_2^*( \class{\xi} ) = \class{\xi'}$ for a sequence~$\xi' = (a', E' ,b') \in \Extensions(X', Y)$ such that we have a commutative diagram
      \[
        \begin{tikzcd}
            0
            \arrow{r}
          & Y
            \arrow{r}[above]{a'}
            \arrow[equal]{d}
          & E'
            \arrow{r}[above]{b'}
            \arrow{d}[right]{f'_2}
            \arrow[phantom]{dr}[description]{\pb}
          & X'
            \arrow{r}
            \arrow{d}[right]{f_2}
          & 0
          \\
            0
            \arrow{r}
          & Y
            \arrow{r}[below]{a}
          & E
            \arrow{r}[below]{b}
          & X
            \arrow{r}
          & 0
        \end{tikzcd}
      \]
      in which the right square is a pullback.
      We similarly have that~$f_1^*( \class{\xi'} ) = \class{\xi''}$ for a sequence~$\xi'' = (a'', E'', b'') \in \Extensions(X'', Y)$ such that we have a commutative diagram
      \[
        \begin{tikzcd}
            0
            \arrow{r}
          & Y
            \arrow{r}[above]{a''}
            \arrow[equal]{d}
          & E''
            \arrow{r}[above]{b''}
            \arrow{d}[right]{f'_1}
            \arrow[phantom]{dr}[description]{\pb}
          & X''
            \arrow{r}
            \arrow{d}[right]{f_1}
          & 0
          \\
            0
            \arrow{r}
          & Y
            \arrow{r}[below]{a'}
          & E'
            \arrow{r}[below]{b'}
          & X'
            \arrow{r}
          & 0
        \end{tikzcd}
      \]
      in which the right square is a pullback.
      By glueing the above two diagrams together we get the following commutative diagram:
      \[
        \begin{tikzcd}
            0
            \arrow{r}
          & Y
            \arrow{r}[above]{a''}
            \arrow[equal]{d}
          & E''
            \arrow{r}[above]{b''}
            \arrow{d}[right]{f'_1}
            \arrow[phantom]{dr}[description]{\pb}
          & X''
            \arrow{r}
            \arrow{d}[right]{f_1}
          & 0
          \\
            0
            \arrow{r}
          & Y
            \arrow{r}[above]{a'}
            \arrow[equal]{d}
          & E'
            \arrow{r}[above]{b'}
            \arrow{d}[right]{f'_2}
            \arrow[phantom]{dr}[description]{\pb}
          & X'
            \arrow{r}
            \arrow{d}[right]{f_2}
          & 0
          \\
            0
            \arrow{r}
          & Y
            \arrow{r}[below]{a}
          & E
            \arrow{r}[below]{b}
          & X
            \arrow{r}
          & 0
        \end{tikzcd}
      \]
      It follow from the \hyperref[transitivity of pullback and pushout]{transitivity of pullbacks} that in the subdiagram
      \[
        \begin{tikzcd}[column sep = large]
            0
            \arrow{r}
          & Y
            \arrow{r}[above]{a''}
            \arrow[equal]{d}
          & E''
            \arrow{r}[above]{b''}
            \arrow{d}[right]{f'_2 f'_1}
            \arrow[phantom]{dr}[description]{\pb}
          & X''
            \arrow{r}
            \arrow{d}[right]{f_2 f_1}
          & 0
          \\
            0
            \arrow{r}
          & Y
            \arrow{r}[below]{a}
          & E
            \arrow{r}[below]{b}
          & X
            \arrow{r}
          & 0
        \end{tikzcd}
      \]
      the right square is again a pullback.
      We find from this diagram that
      \[
        (f_2 \circ f_1)^*( \class{\xi} 
        =
        \class{\xi''}
      \]
      and therefore
      \[
        (f_2 \circ f_1)^*( \class{\xi} )
        =
        \class{\xi''}
        =
        f_1^*( \class{\xi'} )
        =
        f_1^*( f_2^*( \class{\xi} ) )
        =
        (f_1^* \circ f_2^*)( \class{x_i} )  \,,
      \]
      as desired.
      We note this this equality can also be expressed as
      \[
        (\class{\xi} \cdot f_2) \cdot f_1
        =
        \class{\xi} \cdot (f_2 \cdot f_1) \,.
      \]

      We find similarly for all morphisms~$g_1 \colon Y \to Y'$ and~$g_2 \colon Y' \to Y''$ in~$\Acat$ that
      \[
        (g_2 \circ g_1)_*
        =
        (g_2)_* \circ (g_1)_*
      \]
      for the induced maps
      \begin{align*}
        (g_1)_*
        &\colon
        \Ext^1(X, Y)
        \to
        \Ext^1(X, Y') \,,
        \\
        (g_2)_*
        &\colon
        \Ext^1(X, Y')
        \to
        \Ext^1(X, Y'') \,.
      \end{align*}
      This equality can also be expressed as
      \[
        g_1 \cdot (g_2 \cdot \class{\xi})
        =
        (g_1 \circ g_2) \cdot \class{\xi}
      \]
      for every~$\class{\xi} \in \Ext^1(X,Y)$.
      
    \item
      We have for all morphisms~$f \colon X' \to X$ and~$g \colon Y \to Y'$ in~$\Acat$ that
      \[
        g_* \circ f^*
        =
        f^* \circ g_* \,,
      \]
      which an also be expressed as
      \[
        g \cdot (\class{\xi} \cdot f)
        =
        (g \cdot \class{\xi}) \cdot f
      \]
      for every~$\class{\xi} \in \Ext^1(X,Y)$:
      We consider the following commutative diagram:
      \[
        \begin{tikzcd}
            g_* f^* \xi:
          & 0
            \arrow{r}
          & Y'
            \arrow{r}
          & \widetilde{E}
            \arrow{r}
          & X'
            \arrow{r}
          & 0
          \\
            f^* \xi:
          & 0
            \arrow{r}
          & Y
            \arrow{u}[left]{g}
            \arrow[phantom]{ur}[description]{\llcorner}
            \arrow{r}
            \arrow[equal]{d}
          & E'
            \arrow{u}[left]{\tilde{g}}
            \arrow{r}
            \arrow[phantom]{dr}[description]{\lrcorner}
            \arrow{d}[right]{f'}
          & X'
            \arrow[equal]{u}
            \arrow{r}
            \arrow{d}[right]{f}
          & 0
          \\
            \xi:
          & 0
            \arrow{r}
          & Y
            \arrow{r}
            \arrow[phantom]{dr}[description]{\ulcorner}
            \arrow{d}[left]{g}
          & E
            \arrow{r}
            \arrow{d}[left]{g''}
          & X
            \arrow{r}
            \arrow[equal]{d}
          & 0
          \\
            g^* \xi:
          & 0
            \arrow{r}
          & Y'
            \arrow{r}
          & E''
            \arrow{r}
          & X
            \arrow{r}
          & 0
          \\
            f^* g_* \xi:
          & 0
            \arrow{r}
          & Y'
            \arrow[equal]{u}
            \arrow{r}
          & \widetilde{\widetilde{E}}
            \arrow{u}[right]{\tilde{\tilde{f}}}
            \arrow[phantom]{ur}{\urcorner}
            \arrow{r}
          & X'
            \arrow{u}[right]{f}
            \arrow{r}
          & 0
        \end{tikzcd}
      \]
  \end{enumerate}
  
  
  
  
  
  \lecturend{26}
  
  
  
  
  
\end{remark}


