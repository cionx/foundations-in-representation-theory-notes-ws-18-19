\chapter{Categories and Functors}


\begin{definition}
  A \emph{category}~$\Ccat$\index{category} consists of the following data:
  \begin{itemize}
    \item
      A class~$\Ob(\Ccat)$.
      The elements are the \emph{objects}\index{objects of a category} of~$\Ccat$.
    \item
      For any two objects~$X, Y \in \Ob(\Ccat)$ a set~$\Ccat(X,Y)$.
      The elements of~$\Ccat(X, Y)$ are the \emph{morphisms}\index{morphisms in a category} from~$X$ to~$Y$.
      That~$f \in \Ccat(X,Y)$ is denoted by~$f \colon X \to Y$ or~$X \xto{f} Y$.
    \item
      For any three objects~$X, Y, Z \in \Ob(\Ccat)$ a map
      \[
                \Ccat(Y,Z) \times \Ccat(X,Y)
        \to     \Ccat(X,Z) \,,
        \quad   (g \circ f)
        \mapsto g \circ f \,.
      \]
      For any two morphisms~$f \colon X \to Y$ and~$g \colon Y \to Z$ the morphism~$g \circ f \colon X \to Z$ is the \emph{composition} of~$g$ and~$f$.
  \end{itemize}
  These data are subject to the following conditions:
  \begin{enumerate}[label=(C\arabic*)]
    \item
      The composition of morphisms is associative\index{associativity}:
      For all objects~$X, Y, Z, W \in \Ob(\Ccat)$ and morphisms~$f \colon X \to Y$,~$g \colon Y \to Z$ and~$h \colon Z \to W$ it holds that
      \[
          (h \circ g) \circ f
        = h \circ (g \circ f) \,.
      \]
    \item
      There exists for every object~$X \in \Ob(\Ccat)$ an \emph{identity morphism}\index{identity morphism}~$\id_X \colon X \to X$ such that
      \[
        f \circ \id_X = f
        \quad\text{and}\quad
        \id_X \circ g = g
      \]
      for all morphisms~$f \colon X \to Y$ and~$g \colon Y \to X$ in~$\Ccat$.
  \end{enumerate}
\end{definition}


\begin{remark}
  Let~$\Ccat$ be a category.
  \begin{enumerate}
    \item
      It could happen for objects~$X, Y \in \Ccat$ that~$\Ccat(X,Y) = \emptyset$, i.e.\ that there exists no morphism from~$X$ to~$Y$ in~$\Ccat$.
    \item
      For every object~$X \in \Ccat$ the identity morphism~$\id_X$ is unique.
      If~$\id'_X$ is another identity morphism of~$X$ then
      \[
          \id_X
        = \id_X \id'_X
        = \id'_X \,.
      \]
  \end{enumerate}
\end{remark}


\begin{remark}
  We sometimes want to consider categories whose objects are all sets (which satisfy certain conditions).
  This can lead to set theoretic problems (also known as \emph{set theoretic difficulties}\index{set theoretic difficulties}).
  One way out of this predicament are \emph{universes}\index{universe}.
  (See \cite[I.6]{Working} and \cite[3.2]{Schubert} for more details on this.)
  We will always fix a universe~$U$ and say that
  \begin{itemize}
    \item
      $X$ is a set if~$X \in U$, and that
    \item
      $X$ is a class if~$X \subseteq U$.
  \end{itemize}
\end{remark}





\lecturend{4}
