\section{Existence of Derived Functors}


\begin{goalnonum}
  We want to construct for every left exact (resp.\ right exact) functor a right (resp.\ left) derived functor.
\end{goalnonum}


\begin{conventionnonum}
  Let~$F \colon \Acat \to \Bcat$ be a left exact functor between abelian categories~$\Acat$ and~$\Bcat$, where the category~$\Acat$ has enough injectives.
\end{conventionnonum}


\begin{notation*}
  If~$\Cccc$ is a cochain complex in~$\Acat$ then we denote by~$F(\Cccc)$ the cochain complex
  \[
    \dotsb
    \to
    F(C^{n+1})
    \xlongto{F(d^{n+1})}
    F(C^n)
    \xlongto{F(d^n)}
    F(C^{n-1})
    \to
    \dotsb
  \]
  in~$\Bcat$.
  If~$f \colon \Cccc \to \Dccc$ is a morphism of cochain complexes in~$\CCh(\Acat)$ then we denote by~$F(f)$ the morphism of cochain complexes~$F(f) \to F(\Cccc) \to F(\Dccc)$ that is given by~$F(f) \defined (F(f^n))_{n \in \Integer}$.
\end{notation*}


\begin{lemma}
  \label{preparation for right derived functors}
  Let~$X \in \Ob(\Acat)$ be an object in~$\Acat$.
  Let~$(\Iccc, i^0)$ be an injective resolution of~$X$, and set
  \[
    (\Right^n_{(\Iccc,i^0)}F)(X)
    \defined
    \Hl^n( F(\Iccc) )
  \]
  for every~$n \geq 0$.
  \begin{enumerate}
    \item
      \label{induced morphism in right derived}
      Let~$f \colon X \to Y$ be a morphism in~$\Acat$ and let~$(\Jccc, j^0)$ be an injective resolution of~$Y$.
      Then for any two morphisms of cochain complexes~$\hat{f}, \hat{f}' \colon \Iccc \to \Jccc$ that extend the morphism~$f$, it holds that~$\Hl^n(F(\hat{f})) = \Hl^n(F(\hat{f}'))$  for every~$n \geq 0$.
  \end{enumerate}
  Let~$(\Right^n_{(\Iccc,i^0), (\Jccc,j^0)} F)(f) \defined \Hl^n(F(\hat{f}))$ in the above situation.
  \begin{enumerate}[resume]
    \item
      \label{functoriality of morphism between derived functors}
      For any two composable morphisms~$f \colon X \to Y$ and~$g \colon Y \to Z$ in~$\Acat$ and injective resolutions~$(\Jccc, j^0)$ of~$Y$ and~$(\Kccc,k^0)$ of~$Z$ it holds for every~$n \geq 0$ that
      \[
        (\Right^n_{(\Jccc,j^0), (\Kccc,k^0)} F)(g)
        \circ
        (\Right^n_{(\Iccc,i^0), (\Jccc,j^0)} F)(f)
        =
        (\Right^n_{(\Iccc,i^0), (\Kccc,k^0)} F)(g \circ f)  \,.
      \]
    \item
      If~$(\bp\Iccc, \bp i^0)$ is another injective resolution of~$X$ then~$(\Right^n_{(\Iccc,i^0), (\bp\Iccc, \bp i^0)} F)(\id_X)$ is an isomorphism for every~$n \geq 0$.
    \item
      If furthemore~$f \colon X \to Y$ is a morphism in~$\Acat$  and~$(\Jccc, j^0)$ and~$(\bp\Jccc, \bp j^0)$ are two injective resolutions of~$Y$, then the following square commutes:
      \[
        \begin{tikzcd}[column sep = 11em, row sep = 6em]
            (\Right^n_{(\Iccc,i^0)} F)(X)
            \arrow{r}[above]{ (\Right^n_{(\Iccc,i^0), (\Jccc,j^0)} F)(f) }
            \arrow{d}[right, near start]{ (\Right^n_{(\Iccc,i^0), (\bp\Iccc, \bp i^0)} F)(\id_X) }
          & (\Right^n_{(\Jccc,j^0)} F)(Y)
            \arrow{d}[left, near end]{ (\Right^n_{(\Jccc,j^0), (\bp\Jccc, \bp j^0)} F)(\id_Y) }
          \\          
            (\Right^n_{(\bp\Iccc, \bp i^0)} F)(X)
            \arrow{r}[below]{ (\Right^n_{(\bp\Iccc, \bp i^0), (\bp\Jccc, \bp j^0)} F)(f) }
          & (\Right^n_{(\bp \Jccc, \bp j^0)} F)(Y)
        \end{tikzcd}
      \]
  \end{enumerate}
\end{lemma}


\begin{proof}
  \leavevmode
  \begin{enumerate}
    \item
      It follows from the comparison theorem that any two such extensions~$\hat{f}$ and~$\hat{f}'$ are homotopic.
      Then the morphisms~$F(\hat{f})$ and~$F(\hat{f}')$ is are also homotopic, and hence induce the same morphism in homology.
    \item
      If~$\hat{f} \colon \Iccc \to \Jccc$ is a lift of~$f$ and~$\hat{g} \colon \Jccc \to \Kccc$ is a lift of~$g$, then the composition~$\hat{g} \circ \hat{f} \colon \Iccc \to \Kccc$ is a lift of the composition~$g \circ f \colon X \to Z$.
      Hence
      \begin{align*}
         &{}  (\Right^n_{(\Iccc,i^0), (\Kccc,k^0)} F)(g \circ f)  \\
        =&{}  \Hl^n( F(\hat{g} \circ \hat{f}) ) \\
        =&{}  \Hl^n( F(\hat{g}) \circ F(\hat{f}) )  \\
        =&{}  \Hl^n( F(\hat{g}) ) \circ \Hl^n( F(\hat{f}) ) \\
        =&{}  (\Right^n_{(\Jccc,j^0), (\Kccc,k^0)} F)(g)
              \circ
              (\Right^n_{(\Iccc,i^0), (\Jccc,j^0)} F)(f)  \,.
      \end{align*}
    \item
      It holds that
      \[
        (\Right^n_{(\Iccc,i^0), (\Iccc, i^0)} F)(\id_X) 
        =
        \id_{(\Right^n_{(\Iccc,i^0)} F)(X)} \,.
      \]
      It hence follows from part~\ref*{functoriality of morphism between derived functors} that the morphism~$(\Right^n_{(\Iccc,i^0), (\bp\Iccc, \bp i^0)} F)(\id_X)$ is an isomorphism with
      \[
          (\Right^n_{(\Iccc,i^0), (\bp\Iccc, \bp i^0)} F)(\id_X)^{-1}
        = (\Right^n_{(\bp\Iccc, \bp i^0), (\Iccc,i^0)} F)(\id_X) \,.
      \]
    \item
      This follows from part~\ref*{functoriality of morphism between derived functors}.
    \qedhere
  \end{enumerate}
\end{proof}





\lecturend{20}





\begin{remarkdefinition}
  For every object~$X \in \Ob(\Acat)$ we fix an injective resolution~$(\Iccc_X, i^0_X)$ of~$X$.
  We define for every~$n \geq 0$ a functor~$\Right^n F \colon \Acat \to \Bcat$ by
  \[
    (\Right^n F)(X)
    \defined
    \Hl^n( F(\Iccc_X) )
  \]
  for every object~$X \in \Ob(\Acat)$, and
  \[
    (\Right^n F)(f)
    \defined
    \Hl^n(F(\hat{f}))
  \]
   and every morphism~$f \colon X \to Y$ in~$\Acat$, where~$\hat{f} \colon \Iccc_X \to \Iccc_Y$ is an extension of~$f$.
  This is a {\welldef} functor by parts~\ref*{induced morphism in right derived} and~\ref*{functoriality of morphism between derived functors} of \cref{preparation for right derived functors}.
  The functor~$\Right^n F$ is up to isomorphism independent of the choice of injective resolutions, again by \cref{preparation for right derived functors}.
\end{remarkdefinition}


\begin{lemma}
  The functors~$\Right^n F \colon \Acat \to \Bcat$ with~$n \geq 0$ are additive.
\end{lemma}


\begin{proof}
  Let~$f, g \colon X \to Y$ be two parallel morphisms in~$\Acat$ and let~$\hat{f}, \hat{g} \colon \Iccc_X \to \Iccc_Y$ be lifts of~$f$ and~$g$ to morphisms of chain complexes.
  Then~$\hat{f} + \hat{g}$ is a lift of the morphism~$f + g$.
  It follows that
  \begin{align*}
        (\Right^n F)(f + g)
    &=  \Hl^n(F(\hat{f}+\hat{g})) \\
    &=  \Hl^n(F(\hat{f})) + \Hl^n(F(\hat{g})) \\
    &=  (\Right^n F)(f) + (\Right^n F)(g) \,,
  \end{align*}
  as desired.
\end{proof}


\begin{lemma}
  It holds that~$\Right^0 F \cong F$.
\end{lemma}


\begin{proof}
  It follows for every object~$X \in \Ob(\Acat)$ from the exactness of the sequence
  \[
    0
    \to
    X
    \to
    I^0_X
    \to
    I^1_X
    \to
    I^2_X
    \to
    \dotsb
  \]
  that the sequence
  \[
    0
    \to
    F(X)
    \to
    F(I^0_X)
    \to
    F(I^1_X)
  \]
  is again exact, because the functor~$F$ is left exact.
  This exactness means that the morphism~$F(X) \to F(I^0_X)$ is a kernel of the morphism~$F(I^0_X) \to F(I^1_X)$.
  The chain complex~$F(\Iccc_X)$ is given by
  \[
    \dotsb
    \to
    0
    \to
    F(I^0_X)
    \to
    F(I^1_X)l
    \to
    F(I^2_X)
    \to
    \dotsb
  \]
  The zeroeth cohomology of this cochain complex, which is~$(\Right^0 F)(X)$, is also a kernel of the morphism~$F(I^0_X) \to F(I^1_X)$.
  It follows that~$F(X) \cong (\Right^0 F)(X)$.
  More precisely, there exists a unique morphism~$\lambda_X \colon (\Right^0 F)(X) \to F(X)$ that makes the square
  \[
    \begin{tikzcd}
        (\Right^0 F)(X)
        \arrow{r}[above]{j_X}
        \arrow{d}[left]{\lambda_X}
      & F(I^0_X)
        \arrow[equal]{d}
      \\
        F(X)
        \arrow{r}[below]{F(i^0_X)}
      & F(I^0_X)
    \end{tikzcd}
  \]
  commute, and the morphism~$\lambda_X$ is an isomorphism.
  
  We need to show that this isomorphism is natural in~$X$.
  For this, we consider for a morphism~$f \colon X \to Y$ in~$\Acat$ and an extension~$\hat{f} \colon \Iccc_X \to \Iccc_Y$ of~$f$ the following cube:
  \[
    \begin{tikzcd}[row sep = large, column sep = small]
        (\Right^0 F)(X)
        \arrow{rr}[above, near start]{j_X}
        \arrow{dr}[above right]{(\Right^0 F)(f)}
        \arrow{dd}[left, near start]{\lambda_X}
      & {}
      & F(I^0_X)
        \arrow{dr}[above right]{F(\hat{f}^0)}
        \arrow[equal]{dd}
      & {}
      \\
        {}
      & (\Right^0 F)(Y)
      & {}
      & F(I^0_Y)
        \arrow[from=ll, crossing over, "j_Y", near start]
        \arrow[equal]{dd}
      \\
        F(X)
        \arrow{rr}[above, near end]{F(i^0_X)}
        \arrow{dr}[below left]{F(f)}
      & {}
      & F(I^0_X)
        \arrow{dr}[below left]{F(\hat{f}^0)}
      & {}
      \\
        {}
      & F(Y)
        \arrow[from=uu, crossing over, "\lambda_Y", swap, near start]
        \arrow{rr}[below, near end]{F(i^0_Y)}
      & {}
      & F(I^0_Y)
    \end{tikzcd}
  \]
  This cube commutes:
  The front and back squares commute by choice of the isomorphisms~$\lambda_X$ and~$\lambda_Y$.
  The top square commutes by construction of~$(\Right^0 F)(f)$.
  The bottom square results by applying the functor~$F$ to the following commutative square:
  \[
    \begin{tikzcd}
        X
        \arrow{r}[above]{i^0_X}
        \arrow{d}[left]{f}
      & I^0_X
        \arrow{d}[right]{\hat{f}}
      \\
        Y
        \arrow{r}[below]{i^0_Y}
      & I^0_Y
    \end{tikzcd}
  \]
  The commutativity of the left square follows from the commutativity of the other sides and~$F(i^0_Y)$ being a monomorphism (which holds because it is a kernel):
  It holds that
  \begin{align*}
        F(i^0_Y) \lambda_Y (\Right^0 F)(f)
    &=  \id_{F(I^0_Y)} j_Y (\Right^0 F)(f)  \\
    &=  \id_{F(I^0_Y)} F(\hat{f}^0) j_X \\
    &=  F(\hat{f}^0) \id_{F(I^0_X)} j_X \\
    &=  F(\hat{f}^0) F(i^0_X) \lambda_X \\
    &=  F(i^0_Y) F(f) \lambda_X \,,
  \end{align*}
  and hence
  \[
      \lambda_Y (\Right^0 F)(f)
    = F(f) \lambda_X  \,.
  \]
  This commutativity of the left square
  \[
    \begin{tikzcd}[column sep = huge]
        (\Right^0 F)(X)
        \arrow{r}[above]{(\Right^0 F)(f)}
        \arrow{d}[left]{\lambda_X}
      & (\Right^0 F)(Y)
        \arrow{d}[right]{\lambda_Y}
      \\
        F(X)
        \arrow{r}[below]{F(f)}
      & F(Y)
    \end{tikzcd}
  \]
  shows that the family~$\lambda = (\lambda_X)_{X \in \Ob(\Acat)}$ is a natural isomorphism~$\Right^0 F \to F$.
\end{proof}


\begin{goalnonum}
  We want to show that the family~$(\Right^n F)_{n \geq 0}$ of functors~$\Right^n F \colon \Acat \to \Bcat$ can be made into a universal cohomlogical~{\deltafun}.
\end{goalnonum}


\begin{lemmanonum}
  Let
  \[
    \begin{tikzcd}[column sep = 2em]
        {}
      & {}
      & {}
      & X'
        \arrow{rr}[above, near start]{a}
        \arrow{dl}[above left]{i'}
        \arrow{dd}[right, near start]{f'}
      & {}
      & X
        \arrow{rr}[above, near start]{b}
        \arrow{dl}[above left]{i}
        \arrow{dd}[right, near start]{f}
      & {}
      & X''
        \arrow{rr}
        \arrow{dl}[above left]{i''}
        \arrow{dd}[right, near start]{f''}
      & {}
      & 0
      \\
        0
        \arrow{rr}
      & {}
      & I'
        \arrow{dd}[right, near start]{g'}
      & {}
      & I
        \arrow[from=ll, crossing over, "\alpha", near end]
      & {}
      & I''
        \arrow[from=ll, crossing over, "\beta", near end]
      & {}
      & 0
        \arrow[from=ll, crossing over]
      & {}
      \\
        {}
      & {}
      & {}
      & Y'
        \arrow{rr}[above, near start]{c}
        \arrow{dl}[below right]{j'}
      & {}
      & Y
        \arrow{rr}[above, near start]{d}
        \arrow{dl}[below right]{j}
      & {}
      & Y''
        \arrow{dl}[below right]{j''}
      & {}
      & {}
      \\
        0
        \arrow{rr}
      & {}
      & J'
        \arrow{rr}[above, near end]{\gamma}
      & {}
      & J
        \arrow{rr}[above, near end]{\delta}
      & {}
      & J''
        \arrow[from=uu, crossing over, "g''", near start]
        \arrow{rr}
      & {}
      & 0
      & {}
    \end{tikzcd}
  \]
  be a commutative diagram in~$\Acat$ with exact rows, such that the objects~$I'$ and~$J'$ are injective, and the diagonal morphism~$i''$ is a monomorphism.%
  \footnote{One should think about the diagonal morphisms in the diagram as starting points of injective resolutions.}
  Then there exists a morphism~$g'' \colon I \to J$ that makes the resulting diagram commute.
\end{lemmanonum}


\begin{proof}
  The frontal short exact sequences
  \[
    0 
    \to
    I'
    \to
    I
    \to
    I''
    \to
    0
    \qquad\text{and}\qquad
    0
    \to
    J'
    \to
    J
    \to
    J''
    \to
    0
  \]
  split because the objects~$I'$ and~$J'$ are injective.
  We may therefore assume that the given diagram is of the form
  \[
    \begin{tikzcd}[column sep = 1.8em]
        {}
      & {}
      & {}
      & X'
        \arrow{rr}[above, near start]{a}
        \arrow{dl}[above left]{i'}
        \arrow{dd}[right, near start]{f'}
      & {}
      & X
        \arrow{rr}[above, near start]{b}
        \arrow{dl}[above left]{i}
        \arrow{dd}[right, near start]{f}
      & {}
      & X''
        \arrow{rr}
        \arrow{dl}[above left]{i''}
        \arrow{dd}[right, near start]{f''}
      & {}
      & 0
      \\
        0
        \arrow{rr}
      & {}
      & I'
        \arrow{dd}[right, near start]{g'}
      & {}
      & I' \oplus I''
        \arrow[from=ll, crossing over, "\alpha", near end]
      & {}
      & I''
        \arrow[from=ll, crossing over, "\beta", near end]
      & {}
      & 0
        \arrow[from=ll, crossing over]
      & {}
      \\
        {}
      & {}
      & {}
      & Y'
        \arrow{rr}[above, near start]{c}
        \arrow{dl}[below right]{j'}
      & {}
      & Y
        \arrow{rr}[above, near start]{d}
        \arrow{dl}[below right]{j}
      & {}
      & Y''
        \arrow{dl}[below right]{j''}
      & {}
      & {}
      \\
        0
        \arrow{rr}
      & {}
      & J'
        \arrow{rr}[above, near end]{\gamma}
      & {}
      & J' \oplus J''
        \arrow{rr}[above, near end]{\delta}
      & {}
      & J''
        \arrow[from=uu, crossing over, "g''", near start]
        \arrow{rr}
      & {}
      & 0
      & {}
    \end{tikzcd}
  \]
  with
  \[
    \alpha
    =
    \begin{bmatrix}
      1 \\
      0
    \end{bmatrix} \,,
    \qquad
    \beta
    =
    \begin{bmatrix}
      0 & 1
    \end{bmatrix} \,,
    \qquad
    \gamma
    =
    \begin{bmatrix}
      1 \\
      0
    \end{bmatrix} \,,
    \qquad
    \delta
    =
    \begin{bmatrix}
      0 & 1
    \end{bmatrix} \,.
  \]
  We will define the desired morphism~$g \colon I' \oplus I'' \to J' \oplus J''$ as
  \[
    g
    =
    \begin{bmatrix}
      g'  & \varepsilon \\
          & g''
    \end{bmatrix}
  \]
  for a suitable morphism~$\varepsilon \colon I'' \to J'$.
  To contruct~$\varepsilon$, we start by defining a morphism~$ X \to J'$, that we then extend to a morphism~$X'' \to J'$, and then further extend to the desired morphism~$I'' \to J'$.
  
  Let~$\alpha' \colon I' \oplus I'' \to I'$ and~$\gamma' \colon J' \oplus J'' \to J'$ be the splits of~$\alpha$ and~$\gamma$ given by given by
  \[
    \alpha'
    \defined
    \begin{bmatrix}
      1 & 0
    \end{bmatrix}
    \quad
    \text{and}\quad
    \gamma'
    \defined
    \begin{bmatrix}
      1 & 0
    \end{bmatrix}
  \]
  We start off with the morphism~$\varepsilon'' \colon X \to J'$ given by
  \[
    \varepsilon''
    \defined
    \gamma' j f - g' \alpha' i \,.
  \]
  Then
  \begin{align*}
    \varepsilon'' a
    =
    (\gamma' j f - g' \alpha' i) a
    =
    \gamma' j f a - g' \alpha i a
    &=
    \gamma' j c f' - g' \alpha i a  \\
    &=
    \gamma' \gamma j' f' - g' \alpha' \alpha i'
    =
    j' f' - g' i'
    =
    0 \,.
  \end{align*}
  It follows that the morphism~$\varepsilon''$ factors through the morphism~$b$, because~$b$ is a cokernel of~$a$ by the exactness of the sequence
  \[
    X'
    \xlongto{a}
    X
    \xlongto{b}
    X''
    \to
    0 \,.
  \]
  There hence exists a (unique) morphism~$\varepsilon' \colon X'' \to  J'$ that makes the triangle
  \[
    \begin{tikzcd}
        X
        \arrow{r}[above]{b}
        \arrow{d}[left]{\varepsilon''}
      & X''
        \arrow[dashed]{dl}[below right]{\varepsilon'}
      \\
        J'
      & {}
    \end{tikzcd}
  \]
  commute.
  It follows from the injectivity of the object~$J'$ and~$i''$ being a monomorphism that the morphism~$\varepsilon'$ further extends to a morphism~$I'' \to J'$, i.e.\ that there exists a morphism~$\varepsilon \colon I'' \to J'$ that makes the following triangle commute:
  \[
    \begin{tikzcd}
        X''
        \arrow{r}[above]{i''}
        \arrow{d}[left]{\varepsilon'}
      & I''
        \arrow[dashed]{dl}[below right]{\varepsilon''}
      \\
        J'
      & {}
    \end{tikzcd}
  \]
  
  We now find for the morphism
  \[
    g
    \defined
    \begin{bmatrix}
      g'  & \varepsilon \\
          & g''
    \end{bmatrix}
    \colon
    I' \oplus I''
    \to
    J' \oplus J''
  \]
  that the diagram
  \[
    \begin{tikzcd}[column sep = 1.8em]
        {}
      & {}
      & {}
      & X'
        \arrow{rr}[above, near start]{a}
        \arrow{dl}[above left]{i'}
        \arrow{dd}[right, near start]{f'}
      & {}
      & X
        \arrow{rr}[above, near start]{b}
        \arrow{dl}[above left]{i}
        \arrow{dd}[right, near start]{f}
      & {}
      & X''
        \arrow{rr}
        \arrow{dl}[above left]{i''}
        \arrow{dd}[right, near start]{f''}
      & {}
      & 0
      \\
        0
        \arrow{rr}
      & {}
      & I'
        \arrow{dd}[right, near start]{g'}
      & {}
      & I' \oplus I''
        \arrow[from=ll, crossing over, "\alpha", near end]
      & {}
      & I''
        \arrow[from=ll, crossing over, "\beta", near end]
      & {}
      & 0
        \arrow[from=ll, crossing over]
      & {}
      \\
        {}
      & {}
      & {}
      & Y'
        \arrow{rr}[above, near start]{c}
        \arrow{dl}[below right]{j'}
      & {}
      & Y
        \arrow{rr}[above, near start]{d}
        \arrow{dl}[below right]{j}
      & {}
      & Y''
        \arrow{dl}[below right]{j''}
      & {}
      & {}
      \\
        0
        \arrow{rr}
      & {}
      & J'
        \arrow{rr}[above, near end]{\gamma}
      & {}
      & J' \oplus J''
        \arrow[dashed, from=uu, crossing over, "g", near start]
        \arrow{rr}[above, near end]{\delta}
      & {}
      & J''
        \arrow[from=uu, crossing over, "g''", near start]
        \arrow{rr}
      & {}
      & 0
      & {}
    \end{tikzcd}
  \]
  commutes.
  Indeed, it holds that
  \begin{gather*}
    g \alpha
    =
    \begin{bmatrix}
      g'  & \varepsilon \\
          & g''
    \end{bmatrix}
    \begin{bmatrix}
      1 \\
      0
    \end{bmatrix}
    =
    \begin{bmatrix}
      g'  \\
      0
    \end{bmatrix}
    =
    \begin{bmatrix}
      1 \\
      0
    \end{bmatrix}
    g'
    =
    \gamma g'
  \shortintertext{and}
    \delta g
    =
    \begin{bmatrix}
      0 & 1
    \end{bmatrix}
    \begin{bmatrix}
      g'  & \varepsilon \\
          & g''
    \end{bmatrix}
    =
    \begin{bmatrix}
      0 \\
      g''
    \end{bmatrix}
    =
    g''
    \begin{bmatrix}
      0 \\
      1
    \end{bmatrix}
    =
    g'' \beta \,,
  \end{gather*}
  which shows the commutativity of the frontal two squares.
  (Note that this commutativity only needs that~$g$ has an upper triangular form with diagonal entries~$g'$ and~$g''$.)
  To check that~$gi = jf$ we use the universal property of the product~$(J' \oplus J'', \gamma', \delta)$, and show that~$\delta gi = \delta jf$ and~$\gamma' gi = \gamma' jf$.
  We have that
  \[
    \delta g i
    =
    g'' \beta i
    =
    g'' i'' b
    =
    j'' f'' b
    =
    j'' {d} {f}
    =
    \delta j f \,r
  \]
  which shows the first equality.
  We also have that
  \[
    \gamma' g
    =
    \begin{bmatrix}
      1 & 0
    \end{bmatrix}
    \begin{bmatrix}
      g'  & \varepsilon \\
          & g''
    \end{bmatrix}
    =
    \begin{bmatrix}
      g'  & \varepsilon
    \end{bmatrix}
    =
    g' \alpha' + \varepsilon \beta \,,
  \]
  and therefore
  \begin{align*}
    \gamma' g i
    =
    (g' \alpha' + \varepsilon \beta) i
    &=
    g' \alpha' i + \varepsilon \beta i
    =
    g' \alpha' i + \varepsilon i'' b  \\
    &=
    g' \alpha' i + \varepsilon' b
    =
    g' \alpha' i + \varepsilon''
    =
    g' \alpha' i + \gamma' j f - g' \alpha' i
    =
    \gamma' j f \,.
  \end{align*}
  This shows the second equality.
\end{proof}


\begin{corollary*}
  \label{extension into commutative diagram}
  Let
  \[
    \begin{tikzcd}
        0
        \arrow{r}
      & \bp X
        \arrow{r}
        \arrow{d}
      & X
        \arrow{r}
        \arrow{d}
      & \bpp X
        \arrow{r}
        \arrow{d}
      & 0
      \\
        0
        \arrow{r}
      & \bp Y
        \arrow{r}
      & Y
        \arrow{r}
      & \bpp Y
        \arrow{r}
      & 0
    \end{tikzcd}
  \]
  be a commutative diagram in~$\Acat$ with (short) exact rows.
  Let
  \[
    \begin{tikzcd}[sep = small]
        {}
      & 0
        \arrow{rr}
      & {}
      & \bp X
        \arrow{rr}
        \arrow{dl}
        \arrow{dd}
      & {}
      & X
        \arrow{rr}
        \arrow{dl}
        \arrow{dd}
      & {}
      & \bpp X
        \arrow{rr}
        \arrow{dl}
        \arrow{dd}
      & {}
      & 0
      \\
        0
        \arrow{rr}
      & {}
      & \bp \Iccc
      & {}
      & \Iccc
        \arrow[from=ll, crossing over]
      & {}
      & \bpp \Iccc
        \arrow[from=ll, crossing over]
      & {}
      & 0
        \arrow[from=ll, crossing over]
      & {}
      \\
        {}
      & 0
        \arrow{rr}
      & {}
      & \bp Y
        \arrow{rr}
        \arrow{dl}
      & {}
      & Y
        \arrow{rr}
        \arrow{dl}
      & {}
      & \bpp Y
        \arrow{rr}
        \arrow{dl}
      & {}
      & 0
      \\
        0
        \arrow{rr}
      & {}
      & \bp \Jccc
        \arrow[from=uu, crossing over, "\bp f", swap, near start]
        \arrow{rr}
      & {}
      & \Jccc
        \arrow{rr}
      & {}
      & \bpp \Jccc
        \arrow[from=uu, crossing over, "\bpp f", swap, near start]
        \arrow{rr}
      & {}
      & 0
      & {}
    \end{tikzcd}
  \]
  be a resulting commutative diagram of injective resolutions with short exact rows.
  Then there exists a morphism of chain complexes~$f \colon \Iccc \to \Jccc$ that makes the resulting diagram commute.
\end{corollary*}


\begin{proof}
  We construct the required morphisms~$f^n \colon I^n \to J^n$ inductively.
  For~$n = 0$ we apply the previous lemma to the commutative diagram
  \[
    \begin{tikzcd}[sep = small]
        {}
      & 0
        \arrow{rr}
      & {}
      & \bp X
        \arrow{rr}
        \arrow{dl}
        \arrow{dd}
      & {}
      & X
        \arrow{rr}
        \arrow{dl}
        \arrow{dd}
      & {}
      & \bpp X
        \arrow{rr}
        \arrow{dl}
        \arrow{dd}
      & {}
      & 0
      \\
        0
        \arrow{rr}
      & {}
      & \bp I^0
      & {}
      & I^0
        \arrow[from=ll, crossing over]
      & {}
      & \bpp I^0
        \arrow[from=ll, crossing over]
      & {}
      & 0
        \arrow[from=ll, crossing over]
      & {}
      \\
        {}
      & 0
        \arrow{rr}
      & {}
      & \bp Y
        \arrow{rr}
        \arrow{dl}
      & {}
      & Y
        \arrow{rr}
        \arrow{dl}
      & {}
      & \bpp Y
        \arrow{rr}
        \arrow{dl}
      & {}
      & 0
      \\
        0
        \arrow{rr}
      & {}
      & \bp J^0
        \arrow[from=uu, crossing over, "\bp f^0", swap, near start]
        \arrow{rr}
      & {}
      & J^0
        \arrow{rr}
      & {}
      & \bpp J^0
        \arrow[from=uu, crossing over, "\bpp f^0", swap, near start]
        \arrow{rr}
      & {}
      & 0
      & {}
    \end{tikzcd}
  \]
  to get the desired morphism~$f^0 \colon I^0 \to J^0$.
  To construct the morphism~$f^1 \colon I^1 \to J^0$ we use that the commutative diagram
  \[
    \begin{tikzcd}[column sep = small]
        {}
      & 0
        \arrow{rr}
      & {}
      & \bp X
        \arrow{rr}
        \arrow{dl}[above left]{\bp i^0}
        \arrow{dd}
      & {}
      & X
        \arrow{rr}
        \arrow{dl}[above left]{i^0}
        \arrow{dd}
      & {}
      & \bpp X
        \arrow{rr}
        \arrow{dl}[above left]{\bpp i^0}
        \arrow{dd}
      & {}
      & 0
      \\
        0
        \arrow{rr}
      & {}
      & \bp I^0
      & {}
      & I^0
        \arrow[from=ll, crossing over]
      & {}
      & \bpp I^0
        \arrow[from=ll, crossing over]
      & {}
      & 0
        \arrow[from=ll, crossing over]
      & {}
      \\
        {}
      & 0
        \arrow{rr}
      & {}
      & \bp Y
        \arrow{rr}
        \arrow{dl}[below right]{\bp j^0}
      & {}
      & Y
        \arrow{rr}
        \arrow{dl}[below right]{j^0}
      & {}
      & \bpp Y
        \arrow{rr}
        \arrow{dl}[below right]{\bpp j^0}
      & {}
      & 0
      \\
        0
        \arrow{rr}
      & {}
      & \bp J^0
        \arrow[from=uu, crossing over, "\bp f^0", swap, near start]
        \arrow{rr}
      & {}
      & J^0
        \arrow[from=uu, crossing over, "f^0", swap, near start]
        \arrow{rr}
      & {}
      & \bpp J^0
        \arrow[from=uu, crossing over, "\bpp f^0", swap, near start]
        \arrow{rr}
      & {}
      & 0
      & {}
    \end{tikzcd}
  \]
  induces the following commuative diagram:
  \[
    \begin{tikzcd}
        \coker(\bp i^0)
        \arrow{r}
        \arrow{d}
      & \coker(i^0)
        \arrow{r}
        \arrow{d}
      & \coker(\bpp i^0)
        \arrow{d}
      \\
        \coker(\bp j^0)
        \arrow{r}
      & \coker(j^0)
        \arrow{r}
      & \coker(\bpp j^0)
    \end{tikzcd}
  \]
  We also find from the \hyperref[snake lemma]{snake lemma} that the two sequences
  \begin{gather*}
    \coker(\bp i^0)
    \to
    \coker(i^0)
    \to
    \coker(\bpp i^0)
    \to
    0
  \shortintertext{and}
    \coker(\bp j^0)
    \to
    \coker(j^0)
    \to
    \coker(\bpp j^0)
    \to
    0
  \end{gather*}
  are (right) exact.
  We hence get the following commutative diagram with (right) exact rows:
  \[
    \begin{tikzcd}
        \coker(\bp i^0)
        \arrow{r}
        \arrow{d}
      & \coker(i^0)
        \arrow{r}
        \arrow{d}
      & \coker(\bpp i^0)
        \arrow{r}
        \arrow{d}
      & 0
      \\
        \coker(\bp j^0)
        \arrow{r}
      & \coker(j^0)
        \arrow{r}
      & \coker(\bpp j^0)
        \arrow{r}
      & 0
    \end{tikzcd}
  \]
  The commutative diagram
  \[
    \begin{tikzcd}[row sep = small, column sep = 0.8em]
        {}
      & 0
        \arrow{rr}
      & {}
      & \bp I^0
        \arrow{rr}
        \arrow{dl}
        \arrow{dd}
      & {}
      & I^0
        \arrow{rr}
        \arrow{dl}
        \arrow{dd}
      & {}
      & \bpp I^0
        \arrow{rr}
        \arrow{dl}
        \arrow{dd}
      & {}
      & 0
      \\
        0
        \arrow{rr}
      & {}
      & \bp I^1
        \arrow[from=ll, crossing over]
        \arrow{dd}
      & {}
      & I^1
        \arrow[from=ll, crossing over]
      & {}
      & \bpp I^1
        \arrow[from=ll, crossing over]
      & {}
      & 0
        \arrow[from=ll, crossing over]
      & {}
      \\
        {}
      & 0
        \arrow{rr}
      & {}
      & \bp J^0
        \arrow{rr}
        \arrow{dl}
      & {}
      & J^0
        \arrow{rr}
        \arrow{dl}
      & {}
      & \bpp J^0
        \arrow{rr}
        \arrow{dl}
      & {}
      & 0
      \\
        0
        \arrow{rr}
      & {}
      & \bp J^1
        \arrow[from=uu, crossing over]
        \arrow[from=ll, crossing over]
      & {}
      & J^1
        \arrow[from=ll, crossing over]
      & {}
      & \bpp J^1
        \arrow[from=uu, crossing over]
        \arrow[from=ll, crossing over]
      & {}
      & 0
        \arrow[from=ll, crossing over]
      & {}
    \end{tikzcd}
  \]
  factors through the following commutative diagram:
  \[
    \begin{tikzcd}[row sep = small, column sep = 0.8em]
        {}
      & {}
      & {}
      & \coker(\bp i^0)
        \arrow{rr}
        \arrow{dl}
        \arrow{dd}
      & {}
      & \coker(i^0)
        \arrow{rr}
        \arrow{dl}
        \arrow{dd}
      & {}
      & \coker(\bpp i^0)
        \arrow{rr}
        \arrow{dl}
        \arrow{dd}
      & {}
      & 0
      \\
        0
        \arrow{rr}
      & {}
      & \bp I^1
        \arrow{dd}
      & {}
      & I^1
        \arrow[from=ll, crossing over]
      & {}
      & \bpp I^1
        \arrow[from=ll, crossing over]
      & {}
      & 0
        \arrow[from=ll, crossing over]
      & {}
      \\
        {}
      & {}
      & {}
      & \coker(\bp j^0)
        \arrow{rr}
        \arrow{dl}
      & {}
      & \coker(j^0)
        \arrow{rr}
        \arrow{dl}
      & {}
      & \coker(\bpp j^0)
        \arrow{rr}
        \arrow{dl}
      & {}
      & 0
      \\
        0
        \arrow{rr}
      & {}
      & \bp J^1
        \arrow[from=ll, crossing over]
      & {}
      & J^1
        \arrow[from=ll, crossing over]
      & {}
      & \bpp J^1
        \arrow[from=uu, crossing over]
        \arrow[from=ll, crossing over]
      & {}
      & 0
        \arrow[from=ll, crossing over]
      & {}
    \end{tikzcd}
  \]
  We can now apply the previous lemma to get the desired morphism~$f^1 \colon I^1 \to J^1$.
  To construct the morphisms~$f^n$ with~$n \geq 2$ we can proceed inductively in the same way.
\end{proof}


\begin{theorem}
  There exist morphisms
  \[
    \delta^n_\xi
    \colon
    (\Right^n F)(X'')
    \to
    (\Right^{n+1} F)(X') \,,
  \]
  where~$n \geq 0$ and where
  \[
    \xi
    \colon
    0
    \to
    X'
    \to
    X
    \to
    X''
    \to
    0
  \]
  is any short exact sequence in~$\Acat$, that make~$\Rightdelta F \defined ( (\Right^n F)_{n \geq 0}, (\delta^n_\xi)_{n \geq 0, \xi} )$ into a cohomological~{\deltafun}.
\end{theorem}


\begin{proof}
  Let
  \[
    \xi
    \colon
    0
    \to
    X'
    \xlongto{g}
    X
    \xlongto{h}
    X''
    \to
    0
  \]
  be a short exact sequence in~$\Acat$.
  We get by the \hyperref[horseshoe lemma]{horseshoe lemma} an injective resolution~$(\Iccc, i^0)$ of~$X$ such that the short exact sequence~$\xi$ lifts to a short exact sequence of chain complexes
  \[
    0
    \to
    \Iccc_{X'}
    \to
    \Iccc
    \to
    \Iccc_{X''}
    \to
    0 \,,
  \]
  such that the following diagram commutes:
  \[
    \begin{tikzcd}
        0
        \arrow{r}
      & X''
        \arrow{r}
        \arrow{d}
      & X
        \arrow{r}
        \arrow{d}
      & X'
        \arrow{r}
        \arrow{d}
      & 0
      \\
        0
        \arrow{r}
      & \Iccc_{X'}
        \arrow{r}
      & \Iccc
        \arrow{r}
      & \Iccc_{X''}
        \arrow{r}
      & 0
    \end{tikzcd}
  \]
  The injective resolution~$(\Iccc, i^0)$ can moreover be choosen such that the short sequence~$0 \to \Iccc_{X'} \to \Iccc \to \Iccc_{X''} \to 0$ is split in each degree, i.e.\ such that the sequence~$0 \to I_{X'}^n \to I^n \to I_{X''}^n \to 0$ splits for every~$n \geq 0$ (again by the \hyperref[horseshoe lemma]{horseshoe lemma}).
  It follows with the additivity of~$F$ that for every~$n \geq 0$ the resulting sequence
  \[
    0
    \to
    F(I_{X'}^n)
    \to
    F(I^n)
    \to
    F(I_{X''}^n)
    \to
    0
  \]
  is again exact (as explained in \cref{split preserved under additive}).
  This shows that the sequence
  \[
    0
    \to
    F(\Iccc_{X'})
    \to
    F(\Iccc)
    \to
    F(\Iccc_{X''})
    \to
    0
  \]
  is again (short) exact.
  We can therefore consider the resulting long exact cohomology sequence
  \[
    \dotsb
    \to
    \Hl^n(F(\Iccc_{X'}))
    \to
    \Hl^n(F(\Iccc))
    \to
    \Hl^n(F(\Iccc_{X''}))
    \xlongto{ \delta^n_\xi }
    \Hl^{n+1}(F(\Iccc_{X''}))
    \to
    \dotsb
  \]
  It follows from \cref{preparation for right derived functors} that we get the following commutative diagram, in which the vertical arrows are isomorphisms:
  \[
    \begin{tikzcd}[column sep = small, row sep = large]
        \dotsb
        \arrow{r}
      & \Hl^n(F(\Iccc_{X'}))
        \arrow{r}
        \arrow[equal]{d}
      & \Hl^n(F(\Iccc))
        \arrow{rr}
        \arrow{d}[right]{\Right^n_{(\Iccc, i^0),(\Iccc_X, i_X^0)}(\id_X)}
      & {}
      & \Hl^n(F(\Iccc_{X''}))
        \arrow{r}[above]{\delta^n_\xi}
        \arrow[equal]{d}
      & \Hl^{n+1}(F(\Iccc_{X'}))
        \arrow{r}
        \arrow[equal]{d}
      & \dotsb
      \\
        \dotsb
        \arrow{r}
      & (\Right^n F)(X')
        \arrow{r}
      & (\Right^n F)(X)
        \arrow{rr}
      & {}
      & (\Right^n F)(X'')
        \arrow{r}[above]{\delta^n_\xi}
      & (\Right^{n+1} F)(X')
        \arrow{r}
      & \dotsb
    \end{tikzcd}
  \]
  The lower row of this diagram is therefore a long exact sequence:
  \[
    \dotsb
    \to
    (\Right^n F)(X')
    \to
    (\Right^n F)(X)
    \to
    (\Right^n F)(X'')
    \xlongto{ \delta^n_\xi }
    (\Right^{n+1} F)(X')
    \to
    \dotsb
  \]
  
  We need to show that this induced long exact sequence in natural in~$\xi$:
  We hence consider a commutative diagram
  \[
    \begin{tikzcd}
        0
        \arrow{r}
      & X'
        \arrow{r}
        \arrow{d}
      & X
        \arrow{r}
        \arrow{d}
      & X''
        \arrow{r}
        \arrow{d}
      & 0
      \\
        0
        \arrow{r}
      & Y'
        \arrow{r}
      & Y
        \arrow{r}
      & Y''
        \arrow{r}
      & 0
    \end{tikzcd}
  \]
  in~$\Acat$ with (short) exact rows.
  We get by the \hyperref[horseshoe lemma]{horseshoe lemma} injective resolutions~$(\Iccc, i^0)$ and~$(\Jccc, j^0)$ of~$X$ and~$Y$, and short exact sequences
  \[
    0
    \to
    \Iccc_{X'}
    \to
    \Iccc
    \to
    \Iccc_{X''}
    \to
    0
    \qquad\text{and}\qquad
    0
    \to
    \Jccc_{X'}
    \to
    \Jccc
    \to
    \Jccc_{X''}
    \to
    0 \,,
  \]
  that fit into the following commutative diagram:
  \[
    \begin{tikzcd}[sep = small]
        {}
      & 0
        \arrow{rr}
      & {}
      & X'
        \arrow{rr}
        \arrow{dl}
        \arrow{dd}
      & {}
      & X
        \arrow{rr}
        \arrow{dl}
        \arrow{dd}
      & {}
      & X''
        \arrow{rr}
        \arrow{dl}
        \arrow{dd}
      & {}
      & 0
      \\
        0
        \arrow{rr}
      & {}
      & \Iccc_{X'}
      & {}
      & \Iccc
        \arrow[from=ll, crossing over]
      & {}
      & \Iccc_{X''}
        \arrow[from=ll, crossing over]
      & {}
      & 0
        \arrow[from=ll, crossing over]
      & {}
      \\
        {}
      & 0
        \arrow{rr}
      & {}
      & Y'
        \arrow{rr}
        \arrow{dl}
      & {}
      & Y
        \arrow{rr}
        \arrow{dl}
      & {}
      & Y''
        \arrow{rr}
        \arrow{dl}
      & {}
      & 0
      \\
        0
        \arrow{rr}
      & {}
      & \Jccc_{X'}
        \arrow{rr}
      & {}
      & \Jccc
        \arrow{rr}
      & {}
      & \Jccc_{X''}
        \arrow{rr}
      & {}
      & 0
      & {}
    \end{tikzcd}
  \]
  It follows from the \hyperref[comparison theorem]{comparison theorem} that there exists morphisms of chain complexes~$\Iccc_{X'} \to \Jccc_{X'}$ and~$\Iccc_{X''} \to \Jccc_{X''}$ that make the resulting diagram
  \[
    \begin{tikzcd}[sep = small]
        {}
      & 0
        \arrow{rr}
      & {}
      & X'
        \arrow{rr}
        \arrow{dl}
        \arrow{dd}
      & {}
      & X
        \arrow{rr}
        \arrow{dl}
        \arrow{dd}
      & {}
      & X''
        \arrow{rr}
        \arrow{dl}
        \arrow{dd}
      & {}
      & 0
      \\
        0
        \arrow{rr}
      & {}
      & \Iccc_{X'}
      & {}
      & \Iccc
        \arrow[from=ll, crossing over]
      & {}
      & \Iccc_{X''}
        \arrow[from=ll, crossing over]
      & {}
      & 0
        \arrow[from=ll, crossing over]
      & {}
      \\
        {}
      & 0
        \arrow{rr}
      & {}
      & Y'
        \arrow{rr}
        \arrow{dl}
      & {}
      & Y
        \arrow{rr}
        \arrow{dl}
      & {}
      & Y''
        \arrow{rr}
        \arrow{dl}
      & {}
      & 0
      \\
        0
        \arrow{rr}
      & {}
      & \Jccc_{X'}
        \arrow[dashed, from=uu, crossing over, "\bp f", near start, swap]
        \arrow{rr}
      & {}
      & \Jccc
        \arrow{rr}
      & {}
      & \Jccc_{X''}
        \arrow[dashed, from=uu, crossing over, "\bpp f", near start, swap]
        \arrow{rr}
      & {}
      & 0
      & {}
    \end{tikzcd}
  \]
  commute.
  It follows from \cref{extension into commutative diagram} that there exists a morphism of chain complexes~$\Iccc \to \Jccc$ that makes the diagram
  \[
    \begin{tikzcd}[sep = small]
        {}
      & 0
        \arrow{rr}
      & {}
      & X'
        \arrow{rr}
        \arrow{dl}
        \arrow{dd}
      & {}
      & X
        \arrow{rr}
        \arrow{dl}
        \arrow{dd}
      & {}
      & X''
        \arrow{rr}
        \arrow{dl}
        \arrow{dd}
      & {}
      & 0
      \\
        0
        \arrow{rr}
      & {}
      & \Iccc_{X'}
      & {}
      & \Iccc
        \arrow[from=ll, crossing over]
      & {}
      & \Iccc_{X''}
        \arrow[from=ll, crossing over]
      & {}
      & 0
        \arrow[from=ll, crossing over]
      & {}
      \\
        {}
      & 0
        \arrow{rr}
      & {}
      & Y'
        \arrow{rr}
        \arrow{dl}
      & {}
      & Y
        \arrow{rr}
        \arrow{dl}
      & {}
      & Y''
        \arrow{rr}
        \arrow{dl}
      & {}
      & 0
      \\
        0
        \arrow{rr}
      & {}
      & \Jccc_{X'}
        \arrow[from=uu, crossing over, "\bp f", near start, swap]
        \arrow{rr}
      & {}
      & \Jccc
        \arrow[dashed, from=uu, crossing over, "f", near start, swap]
        \arrow{rr}
      & {}
      & \Jccc_{X''}
        \arrow[from=uu, crossing over, "\bpp f", near start, swap]
        \arrow{rr}
      & {}
      & 0
      & {}
    \end{tikzcd}
  \]
  commute.
  We get in cohomology the following diagram with long exact rows:
  \[
    \begin{tikzcd}[column sep = -0.67em, row sep = small]
        {}
      & \dotsb
        \arrow{rr}
      & {}
      & \Hl^n(F(\Iccc_{X'}))
        \arrow{rr}
        \arrow[equal]{dl}
        \arrow{dd}
      & {}
      & \Hl^n(F(\Iccc))
        \arrow{rr}
        \arrow{dl}
        \arrow{dd}
      & {}
      & \Hl^n(F(\Iccc_{X''}))
        \arrow{rr}
        \arrow[equal]{dl}
        \arrow{dd}
      & {}
      & \dotsb
      \\
        \dotsb
        \arrow{rr}
      & {}
      & (\Right^n F)(X')
      & {}
      & (\Right^n F)(X)
        \arrow[from=ll, crossing over]
      & {}
      & (\Right^n F)(X'')
        \arrow[from=ll, crossing over]
      & {}
      & \dotsb
        \arrow[from=ll, crossing over]
      & {}
      \\
        {}
      & \dotsb
        \arrow{rr}
      & {}
      & \Hl^n(F(\Iccc_{Y'}))
        \arrow{rr}
        \arrow[equal]{dl}
      & {}
      & \Hl^n(F(\Jccc))
        \arrow{rr}
        \arrow{dl}
      & {}
      & \Hl^n(F(\Iccc_{Y''}))
        \arrow{rr}
        \arrow[equal]{dl}
      & {}
      & \dotsb
      \\
        \dotsb
        \arrow{rr}
      & {}
      & (\Right^n F)(Y')
        \arrow[from=uu, crossing over]
        \arrow{rr}
      & {}
      & (\Right^n F)(Y)
        \arrow[from=uu, crossing over]
        \arrow{rr}
      & {}
      & (\Right^n F)(Y'')
        \arrow[from=uu, crossing over]
        \arrow{rr}
      & {}
      & \dotsb
      & {}
    \end{tikzcd}
  \]
  We have seen above that the top and bottom squares commute in this diagram, and the back squares commute by the naturality of the long exact cohomology sequence.
  The commutativity of the front squares folllows because the diagonal morphisms are isomorphisms.
\end{proof}





\lecturend{21}




