\section{Injectives in the Category of Quiver Representations}


\begin{remark}
  Let~$Q$ be a finite quiver and suppose that~$\kf$ is a field.
  For every~$i \in Q_0$ we define a representation~$I(i)$ of~$Q$ by the following data:
  \begin{itemize}
    \item
      For every vertex~$j \in Q_0$ the~{\kvs}~$I(i)_j$ is the free~{\kvs} with basis~$Q_*(j,i)$.
    \item
      For every arrow~$\alpha \in Q_1$ with~$\alpha \colon j \to k$ the~{\klin} map~$I(i)_\alpha \colon I(i)_j \to I(i)_k$ is given on the basis elements~$p \in Q_*(j,i)$ of~$P(i)_j$ by
      \[
        I(i)_\alpha(p)
        \defined
        \begin{cases}
          p'  & \text{if~$p = p' \alpha$} \,, \\
          0   & \text{otherwise}  \,.
        \end{cases}
      \]
      We get for every representation~$X$ of~$Q$ an isomorphism of~{\kvs}
      \[
        \varphi
        \colon
        \Hom(X, I(i))
        \to
        X_i^* \,,
      \]
      where for every~$f \in \Hom(X,I(i))$ the element~$\varphi(f) \in X_i^*$ is given by the composition
      \[
        \varphi(f)
        \colon
        X_i
        \xlongto{f_i}
        I(i)_i
        \xlongto{\pi_{\varepsilon_i}}
        k \,.
      \]
      It follows that~$I(i)$ is injective.
  \end{itemize}
\end{remark}


\begin{remark*}
  \leavevmode
  \begin{enumerate}
    \item
      The injectivity of the~$I(i)$ can be shown similarly to the projectivity of the~$P(i)$ in \cref{P(i) projective via Hom}:
      For every representation~$X$ of~$Q$ over~$\kf$ let
      \[
        \varphi_X
        \colon
        \Hom(X, I(i))
        \to
        X_i^*
      \]
      be the above isomorphism of~{\kvss}.
      Let~$X$ and~$Y$ be two representations of~$Q$ over~$\kf$ and let~$f \colon X \to Y$ be a monomorphism of representations
      We then have the following commutative rectangle:
      \[
        \begin{tikzcd}
            \Hom(Y, I(i))
            \arrow{r}[above]{f^*}
            \arrow{d}[left]{\varphi_X}
          & \Hom(X, I(i))
            \arrow{d}[right]{\varphi_Y}
          \\
            Y_i^*
            \arrow{r}[below]{f_i^*}
          & X_i^*
        \end{tikzcd}
      \]
      Indeed, we have for~$g \in \Hom(Y,I(i))$ and~$x \in X_i$ that
      \[
        \varphi_Y( f^*(g) )(x)
        =
        \varphi_Y( g \circ f )(x)
        =
        \pi_{\varepsilon_i}( (g \circ f)_i(x) )
        =
        \pi_{\varepsilon_i}( g_i( f_i(x) ) )
      \]
      and similarly
      \[
        (f_i^* \circ \varphi_X)(g)(x)
        =
        f_i^*( \varphi_X(g) )(x)
        =
        \varphi_X(g)( f_i(x) )
        =
        \pi_{\varepsilon_i}( g_i( f_i(x) ) )  \,.
      \]
      The~{\klin} map~$f_i \colon X_i \to Y_i$ is injective, so its dual map~$f_i^* \colon Y_i^* \to X_i^*$ is surjective.
      It follows that~$f^* = \varphi_Y^{-1} \circ f_i^* \circ \varphi_X$ is surjective.
      This shows that the functor~$\Hom(-,I(i))$ maps monomorphisms (in~$\Acat$) to surjections, which means that the representation~$I(i)$ is injective.
    \item
      For any right~{\module{$A$}}~$M$ its dual space~$M^*$ becomes a left~{\module{$A$}} via
      \[
        (a \varphi)(m)
        =
        \varphi(m a) 
      \]
      for all~$a \in A$,~$\varphi \in M^*$ and~$m \in M$.
      We find for the (projective) right~{\module{$A$}}~$\varepsilon_i A$ that its dual~$(\varepsilon_i A)^*$ becomes a left~{\module{$A$}}.
      The {\module{$A$}}~$I(i)$ can be constructed as a submodule of~$(\varepsilon_i A)^*$:
      
      The right~{\module{$A$}}~$\varepsilon_i A$ has as a basis the set
      \begin{equation}
        \label{original basis}
        \{
          p \in Q_*
        \suchthat
          t(p) = i
        \}  \,,
      \end{equation}
      hence its dual~$(\varepsilon_i A)^*$ contains the linear independent family
      \begin{equation}
        \label{dual basis}
        \{
          p^*
        \suchthat
          p \in Q_*,
          t(p) = i
        \}  \,.
      \end{equation}
      We have for every arrow~$\alpha \in Q_1$ with~$\alpha \colon j \to k$ that
      \begin{equation}
        \label{dual multiplication rule}
        \alpha \cdot p^*
        =
        \begin{cases}
          (p')^*  & \text{if~$p = p' \alpha$}  \,, \\
          0       & \text{otherwise}  \,.
        \end{cases}
      \end{equation}
      Indeed, it holds for every basis element~$q$ of~$\varepsilon_i A$, i.e.~$q \in Q_*$ with~$t(\alpha) = i$, that
      \[
        (\alpha \cdot p^*)(q)
        =
        p^*(q \alpha)
        =
        \begin{cases}
          1 & \text{if~$p = q \alpha$}  \,, \\
          0 & \text{otherwise}  \,.
        \end{cases}
      \]
      If~$p = p' \alpha$ for some path~$p' \in Q_*$ (with necessarily~$t(p') = t(p) = i$) then therefore
      \[
        (\alpha \cdot p^*)(q)
        =
        \begin{cases}
          1 & \text{if~$q = p'$}  \,, \\
          0 & \text{otherwise}        \\
        \end{cases}
        =
        (p')^*(q)
      \]
      and hence~$\alpha \cdot p^* = (p')^*$.
      If no such~$p'$ exists then~$(\alpha \cdot p^*)(q) = 0$ for every~$q$ and hence~$\alpha \cdot p^* = 0$.
      
      We can now identify the elements of~\eqref{dual basis} with the elements of the original basis~\eqref{original basis} via the bijection~$p^* \mapsto p$.
      We then get a left~{\module{$A$}}~$I$ with basis~\eqref{original basis}, and on which the multiplication rule~\eqref{dual multiplication rule} becomes
      \[
        \alpha \cdot p
        =
        \begin{cases}
          p'  & \text{if~$p = p' \alpha$} \,, \\
          0   & \text{otherwise}  \,.
        \end{cases}
      \]
      We have in particular for every~$j \in Q_0$ that~$\varepsilon_j I$ has as a basis the set
      \begin{align*}
        {}&
        \{
          p' \in Q_*
        \suchthat
          \text{$p = p' \varepsilon_j$ for some~$p \in Q_*$ with~$t(p) = i$}
        \}
        \\
        ={}&
        \{
          p' \in Q_*
        \suchthat
          \text{$p = p'$ for some~$p \in Q_*$ with~$t(p) = i$ and~$s(p) = j$}
        \}
        \\
        ={}&
        \{
          p \in Q_*
        \suchthat
          s(p) = j,
          t(p) = i
        \}
        =
        Q_*(j,i)  \,.
      \end{align*}
      This shows that~$I_j = I(i)_j$ for every~$j \in Q_0$ and that the actions of~$Q$ on~$I$ and~$I(i)$ coincide.
      Hence~$I(i) = I \inclusion (\varepsilon_i A)^*$.
  \end{enumerate}
\end{remark*}




