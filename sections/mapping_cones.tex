\section{Mapping Cones}


\begin{definition}
  \leavevmode
  \begin{enumerate}
    \item
      Let~$f \colon \Ccc \to \Dcc$ be a morphism of chain complexes.
      The \emph{mapping cone}\index{mapping cone}\index{cone} of~$f$ is the chain complex~$\cone(f)$ that is given by
      \begin{itemize}
        \item
          the components~$\cone(f)_n \defined C_{n-1} \oplus D_n$ for every~$n \in \Integer$, together with
        \item
          the differential
          \[
                      d^{\,\cone(f)}_n
            \defined  \begin{bmatrix}
                        -d^{\,C}_{n-1}  & 0     \\
                        -f_{n-1}    & d^D_n
                      \end{bmatrix}
            \colon    C_{n-1} \oplus D_n
            \to       C_{n-2} \oplus D_{n-1}
          \]
          for every~$n \in \Integer$.
      \end{itemize}
    \item
      Let~$f \colon \Cccc \to \Dccc$ be a morphism of cochain complexes.
      The \emph{mapping cone}\index{mapping cone}\index{cone} of~$f$ is the cochain complex~$\cone(f)$ that is given by
      \begin{itemize}
        \item
          the components~$\cone(f)^n \defined C^{n+1} \oplus D^n$ for every~$n \in \Integer$, together with
        \item
          the differential
          \[
                      d_{\cone(f)}^n
            \defined  \begin{bmatrix}
                        -d_C^{n+1}  & 0     \\
                        -f^{n+1}    & d_D^n
                      \end{bmatrix}
            \colon    C^{n+1} \oplus D^n
            \to       C^{n+2} \oplus D^{n+1}
          \]
          for every~$n \in \Integer$.
      \end{itemize}
  \end{enumerate}
\end{definition}


\begin{remark}
  If~$f \colon \Ccc \to \Dcc$ is a morphism of chain complexes, or~$f \colon \Cccc \to \Dccc$ a morphism of cochain complexes, then~$\cone(f)$ is indeed a (co)chain complex because
  \[
      \begin{bmatrix}
        -d  & 0 \\
        -f  & d \\
      \end{bmatrix}
      \begin{bmatrix}
        -d  & 0 \\
        -f  & d
      \end{bmatrix}
    = \begin{bmatrix}
        d^2    & 0   \\
        fd-df  & d^2
      \end{bmatrix}
    = \begin{bmatrix}
        0 & 0 \\
        0 & 0
      \end{bmatrix} \,.
  \]
\end{remark}


\begin{proposition}
  \label{exact sequence of cone}
  Let~$f \colon \Ccc \to \Dcc$ be a morphism of chain complexes.
  \begin{enumerate}
    \item
      The family~$\alpha = (\alpha_n)_{n \in \Integer}$ of morphisms
      \[
                  \alpha_n
        \defined  \begin{bmatrix}
                    0 \\
                    \id_{D_n}
                  \end{bmatrix}
        \colon    D_n
        \to       C_{n-1} \oplus D_n
      \]
      is a morphism of chain complexes~$\alpha \colon \Dcc \to \cone(f)$.
    \item
      The family~$\beta = (\beta_n)_{n \in \Integer}$ of morphisms
      \[
                  \beta_n
        \defined  \begin{bmatrix}
                    -\id_{C_{n-1}} & 0
                  \end{bmatrix}
        \colon    C_{n-1} \oplus D_n
        \to       C_{n-1}
      \]
      is a morphism of chain complexes~$\beta \colon \cone(f) \to \Ccc[-1]$.
    \item
      The morphisms~$\alpha$ and~$\beta$ fit into a short exact sequence
      \[
        0
        \to
        \Dcc
        \xlongto{\alpha}
        \cone(f)
        \xlongto{\beta}
        \Ccc[-1]
        \to
        0 \,.
      \]
    \item
      In the induced long exact homology sequence
      \[
        \begin{tikzcd}
            {}
          & \dotsb
            \arrow{r}
            \arrow[d, phantom, ""{coordinate, name=Y}]
          & \Hl_{n+1}(\Ccc[-1])
            \arrow[ dll,
                    "\del_{n+1}",
                    rounded corners,
                    to path={ -- ([xshift=2ex]\tikztostart.east)
                              |- (Y) \tikztonodes
                              -| ([xshift=-2ex]\tikztotarget.west)
                              -- (\tikztotarget)}
                  ]
          \\
            \Hl_n(\Dcc)
            \arrow{r}[above]{\Hl_n(\alpha)}
          & \Hl_n(\cone(f))
            \arrow{r}[above]{\Hl_n(\beta)}
            \arrow[d, phantom, ""{coordinate, name=Z}]
          & \Hl_n(\Ccc[-1])
            \arrow[ dll,
                    "\del_n",
                    rounded corners,
                    to path={ -- ([xshift=2ex]\tikztostart.east)
                              |- (Z) \tikztonodes
                              -| ([xshift=-2ex]\tikztotarget.west)
                              -- (\tikztotarget)}
                  ]
          \\
            \Hl_{n-1}(\Dcc)
            \arrow{r}
          & \dotsb
          & {}
        \end{tikzcd}
      \]
      the connecting morphism
      \[
        \del_{n+1}
        \colon
        \Hl_n(\Ccc)
        =
        \Hl_{n+1}(\Ccc[-1])
        \to
        \Hl_n(\Dcc)
      \]
      is given by~$\Hl_n(f)$ for every~$n \in \Integer$.
  \end{enumerate}
\end{proposition}


\begin{proof}
  This proof is currently missing from these notes, and will be added later.
  (The proof relies on the explicit construction of the connecting homomorphism in the \hyperref[snake lemma]{snake lemma}.)
% TODO: Add this proof.
\end{proof}


\begin{corollary}
  A morphism of chain complexes~$f \colon \Ccc \to \Dcc$ is a {\qim} if and only if its cone is acyclic.
\end{corollary}


\begin{proof}
  We get by~\cref{exact sequence of cone} a long exact sequence
  \[
    \dotsb
    \to
    \Hl_{n+1}(\cone(f))
    \to
    \Hl_n(\Ccc)
    \xlongto{\Hl_n(f)}
    \Hl_n(\Dcc)
    \to
    \Hl_n(\cone(f))
    \to
    \dotsb
  \]
  The morphism~$f$ is a {\qim} if and only if~$\Hl_n(f)$ is for every~$n \in \Integer$ an isomorphism.
  This holds by the exactness of the above sequence if and only if~$\Hl_n(\cone(f)) = 0$ for every~$n \in \Integer$.
  This is precisely what it means for~$\cone(f)$ to be acyclic.
\end{proof}


\begin{remark*}
  Let~$\Ccc$,~$\Dcc$ and~$\Ecc$ be chain complexes in~$\Acat$.
  \begin{enumerate}
    \item
      Let~$f \colon \Ccc \to \Dcc$ be a morphism of chain complexes.
      Then for a family~$g = (g_n)_{n \in \Integer}$ of morphisms
      \[
        g_n
        \colon
        C_{n-1} \oplus D_n
        \to
        E_n
      \]
      we have that~$g$ is a morphism of chain complexes~$g \colon \cone(f) \to \Ecc$ if and only if
      \begin{equation}
        \label{morphism out of mapping cone}
          d^E_n g_n
        = g_{n-1} d^{\,\cone(f)}_n
      \end{equation}
      for every~$n \in \Integer$.
      We may write for every~$n \in \Integer$ the morphism~$g_n$ as
      \[
          g_n
        = \begin{bmatrix}
            s_{n-1} & -h_n
          \end{bmatrix}
      \]
      for unique morphisms~$s_{n-1} \colon C_{n-1} \to E_n$ and~$h_n \colon D_n \to E_n$.
      We can then rewrite the condition~\eqref{morphism out of mapping cone} for every~$n \in \Integer$ as
      \begin{align*}
            &{}   d^E_n g_n
                = g_{n-1} d^{\,\cone(f)}_n  \\
        \iff&{}   d_n
                  \begin{bmatrix}
                    s_{n-1} & -h_n
                  \end{bmatrix}
                =
                  \begin{bmatrix}
                    s_{n-2} & -h_{n-1}
                  \end{bmatrix}
                  \begin{bmatrix}
                    -d_{n-1}  & 0   \\
                    -f_{n-1}  & d_n
                  \end{bmatrix} \\
        \iff&{}   \begin{bmatrix}
                    d_n s_{n-1} & -d_n h_n
                  \end{bmatrix}
                  =
                  \begin{bmatrix}
                    - s_{n-2} d_{n-1} + h_{n-1} f_{n-1} & -h_{n-1} d_n
                  \end{bmatrix} \\
        \iff&{}   \left\{
                    \begin{array}{rcl}
                      h_{n-1} f_{n-1} &=& d_n s_{n-1} + s_{n-2} d_{n-1} \,, \\
                      d_n h_n     &=& h_{n-1} d_n \,.
                    \end{array}
                  \right.
      \end{align*}
      That the second condition holds for every~$n \in \Integer$ means precisely that~$h \defined (h_n)_{n \in \Integer}$ is a morphism of chain complexes~$h \colon \Dcc \to \Ecc$, and the first condition means that~$s \defined (s_n)_{n \in \Integer}$ is a null homotopy for the composition~$h f$.
      
      We hence find that morphisms of chain complexes~$g \colon \cone(f) \to \Ecc$ are in {\onetoone} correspondence to pairs~$(h,s)$ consisting of
      \begin{itemize}
        \item
          a morphism of chain complexes~$h \colon \Dcc \to \Ecc$ such that the composition~$hf$ is null homotopic, and
        \item
          a chosen null homotopy~$s$ for~$hf$.
      \end{itemize}
    \item
      The \emph{cone}\index{cone}\index{chain complex!cone} of the chain complex~$\Ccc$ it the mapping cone of its identity, i.e.,
      \[
                  \cone(\Ccc)
        \defined  \cone(\id_{\Ccc}) \,.
      \]
      We have the canonical morphism of chain complexes~$\Ccc \to \cone(\Ccc)$, whose~\dash{$n$}{th} component is for every~$n \in \Integer$ given by
      \[
        \begin{bmatrix}
            0
          & \id
        \end{bmatrix}
        \colon
        C_n
        \to
        C_{n-1} \oplus C_n \,.
      \]
      We get from the above discussion that a morphism of chain complexes~$f \colon \Ccc \to \Dcc$ extends to a morphism of chain complexes~$\hat{f} \colon \cone(f) \to \Dcc$, in the sense that the diagram
      \[
        \begin{tikzcd}
            \cone(f)
            \arrow[dashed]{dr}[above right]{\hat{f}}
          & {}
          \\
            \Ccc
            \arrow{u}
            \arrow{r}[below]{f}
          & \Dcc
        \end{tikzcd}
      \]
      commutes, if and only if the morphism~$f$ is null homotopic.
      (This is also Exercise~3 of Exercise~Sheet~9.)
      To be more precise, such extensions~$\hat{f}$ of~$f$ are in {\onetoone} correspondence to null homotopies of~$f$.
  \end{enumerate}
\end{remark*}






\lecturend{17}




