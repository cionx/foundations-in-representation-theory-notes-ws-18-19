\section{Pullbacks and Pushouts}


\begin{definition*}
  Let~$\Ccat$ be a category.
  \begin{enumerate}
    \item
      Let~$(f,g)$ be a pair of morphisms
      \[
        X \xto{f} Y \xfrom{g} Y'
      \]
      in~$\Ccat$.
      A \emph{pullback}\index{pullback} of the pair~$(f,g)$ is a triple~$(X', f', g')$ consisting of
      \begin{itemize}
        \item
          an object~$X' \in \Ob(\Ccat)$ and
        \item
          two morphisms \[Y' \xfrom{f'} X' \xto{g'} X \,,\]
      \end{itemize}
      such that
      \begin{enumerate}[label=(PB\arabic*)]
        \item
          the square diagram
          \[
            \begin{tikzcd}
                X'
                \arrow[dashed]{r}[above]{g'}
                \arrow[dashed]{d}[left]{f'}
              & X
                \arrow{d}[right]{f}
              \\
                Y'
                \arrow{r}[above]{g}
              & Y
            \end{tikzcd}
          \]
          commutes, and
        \item
          the triple~$(X', f', g')$ is universal with this property, in the following sense:
          If~$(X'', f'', g'')$ is any other triple consisting of an object~$X'' \in \Ob(\Ccat)$ and two morphisms
          \[
            Y'' \xlongfrom{f''} X'' \xlongto{g''} X
          \]
          such that the square
          \[
            \begin{tikzcd}
                X''
                \arrow[dashed]{r}[above]{g''}
                \arrow[dashed]{d}[left]{f''}
              & X
                \arrow{d}[right]{f}
              \\
                Y''
                \arrow{r}[above]{g}
              & Y
            \end{tikzcd}
          \]
          commutes, then there exists a unique morphism~$\lambda \colon X'' \to X'$ that makes the following diagram commute:
          \[
            \begin{tikzcd}
                X''
                \arrow[bend left]{drr}[above right]{g''}
                \arrow[dashed]{dr}[above right]{\lambda}
                \arrow[bend right]{ddr}[below left]{f''}
              & {}
              & {}
              \\
                {}
              & X'
                \arrow{r}[above]{g'}
                \arrow{d}[left]{f'}
              & X
                \arrow{d}[right]{f}
              \\
                {}
              & Y'
                \arrow{r}[above]{g}
              & Y
            \end{tikzcd}
          \]
      \end{enumerate}
      
    \item
      Let~$(f,g)$ be a pair of morphisms
      \[
        Y \xlongfrom{f} X \xlongto{g} X'
      \]
      in~$\Ccat$.
      A \emph{pushout}\index{pushout} of the pair~$(f,g)$ is a triple~$(Y',f',g')$ consisting of
      \begin{itemize}
        \item
          an object~$Y' \in \Ob(\Acat)$ and
        \item
          two morphisms
          \[
            X' \xlongto{f'} Y' \xlongfrom{g'} Y \,,
          \]
      \end{itemize}
      such that
      \begin{enumerate}[label=(PO\arabic*)]
        \item
          the square diagram
          \[
            \begin{tikzcd}
                X
                \arrow{r}[above]{g}
                \arrow{d}[left]{f}
              & X'
                \arrow[dashed]{d}[left]{f'}
              \\
                Y
                \arrow[dashed]{r}[above]{g'}
              & Y'
            \end{tikzcd}
          \]
          commutes, and
        \item
          the triple~$(Y',f',g')$ is universal with this property, in the following sense:
          If~$(Y'', f'', g'')$ is another triple consisting of an object~$Y'' \in \Ob(\Ccat)$ and two morphisms
          \[
            X'' \xlongto{f''} Y'' \xlongfrom{g''} Y
          \]
          such that the square
          \[
            \begin{tikzcd}
                X
                \arrow{r}[above]{g}
                \arrow{d}[left]{f}
              & X''
                \arrow[dashed]{d}[left]{f''}
              \\
                Y
                \arrow[dashed]{r}[above]{g''}
              & Y''
            \end{tikzcd}
          \]
          commutes, then there exists a unique morphism~$\mu \colon Y' \to Y''$ in~$\Ccat$ that makes the following diagram commutes:
          \[
            \begin{tikzcd}
                X
                \arrow{r}[above]{g}
                \arrow{d}[left]{f}
              & X'
                \arrow[bend left]{ddr}[above right]{f''}
                \arrow{d}[left]{f'}
              & {}
              \\
                Y
                \arrow{r}[above]{g'}
                \arrow[bend right]{drr}[below left]{g''}
              & Y'
                \arrow[dashed]{dr}[above right]{\mu}
              & {}
              \\
                {}
              & {}
              & Y''
            \end{tikzcd}
          \]
      \end{enumerate}
  \end{enumerate}
\end{definition*}


\begin{remark*}
  \leavevmode
  \begin{enumerate}
    \item
      The notions of pullback and pushout are dual to each other:
      A tripel~$(X', f', g')$ is a pullback of morphisms~$X \xto{f} Y \xfrom{g} Y'$ in~$\Ccat$ if and only if~$(X', f', g')$ is a pushout of~$X \xfrom{f} Y \xto{g} Y'$ in~$\Ccat^\op$.
    \item
      Pullbacks and pushouts are unique up to unique isomorphism:
      If~$(X', f', g')$ and~$(X'', f'', g'')$ are two pullbacks of morphisms~$X \xto{f} Y \xfrom{g} Y'$, then the unique morphism~$\lambda \colon X'' \to X'$ that makes the diagram
      \[
        \begin{tikzcd}
            X''
            \arrow[bend left]{drr}[above right]{g''}
            \arrow[dashed]{dr}[above right]{\lambda}
            \arrow[bend right]{ddr}[below left]{f''}
          & {}
          & {}
          \\
            {}
          & X'
            \arrow{r}[above]{g'}
            \arrow{d}[right]{f'}
          & X
            \arrow{d}[right]{f}
          \\
            {}
          & Y'
            \arrow{r}[above]{g}
          & Y
        \end{tikzcd}
      \]
      commute is already an isomorphism.
      For pushouts the dual statement holds.
    \item
      Pullbacks are also know as \emph{fibre products}\index{fibre!product}, and the fibre product of two morphisms~$X \xto{f} Y \xfrom{g} Y'$ is then denoted by~$X \times_Y Y'$.
      Pushouts are knows as~\emph{fibre coproducts}\index{fibre!coproduct} and \emph{amalgamated sums}\index{amalgamalted sum}, and the amalgamated sum of two morphisms~$Y \xfrom{f} X \xto{g} X'$ is then denoted by~$X' \dcup_X Y$.
    \item
      A square diagram
      \[
        \begin{tikzcd}
            X'
            \arrow[dashed]{r}[above]{g'}
            \arrow[dashed]{d}[left]{f'}
          & X
            \arrow{d}[right]{f}
          \\
            Y'
            \arrow{r}[above]{g}
          & Y
        \end{tikzcd}
      \]
      is a \emph{pullback square}\index{pullback!square} if the triple~$(X',f',g')$ is a pullback of the two morphisms~$X \xto{f} Y \xfrom{g} Y'$.
      This is often denoted by adding the symbol~$\pb$ inside of the square: 
      \[
        \begin{tikzcd}
            X'
            \arrow{r}[above]{g'}
            \arrow{d}[left]{f'}
            \arrow[phantom]{dr}[description]{\pb}
          & X
            \arrow{d}[right]{f}
          \\
            Y'
            \arrow{r}[above]{g}
          & Y
        \end{tikzcd}
      \]
      Dually, a square diagram
      \[
        \begin{tikzcd}
            X
            \arrow{r}[above]{g}
            \arrow{d}[left]{f}
          & X'
            \arrow[dashed]{d}[right]{f'}
          \\
            Y
            \arrow[dashed]{r}[above]{g'}
          & Y'
        \end{tikzcd}
      \]
      is a \emph{pushout square}\index{pushout!square} if the triple~$(X', f', g')$ is a pushout of the two morphisms~$Y \xfrom{f} X \xto{g} X'$.
      This is often denoted by adding the symbol~$\po$ inside of the square:
      \[
        \begin{tikzcd}
            X
            \arrow{r}[above]{g}
            \arrow{d}[left]{f}
            \arrow[phantom]{dr}[description]{\po}
          & X'
            \arrow{d}[right]{f'}
          \\
            Y
            \arrow{r}[above]{g'}
          & Y'
        \end{tikzcd}
      \]
    \item
      We say that the category~$\Ccat$ \emph{has pullbacks}\index{category!has!pullbacks} if every pair of morphisms~$X \xto{f} Y \xfrom{g} Y'$ in~$\Ccat$ admits a pullback in~$\Ccat$.
      We say dually that~$\Ccat$ \emph{has pushouts}\index{category!has!pushouts} if every pair of morphisms~$Y \xfrom{f} X \xto{g} X'$ in~$\Ccat$ admits a pushout in~$\Ccat$.
  \end{enumerate}
\end{remark*}


\begin{example*}
  \leavevmode
  \begin{enumerate}
    \item
      In the category~$\Set$ any pair of maps~$X \xto{f} Y \xfrom{g} Y'$ admits a pullback.
      This pullback is given by the set
      \[
                  X'
        \defined  \{
                    (x,y') \in X \times Y'
                  \suchthat
                    f(x) = g(y')
                  \}
      \]
      together with the maps~$f' \colon X' \to Y'$ and~$g' \colon X' \to X$ that are the restrictions of the canonical projections~$X \times Y' \to X$ and~$X \times Y' \to Y'$.
      
      We observe that the set~$X'$ is given by
      \[
          X'
        = \coprod_{y' \in Y'} \bigl( \{ y' \} \times f^{-1}(g(y')) \bigr)  \,,
      \]
      which explains the term \emph{fibre product} and the notation~$X \times_Y Y'$
      
      The existence and above construction of pullbacks also holds for the categories
      \[
        \Top \,,
        \enskip
        \Group \,,
        \enskip
        \Ring \,,
        \enskip
        \CommRing \,,
        \enskip
        \kAlg \,,
        \enskip
        \kCommAlg \,,
        \enskip
        \Modl{A} \,,
        \enskip
        \Modr{A} \,,
      \]
      where~$A$ is a~{\kalg}.
    \item
      In the category~$\Set$ any every pair of maps~$Y \xfrom{f} X \xto{g} X'$ admits a pushout.
      This pushout is given by the set
      \[
                  Y'
        \defined  (X' \dcup Y) / {\sim}
      \]
      where~$\sim$ is the equivalence relation generated by~$i(f(x)) \sim j(g(x))$ for~$x \in X$, where~$i \colon X' \to X' \dcup Y$ and~$j \colon Y \to X' \dcup Y$ are the canonical inclusions.
      The map~$f' \colon X' \to Y'$ is induced by the inclusion~$i$ and the map~$g' \colon Y \to Y'$ is induced by the inclusion~$j$.
      
      This construction also explains the notation~$X' \dcup_X Y$.
  \end{enumerate}
\end{example*}


\begin{remark*}
  Let~$\Ccat$ be a category and let~$Y \in \Ob(\Ccat)$ be an object in~$\Ccat$.
  The \emph{over\nobreakdash-$Y$\nobreakdash-category}~$\Ccat/Y$ is defined as follows:
  \begin{itemize}
    \item
      The objects of~$\Ccat/Y$ are pairs~$(X,x)$ consisting of an an object~$X \in \Ob(\Ccat)$ and a morphism~$x \colon X \to Y$.
      The objects of~$\Ccat/Y$ may be visualized as follows:
      \[
        \begin{tikzcd}
            X
            \arrow{d}[left]{x}
          \\
            Y
        \end{tikzcd}
      \]
    \item
      A morphism~$f \colon (X,x) \to (X',x')$ in~$\Ccat/Y$ is a morphism~$f \colon X \to X'$ that makes the triangle
      \[
        \begin{tikzcd}
            X
            \arrow[dashed]{rr}[above]{f}
            \arrow{dr}[below left]{x}
          & {}
          & X'
            \arrow{dl}[below right]{x'}
          \\
            {}
          & Y
          & {}
        \end{tikzcd}
      \]
      commute.
    \item
      The composition of two morphism~$f \colon (X,x) \to (X,x')$ and~$f' \colon (X',x') \to (X'',x'')$ is given by~$f' \circ f$, the composition of~$f$ and~$f'$ in~$\Ccat$.
      This is again a morphism in~$\Ccat/Y$ by the commutativity of the following diagram:
      \[
        \begin{tikzcd}[column sep = large]
            X
            \arrow[dashed,bend left = 40]{rr}{f' \circ f}
            \arrow{r}[above]{f}
            \arrow{dr}[below left]{x}
          & X'
            \arrow{r}[above]{f'}
            \arrow{d}[right]{x'}
          & X''
            \arrow{dl}[below right]{x''}
          \\
            {}
          & Y
          & {}
        \end{tikzcd}
      \]
    \item
      The identity morphism of an object~$(X,x) \in \Ob(\Ccat/Y)$ is given by~$\id_{(X,x_0)} = \id_X$.
  \end{itemize}
  
  A pair of morphisms~$X \xto{f} Y \xfrom{g} Y'$ can be seen as two objects~$(X,f)$ and~$(Y',g)$ of the category~$\Ccat/Y$.
  Let~$(X', f', g')$ be a triple consisting of an object~$X \in \Ob(\Ccat)$ and two morphisms~~$Y' \xfrom{f'} X' \xto{g'} X$ such that the square
  \begin{equation}
    \label{pullback to product}
    \begin{tikzcd}
        X'
        \arrow[dashed]{r}[above]{g'}
        \arrow[dashed]{d}[left]{f'}
      & X
        \arrow{d}[right]{f}
      \\
        Y'
        \arrow{r}[above]{g}
      & Y
    \end{tikzcd}
  \end{equation}
  commutes.
  For~$h \defined f \circ g' = g \circ f'$ we then have the following commutative diagram:
  \[
    \begin{tikzcd}
        Y'
        \arrow{dr}[below left]{g}
      & X'
        \arrow[dashed]{l}[above]{f'}
        \arrow{d}[right]{h}
        \arrow[dashed]{r}[above]{g'}
      & X
        \arrow{dl}[below right]{f}
      \\
        {}
      & Y
      & {}
    \end{tikzcd}
  \]
  The two morphisms~$f'$ and~$g'$ are therefore also morphisms~$f' \colon (X',h) \to (Y',g)$ and~$g' \colon (X',h) \to (X,f)$ in~$\Ccat/Y$.
  That the diagram~\eqref{pullback to product} is a pullback square means precisely that the triple~$((X',h), f', g')$ is a product of the two objects~$(X,f)$ and~$(Y,g)$ in the category~$\Ccat/Y$.
\end{remark*}


\begin{lemma*}[Transitivity of pullback and pushout]
  \label{transitivity of pullback and pushout}
  Let~$\Ccat$ be a category.
  \begin{enumerate}
    \item
      Let
      \[
        \begin{tikzcd}
            X''
            \arrow{r}[above]{h'}
            \arrow{d}[right]{f''}
          & X'
            \arrow{r}[above]{h}
            \arrow{d}[right]{f'}
          & X
            \arrow{d}[right]{f}
          \\
            Y''
            \arrow{r}[above]{g'}
          & Y'
            \arrow{r}[above]{g}
          & Y
        \end{tikzcd}
      \]
      be a commutative diagram in~$\Ccat$.
      \begin{enumerate}
        \item
          If the left square and the right square are both pullback squares, then the outer reactangle is again a pullback square.
        \item
          If the right square and the outer reactangle are both pullback squares, then the left square is also a pullback square.
      \end{enumerate}
    \item
      \label{pasting pushouts}
      Let
      \[
        \begin{tikzcd}
            X
            \arrow{r}[above]{g}
            \arrow{d}[left]{f}
          & X'
            \arrow{r}[above]{g'}
            \arrow{d}[left]{f'}
          & X''
            \arrow{d}[left]{f''}
          \\
            Y
            \arrow{r}[above]{h}
          & Y'
            \arrow{r}[above]{h'}
          & Y''
        \end{tikzcd}
      \]
      be a commutative diagram in~$\Ccat$.
      \begin{enumerate}
        \item
          If the right square and the left square are both pushout diagrams, then the outer rectangle is again a pushout square.
        \item
          If the left square and outer reactangle are both pushout squares, then the right square is also a pushout square.
      \end{enumerate}
  \end{enumerate}
\end{lemma*}


\begin{proof}
  \leavevmode
  \begin{enumerate}
    \item
      This is Exercise~1 on Exercise~sheet~6.
    \item
      This follows from part~\ref*{pasting pushouts} by duality.
    \qedhere
  \end{enumerate}
\end{proof}


% TODO: Counterexamples for the other pullback lemma.


\begin{lemma*}
  Let~$\Acat$ be an additive category that has kernels and cokernels.
  Then the category~$\Acat$ has pullbacks and pushouts.
\end{lemma*}


\begin{proof}
  This is Exercise~3 on the Exercise~sheet~6.
\end{proof}


\begin{proposition*}
  \label{mono epi under pull push}
  Let~$\Ccat$ be a category.
  \begin{enumerate}
    \item
      \label{pullback of mono}
      Let
      \[
        \begin{tikzcd}
            X'
            \arrow{r}[above]{g'}
            \arrow{d}[right]{f'}
          & X
            \arrow{d}[right]{f}
          \\
            Y'
            \arrow{r}[above]{g}
          & Y
        \end{tikzcd}
      \]
      be a pullback square in~$\Ccat$.
      If~$f$ is a monomorphism then~$f'$ is again a monomorphism.
    \item
      Let
      \[
        \begin{tikzcd}
            X
            \arrow{r}[above]{g}
            \arrow{d}[left]{f}
          & X'
            \arrow{d}[left]{f'}
          \\
            Y
            \arrow{r}[above]{g'}
          & Y'
        \end{tikzcd}
      \]
      be a pushout square in~$\Ccat$.
      If~$f$ is an epimorphism then~$f'$ is again an epimorphism.
  \end{enumerate}
  In an abelian category~$\Acat$ the same holds when we switch \enquote{monomorphism} and \enquote{epimorphism} in the above statements:
  \begin{enumerate}[resume]
    \item
      \label{pullback of epi}
      Let
      \[
        \begin{tikzcd}
            X'
            \arrow{r}[above]{g'}
            \arrow{d}[right]{f'}
          & X
            \arrow{d}[right]{f}
          \\
            Y'
            \arrow{r}[above]{g}
          & Y
        \end{tikzcd}
      \]
      be a pullback square in~$\Acat$.
      If~$f$ is an epimorphism then~$f'$ is again a epimorphism.
    \item
      \label{pushout of mono}
      Let
      \[
        \begin{tikzcd}
            X
            \arrow{r}[above]{g}
            \arrow{d}[left]{f}
          & X'
            \arrow{d}[left]{f'}
          \\
            Y
            \arrow{r}[above]{g'}
          & Y'
        \end{tikzcd}
      \]
      is a pushout square in~$\Acat$.
      If~$f$ is a monomorphism then~$f'$ is again a monomorphism.
  \end{enumerate}
\end{proposition*}


\begin{proof}
  \leavevmode
  \begin{enumerate}
    \item
      This is Exercise~2 on the Exercise~sheet~2.
    \item
      This is the dual statement to part~\ref*{pullback of mono}.
    \item
      This is Exercise~4 on the Exercise~sheet~6.
    \item
      This is the dual statement to part~\ref*{pullback of epi}.
    \qedhere
  \end{enumerate}
\end{proof}


\begin{remark*}[Functoriality of pullback and pushout]
  \label{functoriality of pullback and pushout}
  Let~$\Ccat$ be a category.
  \begin{enumerate}
    \item
      Suppose that we are given two diagrams
      \[
        \begin{tikzcd}
            {}
          & X_1
            \arrow{d}
          \\
            Y'_1
            \arrow{r}
          & Y_1
        \end{tikzcd}
        \qquad\text{and}\qquad
        \begin{tikzcd}
            {}
          & X_2
            \arrow{d}
          \\
            Y'_2
            \arrow{r}
          & Y_2
        \end{tikzcd}
      \]
      in~$\Ccat$ and morphisms~$X_1 \to X_2$,~$Y_1 \to Y_2$ and~$Y'_1 \to Y'_2$ that make the resulting diagram
      \begin{equation}
        \label{morphism between prepullbacks}
        \begin{tikzcd}[cramped]
            {}
          & {}
          & X_1
            \arrow{dd}
            \arrow[dashed]{dr}
          & {}
          \\
            {}
          & {}
          & {}
          & X_2
            \arrow{dd}
          \\
            Y'_1
            \arrow{rr}
            \arrow[dashed]{dr}
          & {}
          & Y_1
            \arrow[dashed]{dr}
          & {}
          \\
            {}
          & Y'_2
            \arrow{rr}
          & {}
          & Y_2
          \\
        \end{tikzcd}
      \end{equation}
      commute.
      If the pullbacks
      \[
        \begin{tikzcd}
            X'_1
            \arrow[dashed]{r}
            \arrow[dashed]{d}
          & X_1
            \arrow{d}
          \\
            Y'_1
            \arrow{r}
          & Y_1
        \end{tikzcd}
        \qquad\text{and}\qquad
        \begin{tikzcd}
            X'_2
            \arrow[dashed]{r}
            \arrow[dashed]{d}
          & X_2
            \arrow{d}
          \\
            Y'_2
            \arrow{r}
          & Y_2
        \end{tikzcd}
      \]
      exist, then there exists a unique morphism~$X'_1 \to X'_2$ that makes the resulting cube
      \[
        \begin{tikzcd}[cramped]
            X'_1
            \arrow{rr}
            \arrow[dashed]{dr}
            \arrow{dd}
          & {}
          & X_1
            \arrow{dd}
            \arrow{dr}
          & {}
          \\
            {}
          & X'_2
          & {}
          & X_2
            \arrow[from=ll, crossing over]
            \arrow{dd}
          \\
            Y'_1
            \arrow{rr}
            \arrow{dr}
          & {}
          & Y_1
            \arrow{dr}
          & {}
          \\
            {}
          & Y'_2
            \arrow[from=uu, crossing over]
            \arrow{rr}
          & {}
          & Y_2
          \\
        \end{tikzcd}
      \]
      commute.
      Both the existence and uniqueness follow by the universal property of the pullback (applied to~$X'_2$) from the commutativity of the diagram~\eqref{morphism between prepullbacks}.
      This induced morphism is functorial in the following sense:
      \begin{itemize}
        \item
          Let
          \[
            \begin{tikzcd}
                {}
              & X
                \arrow{d}
              \\
                Y'
                \arrow{r}
              & Y
            \end{tikzcd}
          \]
          be a diagram in~$\Ccat$ whose pushout
          \[
            \begin{tikzcd}
                X'
                \arrow[dashed]{r}
                \arrow[dashed]{d}
              & X
                \arrow{d}
              \\
                Y'
                \arrow{r}
              & Y
            \end{tikzcd}
          \]
          exists.
          Then the diagram
          \[
            \begin{tikzcd}[cramped]
                {}
              & {}
              & X
                \arrow{dd}
                \arrow[dashed]{dr}[above right]{\id_X}
              & {}
              \\
                {}
              & {}
              & {}
              & X
                \arrow{dd}
              \\
                Y'
                \arrow{rr}
                \arrow[dashed]{dr}[below left]{\id_{Y'}}
              & {}
              & Y
                \arrow[dashed]{dr}[above right]{\id_Y}
              & {}
              \\
                {}
              & Y'
                \arrow{rr}
              & {}
              & Y
              \\
            \end{tikzcd}
          \]
          commutes, and the induced morphism~$X' \to X'$ that makes the cube
          \[
            \begin{tikzcd}[cramped]
                X'
                \arrow{rr}
                \arrow[dashed]{dr}
                \arrow{dd}
              & {}
              & X
                \arrow{dd}
                \arrow{dr}[above right]{\id_X}
              & {}
              \\
                {}
              & X'
              & {}
              & X
                \arrow[from=ll, crossing over]
                \arrow{dd}
              \\
                Y'
                \arrow{rr}
                \arrow{dr}[below left]{\id_{Y'}}
              & {}
              & Y
                \arrow{dr}[above right]{\id_Y}
              & {}
              \\
                {}
              & Y'
                \arrow[from=uu, crossing over]
                \arrow{rr}
              & {}
              & Y
              \\
            \end{tikzcd}
          \]
          commute is the identity~$\id_{X'}$.
        \item
          Let
          \[
            \begin{tikzcd}
                {}
              & X_1
                \arrow{d}
              \\
                Y'_1
                \arrow{r}
              & Y_1
            \end{tikzcd}
            \qquad\text{and}\qquad
            \begin{tikzcd}
                {}
              & X_2
                \arrow{d}
              \\
                Y'_2
                \arrow{r}
              & Y_2
            \end{tikzcd}
            \qquad\text{and}\qquad
            \begin{tikzcd}
                {}
              & X_3
                \arrow{d}
              \\
                Y'_3
                \arrow{r}
              & Y_3
            \end{tikzcd}
          \]
          be diagram in~$\Ccat$ whose pullbacks
          \[
            \begin{tikzcd}
                X'_1
                \arrow[dashed]{r}
                \arrow[dashed]{d}
              & X_1
                \arrow{d}
              \\
                Y'_1
                \arrow{r}
              & Y_1
            \end{tikzcd}
            \qquad\text{and}\qquad
            \begin{tikzcd}
                X'_2
                \arrow[dashed]{r}
                \arrow[dashed]{d}
              & X_2
                \arrow{d}
              \\
                Y'_2
                \arrow{r}
              & Y_2
            \end{tikzcd}
            \qquad\text{and}\qquad
            \begin{tikzcd}
                X'_3
                \arrow[dashed]{r}
                \arrow[dashed]{d}
              & X_3
                \arrow{d}
              \\
                Y'_3
                \arrow{r}
              & Y_3
            \end{tikzcd}
          \]
          exist.
          Suppose that we are given morphisms
          \[
            X_i \to X_{i+1} \,,
            \quad
            Y_i \to Y_{i+1} \,,
            \quad
            Y'_i \to Y'_{i+1}
          \]
          that make the resulting diagram
          \[
            \begin{tikzcd}[cramped]
                {}
              & {}
              & {}
              & X_1
                \arrow[dashed]{dr}
                \arrow{ddd}
              & {}
              & {}
              \\
                {}
              & {}
              & {}
              & {}
              & X_2
                \arrow[dashed]{dr}
                \arrow{ddd}
              & {}
              \\
                {}
              & {}
              & {}
              & {}
              & {}
              & X_3
                \arrow{ddd}
              \\
                Y'_1
                \arrow{rrr}
                \arrow[dashed]{dr}
              & {}
              & {}
              & Y_1
                \arrow[dashed]{dr}
              & {}
              & {}
              \\
                {}
              & Y'_2
                \arrow{rrr}
                \arrow[dashed]{dr}
              & {}
              & {}
              & Y_2
                \arrow[dashed]{dr}
              & {}
              \\
                {}
              & {}
              & Y'_3
                \arrow{rrr}
              & {}
              & {}
              & Y_3
            \end{tikzcd}
          \]
          commutes.
          Let~$f \colon X'_1 \to X'_2$ and~$g \colon X'_2 \to X'_3$ be the induced morphisms that make the diagram
          \[
            \begin{tikzcd}[cramped]
                X'_1
                \arrow{rrr}
                \arrow[dashed]{dr}[above right]{f}
                \arrow{ddd}
              & {}
              & {}
              & X_1
                \arrow{dr}
                \arrow{ddd}
              & {}
              & {}
              \\
                {}
              & X'_2
                \arrow[dashed]{dr}[above right]{g}
              & {}
              & {}
              & X_2
                \arrow[from=lll, crossing over]
                \arrow{dr}
                \arrow{ddd}
              & {}
              \\
                {}
              & {}
              & X'_3
              & {}
              & {}
              & X_3
                \arrow[from=lll, crossing over]
                \arrow{ddd}
              \\
                Y'_1
                \arrow{rrr}
                \arrow{dr}
              & {}
              & {}
              & Y_1
                \arrow{dr}
              & {}
              & {}
              \\
                {}
              & Y'_2
                \arrow[from=uuu, crossing over]
                \arrow{rrr}
                \arrow{dr}
              & {}
              & {}
              & Y_2
                \arrow{dr}
              & {}
              \\
                {}
              & {}
              & Y'_3
                \arrow[from=uuu, crossing over]
                \arrow{rrr}
              & {}
              & {}
              & Y_3
            \end{tikzcd}
          \]
          commute.
          It then follows from the commutativity of the subcube
          \[
            \begin{tikzcd}[cramped]
                X'_1
                \arrow{rr}
                \arrow[dashed]{dr}[above right]{gf}
                \arrow{dd}
              & {}
              & X_1
                \arrow{dd}
                \arrow{dr}
              & {}
              \\
                {}
              & X'_3
              & {}
              & X_3
                \arrow[from=ll, crossing over]
                \arrow{dd}
              \\
                Y'_1
                \arrow{rr}
                \arrow{dr}
              & {}
              & Y_1
                \arrow{dr}
              & {}
              \\
                {}
              & Y'_3
                \arrow[from=uu, crossing over]
                \arrow{rr}
              & {}
              & Y_3
              \\
            \end{tikzcd}
          \]
          that~$gf$ is the morphism~$X'_1 \to X'_3$ induced by the commutativity of the following diagram:
          \[
            \begin{tikzcd}[cramped]
                {}
              & {}
              & X_1
                \arrow{dd}
                \arrow[dashed]{dr}
              & {}
              \\
                {}
              & {}
              & {}
              & X_3
                \arrow{dd}
              \\
                Y'_1
                \arrow{rr}
                \arrow[dashed]{dr}
              & {}
              & Y_1
                \arrow[dashed]{dr}
              & {}
              \\
                {}
              & Y'_3
                \arrow{rr}
              & {}
              & Y_3
              \\
            \end{tikzcd}
          \]
      \end{itemize}
    \item
      We observe similarly the functoriality of the pushout:
      Suppose that
      \[
        \begin{tikzcd}
            X'_1
            \arrow{r}
            \arrow{d}
          & Y'_1
          \\
            X_1
          & {}
        \end{tikzcd}
        \qquad\text{and}\qquad
        \begin{tikzcd}
            X'_2
            \arrow{r}
            \arrow{d}
          & Y'_2
          \\
            X_2
          & {}
        \end{tikzcd}
      \]
      are diagrams in~$\Ccat$ whose pushouts
      \[
        \begin{tikzcd}
            X'_1
            \arrow{r}
            \arrow{d}
          & Y'_1
            \arrow[dashed]{d}
          \\
            X_1
            \arrow[dashed]{r}
          & Y_1
        \end{tikzcd}
        \qquad\text{and}\qquad
        \begin{tikzcd}
            X'_2
            \arrow{r}
            \arrow{d}
          & Y'_2
            \arrow[dashed]{d}
          \\
            X_2
            \arrow[dashed]{r}
          & Y_2
        \end{tikzcd}
      \]
      exist.
      If~$X_1 \to X_2$,~$Y_1 \to Y_2$ and~$X'_1 \to X'_2$ are morphisms that make the diagram
      \[
        \begin{tikzcd}[cramped]
            X'_1
            \arrow{rr}
            \arrow[dashed]{dr}
            \arrow{dd}
          & {}
          & X_1
            \arrow[dashed]{dr}
          & {}
          \\
            {}
          & X'_3
          & {}
          & X_3
            \arrow[from=ll, crossing over]
          \\
            Y'_1
            \arrow[dashed]{dr}
          & {}
          & {}
          & {}
          \\
            {}
          & Y'_3
            \arrow[from=uu, crossing over]
          & {}
          & {}
          \\
        \end{tikzcd}
      \]
      commute, then there exists a unique morphism~$Y'_1 \to Y'_1$ that makes the resulting cube
      \[
        \begin{tikzcd}[cramped]
            X'_1
            \arrow{rr}
            \arrow{dr}
            \arrow{dd}
          & {}
          & X_1
            \arrow{dd}
            \arrow{dr}
          & {}
          \\
            {}
          & X'_3
          & {}
          & X_3
            \arrow[from=ll, crossing over]
            \arrow{dd}
          \\
            Y'_1
            \arrow{rr}
            \arrow{dr}
          & {}
          & Y_1
            \arrow[dashed]{dr}
          & {}
          \\
            {}
          & Y'_3
            \arrow[from=uu, crossing over]
            \arrow{rr}
          & {}
          & Y_3
          \\
        \end{tikzcd}
      \]
      commute.
      We also find that this induced morphism is functorial in the same sense as for the pullback.
  \end{enumerate}
\end{remark*}

