\chapter{Abelian Categories}





\section{Monomorphisms and Epimorphisms}


\begin{definition}
  Let~$f \colon X \to Y$ be a morphism in a category~$\Ccat$.
  \begin{enumerate}
    \item
      The morphism~$f$ is a \emph{monomorphism}\index{monomorphism}\index{morphism!mono-} if it follows for every pair of parallel morphisms~$u, v \colon Z \to X$ from~$f \circ u = f \circ v$ that already~$u = v$.
    \item
      The morphism~$f$ is an \emph{epimorphism}\index{epimorphism}\index{morphism!epi-} if it follows for every pair of parallel morphisms~$u, v \colon Y \to Z$ from~$u \circ f = v \circ f$ that already~$u = v$.
  \end{enumerate}
\end{definition}


\begin{remark}
  Let~$f \colon X \to Y$ and~$g \colon Y \to Z$ be composable morphisms in a category~$\Ccat$.
  \begin{enumerate}
    \item
      If~$f$ is an isomorphism then it is both a monomorphism and an epimorphism.
    \item
      If both~$f$ and~$g$ are monomorphisms then their composition~$g \circ f$ is again a monomorphism.
      If both~$f$ and~$g$ are epimorphisms then their composition~$g \circ f$ is again an epimorphism.
    \item
      If the composition~$g \circ f$ is a monomorphism then~$f$ is a monomorphism.
      If the composition~$g \circ f$ is an epimorphism then~$g$ is an epimorphism.
    \item
      The morphism~$f$ is a monomorphism (in~$\Ccat$) if and only if it is an epimorphism in~$\Ccat^\op$.
  \end{enumerate}
\end{remark}


\begin{example}
  We give examples of monomorphisms.
  \begin{enumerate}
    \item
      In the category~$\Set$ the monomorphisms are precisely the injective maps.
      The same holds for the categories~$\Modl{A}$,~$\Group$,~$\Ring$,~$\CommRing$,~$\kAlg$,~$\kCommAlg$,~$\Top$.
    \item
      If~$Q$ is a quiver then in its path category~$\Path(Q)$ every morphism is a monomorphism:
      Let~$p = \alpha_\ell \dotsm \alpha_1$ be a morphism in~$Q$, i.e.\ a path in~$Q$.
      If~$u = u_r \dotsm u_1$ and~$v = v_s \dotsm v_1$ are morphisms in~$\Path(Q)$, i.e.\ paths in~$Q$, with~$s(u) = s(v)$ and~$t(u) = t(v) = s(p)$ then the equality~$p \circ u = p \circ v$ means that
      \[
          \alpha_\ell \dotsm \alpha_1 u_r \dotsm u_1
        = \alpha_\ell \dotsm \alpha_1 v_s \dotsm v_1 \,.
      \]
      It then follows that~$r = s$ and~$u_i = v_i$ for all~$i = 1, \dotsc, r$.
    \item
      Let~$\Conn_*$ be the category of pointed, connected topological spaces:
      The objects of~$\Conn_*$ are pairs~$(X, x_0)$ consisting of a connected topological space~$X$ and a base point~$x_0 \in X$.
      A morphism~$f \colon (X, x_0) \to (Y, y_0)$ is a continuous map~$f \colon X \to Y$ with~$f(x_0) = y_0$.
      The morphism~$f \colon (\Real, 0) \to (S^1, 1)$ with~$f(x) = e^{2 \pi i x}$ is then a monomorphism.
  \end{enumerate}
\end{example}


\begin{example}
  We give examples for epimorphisms.
  \begin{enumerate}
    \item
      In the category~$\Set$ a morphism is an epimorphism if and only if it surjective.
    \item
      If~$Q$ is a quiver then in its path category~$\Path(Q)$ every morphism in an epimorphism.
    \item
      Let~$\Haus$ be the category of Hausdorff topological spaces (where morphisms are just continuous maps).
      A morphism~$f \colon X \to Y$ in~$\Haus$ is an epimorphism if and only if it has dense image.
    \item
      Let~$A$ be a commutative ring and let~$S \subseteq A$ be a multiplicative set.
      Then the canonical map~$f \colon A \to S^{-1} A$,~$a \mapsto a/1$ is an epimorphism:
      If~$u,v \colon S^{-1} A \to B$ are two ring homomorphisms with~$u \circ f = v \circ f$ then~$u(a/1) = v(a/1)$ for every~$a \in A$.
      It then follows for every fraction~$a/s \in S^{-1} A$ that
      \[
          u\left( \frac{a}{s} \right)
        = u\left( \frac{a}{1} \right) u\left( \frac{s}{1} \right)^{-1}
        = v\left( \frac{a}{1} \right) v\left( \frac{s}{1} \right)^{-1}
        = v\left( \frac{a}{s} \right) \,,
      \]
      which then shows that~$u = v$.
  \end{enumerate}
\end{example}





\section{Special Objects}


\begin{definition}
  Let~$X$ be an object in a category~$\Ccat$.
  \begin{enumerate}
    \item
      The object~$X$ is \emph{initial}\index{initial object}\index{object!initial} if there exists  for every object~$Y \in \Ob(\Ccat)$ a unique morphism~$X \to Y$ in~$\Ccat$.
    \item
      The object~$X$ is \emph{terminal}\index{terminal object}\index{object!terminal} or \emph{final}\index{final object}\index{object!final} if there exists for every object~$Y \in \Ob(\Ccat)$ a unique morphism~$Y \to X$ in~$\Ccat$.
    \item
      The object~$X$ is a \emph{zero object}\index{zero!object}\index{object!zero} if it is both initial and terminal.
  \end{enumerate}
\end{definition}


\begin{remark}
  Let~$\Ccat$ be a category.
  \begin{enumerate}
    \item
      An object~$X$ of~$\Ccat$ is initial (in~$\Ccat$) if and only if it is terminal in~$\Ccat^\op$.
    \item
      Initial and terminal objects are unique up to unique isomorphisms (if they exist).
    \item
      If~$\Ccat$ admits a zero object then it is denoted by~$0 = 0_\Ccat$.
  \end{enumerate}
\end{remark}


\begin{example}
  \leavevmode
  \begin{enumerate}
    \item
      In the category~$\Set$, the empty set~$\emptyset$ is the unique initial object, and every \dash{one}{point} set~$\{\ast\}$ is a final object.
    \item
      In the category~$\Modl{A}$ the zero module~$0$ is the zero object.
    \item
      In the category~$\kAlg$ the~{\kalg}~$\kf$ is initial, while the zero algebra~$0$ is final.
  \end{enumerate}
\end{example}


\begin{remarkdefinition}
  Let~$\Ccat$ be a category which admits a zero object~$0$.
  Then there exist for any two objects~$X, Y \in \Ob(\Ccat)$ a unique morphism~$0_{XY} \colon X \to Y$ which factors trough the zero object, i.e.\ which makes the triangle
  \[
    \begin{tikzcd}
        X
        \arrow{dr}
        \arrow{rr}[above]{0_{XY}}
      & {}
      & Y
      \\
        {}
      & 0
        \arrow{ur}
      & {}
    \end{tikzcd}
  \]
  commute.
  The morphism~$0_{XY}$ is the \emph{zero morphism}\index{zero!morphism}\index{morphism!zero} from~$X$ to~$Y$.
\end{remarkdefinition}






\lecturend{8}





\section{Products and Coproducts}

\begin{definition}
  Let~$(X_i)_{i \in I}$ be a family of objects in a category~$\Ccat$.
  \begin{enumerate}
    \item
      A \emph{product}\index{product} of the family of objects~$(X_i)_{i \in I}$ is a pair~$(P, (p_i)_{i \in I})$ consisting of an object~$P \in \Ob(\Ccat)$ and morphisms~$p_i \colon P \to X_i$, such that for every other pair~$(Q, (q_i)_{i \in I})$ consisting of an object~$Q \in \Ob(\Ccat)$ and morphisms~$q_i \colon Q \to X_i$ there exists a unique morphism~$\lambda \colon Q \to P$ which makes the triangle
      \[
        \begin{tikzcd}[sep = large]
            P
            \arrow{d}[left]{p_i}
          & Q
            \arrow{dl}[below right]{q_i}
            \arrow[dashed]{l}[above]{\lambda}
          \\
            X_i
          & {}
        \end{tikzcd}
%         \begin{tikzcd}[sep = large]
%             Q
%             \arrow{dr}[above right]{q_i}
%             \arrow[dashed]{d}[left]{\lambda}
%           & {}
%           \\
%             P
%             \arrow{r}[above]{p_i}
%           & X_i
%         \end{tikzcd}
      \]
      commute for every~$i \in I$.
    \item
      A \emph{coproduct}\index{coproduct} of the family of objects~$(X_i)_{i \in I}$ is a pair~$(C, (c_i)_{i \in I})$ consisting of an object~$C \in \Ob(\Ccat)$ and morphisms~$c_i \colon X_i \to C$, such that for every pair~$(D, (d_i)_{i \in I})$ consisting of an object~$D \in \Ob(\Ccat)$ and morphisms~$d_i \colon X_i \to C$ there exists a unique morphism~$\mu \colon C \to D$ which makes the triangle
      \[
        \begin{tikzcd}[sep = large]
            C
            \arrow[dashed]{r}[above]{\mu}
          & D
          \\
            X_i
            \arrow{u}[left]{c_i}
            \arrow{ur}[below right]{d_i}
          & {}
        \end{tikzcd}
      \]
      commute for every~$i \in I$.
  \end{enumerate}
\end{definition}


\begin{remark}
  Let~$(X_i)_{i \in I}$ be a family of objects~$X_i$ in a category~$\Ccat$.
  \begin{enumerate}
    \item
      A pair~$(P,(p_i)_{i \in I})$ is a product of the family~$(X_i)_{i \in I}$ in~$\Ccat$ if and only if it is a coproduct of this family in~$\Ccat^\op$.
    \item
      Products are unique up to unique isomorphism, i.e.\ if~$(P, (p_i)_i)$ and~$(P', (p'_i)_{i \in I})$ are two products of the family~$(X_i)_{i \in I}$ in~$\Ccat$ then there exist a unique morphism~$\lambda \colon P \to P'$ which makes the triangle
      \[
        \begin{tikzcd}
            P
            \arrow[dashed]{rr}[above]{\lambda}
            \arrow{dr}[below left]{p_i}
          & {}
          & P'
            \arrow{dl}[below right]{p'_i}
          \\
            {}
          & X_i
          & {}
        \end{tikzcd}
      \]
      commute for every~$i \in I$, and the morphism~$\lambda$ is already an isomorphism.
      Similarly, coproducts are unique up to unique isomorphism.
    \item
      The product of the family~$(X_i)_{i \in I}$ is denoted by~$\prod_{i \in I} X_i$, or by~$X_1 \times \dotsb \times X_n$ if~$I = \{1, \dotsc, n\}$.
      The coproduct of the family~$(X_i)_{i \in I}$ is denoted by~$\coprod_{i \in I} X_i$, or by~$X_1 \dcup \dotsb \dcup X_n$ if~$I = \{1, \dotsc, n\}$.%
      \footnote{In the lecture the notations~$\bigsqcap_{i \in I} X_i$, resp.~$\bigsqcup_{i \in I} X_i$ and~$X_1 \sqcup \dotsb \sqcup X_n$ are used instead.}
    \item
      If every family of objects in~$\Ccat$ has a product (resp.\ coproduct) in~$\Ccat$ then we say that that~$\Ccat$ \emph{has products} (resp.\ \emph{has coproducts}).
      If every finite family of objects in~$\Ccat$ has a product (resp.\ coproducts) in~$\Ccat$ then we say that~$\Ccat$ \emph{has finite coproducts} (resp.\ \emph{has finite coproducts}).
  \end{enumerate}
\end{remark}


\begin{example}
  In the following let~$(X_i)_{i \in I}$ be a family objects in the given category~$\Ccat$.
  \begin{enumerate}
    \item
      Let~$\Ccat = \Set$.
      Then the (categorical) product~$\prod_{ \in I} X_i$ is the cartesian product, and the map~$p_i \colon \prod_{j \in I} X_j \to X_i$ is for every~$i \in I$ the usual projections onto the~\dash{$i$}{th} factor.
      The coproduct~$\coprod_{i \in I} X_i$ is the (formal) disjoint union of the sets~$X_i$, and the map~$c_i \colon X_i \to \coprod_{j \in I} X_j$ is for every~$i \in I$ the inclusion into the~\dash{$j$}{th} set.
    \item
      Let~$\Ccat = \Modl{A}$.
      Then the (categorical) product~$\prod_{i \in I} X_i$ is the cartesian product, and the morphism~$p_i \colon \prod_{j \in I} X_j \to X_i$ is for every~$i \in I$ the usual projection onto the~\dash{$i$}{th} factor.
      The coproduct of the family~$(X_i)_{i \in I}$ is the direct sum~$\bigoplus_{i \in I} X_i$, and the morphisms~$c_i \colon X_i \to \bigoplus_{j \in I} X_j$ is for every~$i \in I$ the inclusion into the~\dash{$i$}{th} summand.
    \item
      Let~$\Ccat = \kCommAlg$.
      Then the (categorical) product~$\prod_{i \in I} X_i$ is the cartesian product, and the morphism~$p_i \colon \prod_{j \in I} X_j \to X_i$ is for every~$i \in I$ the usual projection onto the~\dash{$i$}{th} factor.
      The coproduct of finitely many commutative~{\kalg}~$A_1, \dotsc, A_n$ in the category~$\kCommAlg$ is their tensor product~$A_1 \tensor \dotsb \tensor A_n$, and the morphism~$c_i \colon A_i \to A_1 \tensor \dotsb \tensor A_n$ is for every~$i \in I$ the inclusion into the~\dash{$i$}{th} factor, i.e.\ the algebra homomorphism
      \[
                A_i
        \to     A_1 \tensor \dotsb \tensor A_n \,,
        \quad   x
        \mapsto 1 \tensor \dotsb \tensor 1 \tensor x \tensor 1 \tensor \dotsb \tensor 1 \,.
      \]
      (The coproduct of an arbitrary family~$(A_i)_{i \in I}$ of commutative~{\kalgs} in the category~$\kCommAlg$ can be described similarly.)
  \end{enumerate}
\end{example}


\begin{remark}
  Let~$\Ccat$ be a category.
  A product of an empty family of objects in~$\Ccat$ is the same a terminal object of~$\Ccat$.
  A coproduct of an empty family of objects in~$\Ccat$ is the same an initial object of~$\Ccat$.
\end{remark}


\begin{lemma}
  \label{existence of coproducts}
  Let~$(X_i)_{i \in I}$ be a family of objects in a category~$\Ccat$.
  \begin{enumerate}
    \item
      The following are equivalent:
      \begin{enumerate}
        \item
          The product~$\prod_{i \in I} X_i$ exists in~$\Ccat$.
        \item
          The functor~$\Ccat^\op \to \Set$ given by~$Y \mapsto \prod_{i \in I} \Ccat(Y,X_i)$ is representable.
      \end{enumerate}
    \item
      \label{for coproducts}
      The following are equivalent:
      \begin{enumerate}
        \item
          The coproduct~$\coprod_{i \in I} X_i$ exists in~$\Ccat$.
        \item
          The functor~$\Ccat \to \Set$ given by~$Y \mapsto \prod_{i \in I} \Ccat(X_i,Y)$ is representable.
      \end{enumerate}
  \end{enumerate}
\end{lemma}


\begin{proof}
  It sufficies to show part~\ref*{for coproducts}.
  We denote the given functor~$\Ccat \to \Set$ by~$F$.
  
  Suppose first that the coproduct~$\coprod_{i \in I} X_i$ exists, and denote the associated morphisms by~$c_i \colon X_i \to \coprod_{j \in I} X_j$.
  We claim that the functor~$F$ is represented by the coproduct~$\coprod_{i \in I} X_i$;
  we thus need to construct a natural isomorphism~$\eta \colon h^{(\coprod_{i \in I} X_i)} \to F$.
  We define the components of~$\eta$ as
  \begin{align*}
              \eta_Y
     \colon   h^{(\coprod_{i \in I} X_i)}(Y)
     =        \Ccat\left( \coprod_{i \in I} X_i, Y \right)
    &\to      \prod_{i \in I} \Ccat(X_i, Y)
     =        F(Y) \,,
     \\
              f
    &\mapsto  (f \circ c_i)_{i \in I} \,.
  \end{align*}
  Then the family~$\eta \defined (\eta_Y)_{Y \in \Ob(\Ccat)}$ defines a natural transformation~$\eta \colon h^{(\coprod_{i \in I} X_i)} \to F$.
  Indeed, for every morphism~$g \colon Y \to Y'$ in~$\Ccat$ the diagram
  \[
    \begin{tikzcd}
        h^{(\coprod_{i \in I} X_i)}(Y)
        \arrow{rrr}[above]{h^{(\coprod_{i \in I} X_i)}(g)}
        \arrow{ddd}[left]{\eta_Y}
        \arrow[equal]{dr}
      & {}
      & {}
      & h^{(\coprod_{i \in I} X_i)}(Y')
        \arrow{ddd}[right]{\eta_{Y'}}
        \arrow[equal]{dl}
      \\
        {}
      & \Ccat\left( \coprod_{i \in I} X_i, Y \right)
        \arrow{r}[above]{g_*}
        \arrow{d}[left]{\eta_Y}
      & \Ccat\left( \coprod_{i \in I} X_i, Y' \right)
        \arrow{d}[right]{\eta_{Y'}}
      & {}
      \\
        {}
      & \prod_{i \in I} \Ccat(X_i, Y)
        \arrow{r}[below]{\prod_{i \in I} g_*}
      & \prod_{i \in I} \Ccat(X_i, Y')
      & {}
      \\
        F(Y)
        \arrow[equal]{ur}
        \arrow{rrr}[below]{F(g)}
      & {}
      & {}
      & F(Y')
        \arrow[equal]{ul}
    \end{tikzcd}
  \]
  commutes, because the inner square is given on elements by
  \[
    \begin{tikzcd}[column sep = huge, row sep = large]
        f
        \arrow[maps to]{r}[above]{g_*}
        \arrow[maps to]{d}[left]{\eta_Y}
      & g \circ f
        \arrow[maps to]{d}[right]{\eta_{Y'}}
      \\
        (f \circ c_i)_{i \in I}
        \arrow[maps to]{r}[above]{\prod_{i \in I} g_*}
      & (g \circ f \circ c_i)_{i \in I}
    \end{tikzcd}
  \]
  and thus commutes.
  
  The natural transformation~$\eta$ is already a natural isomorphism:
  There exist at every objects~$Y \in \Ob(\Ccat)$ for every family of morphisms~$(h_i)_{i \in I} \in \prod_{i \in I} \Ccat(X_i, Y)$ by the definition of the coproduct~$\coprod_{i \in I} X_i$ a unique morphism~$g \in \Ccat(\coprod_{i \in I} X_i, Y)$ with~$g \circ c_i = h_i$ for every~$i \in I$, i.e.\ with~$\eta_Y(g) = (h_i)_{i \in I}$.
  This means that~$\eta_Y$ is bijective at every~$Y \in \Ob(\Ccat)$.
  
  Suppose now that the functor~$F$ is representable.
  Let~$C$ be a representing object for~$F$ and let~$\eta \colon h^C \to F$ be a natural isomorphism.
  Then~$\eta_C$ is a map
  \[
      \eta_C
    \colon
      h^C(C)
    =
      \Ccat(C,C)
    \to
      F(C)
    =
      \prod_{i \in I} \Ccat(X_i, C) \,.
  \]
  By setting $(c_i)_{i \in I} \defined \eta_C(\id_C)$ we therefore get for every~$i \in I$ a morphism~$c_i \colon X_i \to C$.
  We show that the pair~$(C, (c_i)_{i \in I})$ is a coproduct of the family~$(X_i)_{i \in I}$.
  So let~$(D, (d_i)_{i \in I})$ be another pair consisting of an object~$D \in \Ob(\Dcat)$ and a family~$(d_i)_{i \in I}$ of morphisms~$d_i \colon X_i \to D$.
  Then
  \[
        (d_i)_{i \in I}
    \in \prod_{i \in I} \Ccat(X_i, D)
    =   F(D)
  \]
  and it follows from~$\eta_D \colon h^C(D) \to F(D)$ being a bijection that there exist a unique element~$\lambda \in h^C(D) = \Ccat(C,D)$, i.e.\ morphism~$\lambda \colon C \to D$, with~$(d_i)_{i \in I} = \eta_D(\lambda)$.
  It follows from the naturality of~$\eta$ that the diagram
  \[
    \begin{tikzcd}[row sep = large]
        \Ccat(C,C)
        \arrow[equal]{r}
        \arrow{d}[left]{\lambda_*}
      & h^C(C)
        \arrow{r}[above]{\eta_C}
        \arrow{d}[left]{h^C(\lambda)}
      & F(C)
        \arrow{d}[right]{F(\lambda)}
        \arrow[equal]{r}
      & \prod_{i \in I} \Ccat(X_i, C)
        \arrow{d}[right]{\prod_{i \in I} \lambda_*}
      \\
        \Ccat(C,D)
        \arrow[equal]{r}
      & h^C(D)
        \arrow{r}[below]{\eta_D}
      & F(D)
        \arrow[equal]{r}
      & \prod_{i \in I} \Ccat(X_i, D)
    \end{tikzcd}
  \]
  commutes.
  It therefore follows for the element~$\id_C \in \Ccat(\Ccat, \Ccat) = h^C(C)$ that
  \begin{align*}
        (d_i)_{i \in I}
     =  \eta_D( \lambda )
     =  \eta_D( \lambda_*( \id_C) )
    &=  \eta_D( h^C(\lambda)( \id_C ) )  \\
    &=  F(\lambda)( \eta_C( \id_C) ) )
     =  F(\lambda)( (c_i)_{i \in I} )
     =  (\lambda \circ c_i)_{i \in I} \,,
  \end{align*}
  and hence that~$\lambda \circ c_i = d_i$ for every~$i \in I$.
  This shows the existence of the required morphism~$\lambda \colon C \to D$.
  By reading the above argumentation from the bottom to the top we also find that the morphism~$\lambda$ is unique.
  
  This shows that the object~$C$ together with the morphisms~$c_i \colon X_i \to C$ is a coproduct of the family~$(X_i)_{i \in I}$.
\end{proof}


\begin{remark*}
  The above proof actually shows that an object~$C \in \Ob(\Ccat)$ is a coproduct of the family~$(X_i)_{i \in I}$, with respect to some suitable morphisms~$c_i \colon X_i \to C$, if and only if it represents the functor~$\prod_{i \in I} \Ccat(X_i, -) \colon \Ccat \to \Set$.
  This statement is stronger than the formulation in \cref{existence of coproducts}.
  
  One can also show a slightly stronger version of this:
  It follows for every object~$C \in \Ob(\Ccat)$ from Yoneda’s~lemma that the map
  \begin{align*}
              \{ \text{natural transformations~$\eta \colon h^C \to F$} \}
    &\to      F(C) \,,
    \\
              \eta
    &\mapsto  \eta_C(\id_C)
  \end{align*}
  is a bijection.
  An element of the right hand side is an element of~$F(C)$, i.e.\ a family~$(c_i)_{i \in I}$ of morphisms~$c_i \colon X_i \to C$.
  It then holds that a natural transformation~$\eta \colon h^C \to F$ is an isomorphism if and only if the corresponding family~$(c_i)_{i \in I}$ makes the pair~$(C, (c_i)_{i \in I})$ into a coproduct of the family~$(X_i)_{i \in I}$.
  
  Indeed, that the natural transformation~$\eta$ is a natural isomorphism means that at every object~$D \in \Ob(\Ccat)$ the map
  \[
            \eta_D
    \colon  \Ccat(C,D)
    =       h^C(D)
    \to     F(D)
  \]
  is a bijection.
  The component~$\eta_D$ can be expressed via the element~$(c_i)_{i \in I} \in F(C)$ as
  \[
      \eta_D(\lambda)
    = F(\lambda)( (c_i)_{i \in I} )
    = (\lambda \circ c_i)_{i \in I} \,.
  \]
  The bijectivity of~$\eta_D$ therefore means that for every~$(d_i)_{i \in I} \in F(D) = \prod_{i \in I} \Ccat(X_i, D)$ there exists a unique element~$\lambda \in \Ccat(C, D)$ with~$(\lambda \circ c_i)_{i \in I} = (d_i)_{i \in I}$.
  In other words, there exists for every object~$D \in \Ob(\Ccat)$ and every family~$(d_i)_{i \in I}$ of morphisms~$d_i \colon X_i \to D$ a unique morphism~$\lambda \colon C \to D$ with~$d_i = \lambda \circ c_i$ for every~$i \in I$.
  But this is precisely what it means for the pair~$(C, (c_i)_{i \in I})$ to be a coproduct of the family~$(X_i)_{i \in I}$.
  
  This shows that for every object~$C \in \Ob(\Ccat)$, a family~$(c_i)_{i \in I}$ of morphisms~$c_i \colon X_i \to C$ that makes~$(C, (c_i)_{i \in I})$ into a coproduct of the family~$(X_i)_{i \in I}$ is \enquote{the same} as a natural isomorphism~$\eta \colon h^C \to F$ (via Yoneda’s~lemma).
  It follows in particular that~$C$ is a coproduct of the family~$(X_i)_{i \in I}$, with respect to a suitable choice of morphisms~$c_i \colon X_i \to C$, if and only if there exist a natural isomorphism~$h^C \to F$.
  (This is what was shown in the above proof.)
\end{remark*}





\section{Additive Categories}


\begin{definition}
  A \emph{preadditive~category}\index{pre-additive category}\index{category!pre-additive} is a category~$\Acat$ together with the structure of an abelian group on~$\Acat(X,Y)$ for all~$X, Y \in \Ob(\Acat)$ such that the composition in~$\Ccat$ is~{\Zbilin}, i.e.\ such that
  \[
    k \circ (g + h) = k \circ g + k \circ h
    \quad\text{and}\quad
    (g + h) \circ f = g \circ f + h \circ f
  \]
  for all morphisms~$f \colon W \to X$,~$g, h \colon X \to Y$ and~$k \colon Y \to Z$ in~$\Acat$.
\end{definition}


\begin{remark}
  \leavevmode
  \begin{enumerate}
    \item
      Preddative categories are also known as~\dash{$\Ab$}{categories} (where~$\Ab$ denotes the category of abelian groups).
      (In the language of enriched category theory, an~\dash{$\Ab$}{category} is precisely a category that is enriched over the monoidal category~$(\Ab,{\tensor})$.)
    \item
      If~$\kf$ is a commutative ring then one can similarly define the notation of a {\preklin} category (also known as~\dash{$\Modl{k}$}{category})~$\Ccat$.
      Every~$\Ccat(X,Y)$ is then endowed with the structure of a~{\module{$\kf$}} and the composition is~{\kbilin}.
  \end{enumerate}
\end{remark}


\begin{example}
  \leavevmode
  \begin{enumerate}
    \item
      The category~$\Ab = \Modl{\Integer}$ is preadditive.
    \item
      If~$A$ is a~{\kalg} then the categories~$\Modl{A}$ and~$\Modr{A}$ are {\preklin}.
    \item
      If~$R$ is a ring then we may think about~$R$ as a preadditive category~$\Rcat$ consisting of a single object~$\Ob(\Rcat) = \{ \ast \}$ with~$\Rcat(\ast,\ast) = R$.
      The composition in~$\Rcat$ is given by the multiplication of~$R$, i.e.\ by~$g \circ f = gf$ for all~$f, g \in R$, and the addition of morphisms is the addition in~$R$.
      
      One can similarly regard every~{\kalg}~$A$ as a {\preklin} category~$\Acat$ which consists of a single object~$\Ob(\Acat) = \{\ast\}$ with~$\Acat(\ast,\ast) = A$.
    \item
      Let~$\Ccat$ be any category and let~$\Acat$ be a preadditive category.
      Then the functor category~$\Fun(\Ccat, \Acat)$ is again preadditive:
      For any two natural transformations~$\eta, \zeta \colon F \to G$ between functors~$F, G \in \Ob(\Fun(\Ccat, \Acat))$, their sum~$\eta + \zeta$ is at an object~$X \in \Ob(\Ccat)$ given by
      \[
          (\eta + \zeta)_X
        = \eta_X + \zeta_X \,.
      \]
      This defines again a natural transformation~$\eta + \zeta \colon F \to G$.
      Indeed, for every morphism~$f \colon X \to X'$ in~$\Ccat$ the square
      \[
        \begin{tikzcd}[sep = large]
            F(X)
            \arrow{r}[above]{F(f)}
            \arrow{d}[left]{\eta_X + \zeta_X}
          & F(X')
            \arrow{d}[right]{\eta_{X'} + \zeta_{X'}}
          \\
            G(X)
            \arrow{r}[above]{G(f)}
          & G(X')
        \end{tikzcd}
      \]
      commutes because
      \begin{align*}
            (\eta_{X'} \circ \zeta_{X'}) \circ F(f)
        &=  \eta_{X'} \circ F(f) + \zeta_{X'} \circ F(f)  \\
        &=  G(f) \circ \eta_X + G(f) \circ \zeta_X
         =  G(f) \circ (\eta_X + \zeta_X) \,.
      \end{align*}
      
      We find similarly that for every category~$\Ccat$ and every {\preklin} category~$\Acat$ the functor category~$\Fun(\Ccat, \Acat)$ is again~{\preklin}. 
  \end{enumerate}
\end{example}


\begin{remark*}
  \leavevmode
  \begin{enumerate}
    \item
      A preadditive category is the same as a pre\nobreakdash-$\Integer$\nobreakdash-linear category.
    \item
      If~$\Acat$ is a preadditive category, then the opposite category~$\Acat^\op$ is again preadditive with the same addition of morphisms.
  \end{enumerate}
\end{remark*}


\begin{definition}
  Let~$F \colon \Acat \to \Bcat$ be a functor between categories~$\Acat$ and~$\Bcat$.
  \begin{enumerate}
    \item
      If~$\Acat$ and~$\Bcat$ are preaddive categories then the functor~$F$ is \emph{additive}\index{additive!functor}\index{functor!additive} if
      \[
          F(f + g)
        = F(f) + F(g)
      \]
      for all morphisms~$f, g \colon X \to X'$ in~$\Acat$, i.e.\ if the map
      \[
                    \Acat(X, Y)
        \xlongto{F} \Bcat(F(X), F(Y))
      \]
      is a group homorphism for all~$X, Y \in \Ob(\Acat)$.
    \item
      If~$\Acat$ and~$\Bcat$ are {\preklin} categories then the functor~$F$ is~\emph{{\klin}}\index{k-linear@$\kf$-linear!functor}\index{functor!k-linear@$\kf$-linear} if
      \[
        F(f + g) = F(f) + F(g)
        \quad\text{and}\quad
        F(\lambda f) = \lambda F(f)
      \]
      for all morphisms~$f, g \colon X \to X'$ in~$\Acat$ and scalars~$\lambda \in \kf$, i.e.\ if the map
      \[
                    \Acat(X, Y)
        \xlongto{F} \Bcat(F(X), F(Y))
      \]
      is~{\klin} for all~$X, Y \in \Ob(\Acat)$.%
      \footnote{The notion of a~{\klin} functor was not introduced in the lecture.}
  \end{enumerate}
\end{definition}


\begin{lemma}
  \label{inital terminal zero}
  Let~$\Acat$ be a preadditive category.
  \begin{enumerate}
    \item
      For any object~$X \in \Ob(\Acat)$ the following conditions are equivalent:
      \begin{enumerate}
        \item
          The object~$X$ is unital in~$\Acat$.
        \item
          The object~$X$ is terminal in~$\Acat$.
        \item
          The object~$X$ is a zero object for~$\Acat$.
        \item
          It holds that~$\id_X = 0_{\Acat(X,X)}$.
        \item
          The abelian group~$\Acat(X,X)$ consists of only a single element.
      \end{enumerate}
    \item
      Suppose that the category~$\Acat~$ has a zero object.
      Then it holds for any two objects~$X, Y \in \Ob(\Acat)$ that~$0_{X,Y} = 0_{\Acat(X,Y)}$.
  \end{enumerate}
\end{lemma}


\begin{proof}
  This is Exercise~3 of the fifth exercise sheet.
\end{proof}





\lecturend{9}


\begin{definition}
  Let~$\Acat$ be a preadditive category and let~$X_1, \dotsc, X_n \in \Ob(\Acat)$ be objects, where~$n \in \Integer_{\geq 0}$.
  A \emph{biproduct}\index{biproduct} of~$X_1, \dotsc, X_n$ is a triple~$(X, (p_1, \dotsc, p_n), (c_1, \dotsc, c_n))$ consisting of an object~$X \in \Ob(\Acat)$ together with morphisms~$p_i \colon X \to X_i$ and morphisms~$c_i \colon X_i \to X$ in~$\Acat$,
  \begin{equation}
    \label{no abuse of notation}
      p_j c_i
    = \begin{cases}
        \id_{X_i}     & \text{if~$i = j$}     \,, \\
        0_{X_i, X_j}  & \text{if~$i \neq j$}  \,,
      \end{cases}
  \end{equation}
  for all~$i,j = 1, \dots, n$, and
  \[
      \sum_{i=1}^n c_i p_i
    = {\id_X} \,.
  \]
% Add notation.
\end{definition}


\begin{remark*}
  In the lecture, the formula~\ref{no abuse of notation} was instead written as
  \[
      p_j c_i
    = \delta_{ij} \id_{X_i} \,.
  \]
  This is an abuse of notation:
  For~$j \neq i$ the composition~$p_j c_i$ is a morphism~$X_i \to X_j$, whereas~$\delta_{ij} \id_{X_i} = 0 \cdot \id_{X_i} = 0_{X_i, X_i}$ is the zero morphism~$X_i \to X_i$.
  The author tries to avoid this abuse of notation, but will still sometimes write~$p_j c_i = \delta_{ij}$ as an abbreviation for~\eqref{no abuse of notation}.
\end{remark*}


\begin{remark*}
%   TODO: Add this footnote.
%   \footnote{This was an offical, but unnumbered remark in the lecture.}
  For a preadditive category~$\Acat$, a biproduct of an empty family of objects in~$\Acat$ is the same as a zero object of~$\Acat$.
\end{remark*}


\begin{lemma}
  \label{product coproduct biproduct}
  Let~$\Acat$ be a preadditive category, let~$X_1, \dotsc, X_n \in \Acat$ where~$n \in \Integer_{\geq 0}$.
  \begin{enumerate}
    \item
      If~$(X, (p_1, \dotsc, p_n), (c_1, \dotsc, c_n))$ is a biproduct of~$X_1, \dotsc, X_n$ then~$(X, (p_1, \dotsc, p_n))$ is a product of~$X_1, \dotsc, X_n$ and~$(X, (c_1, \dotsc, c_n))$ is a coproduct of~$X_1, \dotsc, X_n$.
    \item
      \label{products into biproducts}
      Suppose that~$(X, (p_1, \dotsc, p_n))$ is a product of~$X_1, \dotsc, X_n$.
      Then there exist for every~$i = 1, \dotsc, n$ a unique morphism~$c_i \colon X_i \to X$ with~$p_j c_i = \delta_{ij} \id_{X_i}$ for every~$j = 1, \dotsc, n$.
      The triple~$(X, (p_1, \dotsc, p_n), (c_1, \dotsc, c_n))$ is then a biproduct of~$X_1, \dotsc, X_n$.
    \item
      Dually, suppose that~$(X, (c_1, \dotsc, c_n))$ is a coproduct of~$X_1, \dotsc, X_n$.
      Then there exist for every~$i = 1, \dotsc, n$ a unique morphism~$p_i \colon X \to X_i$ with~$p_i c_j = \delta_{ij} \id_{X_i}$ for every~$j = 1, \dotsc, n$.
      The triple~$(X, (p_1, \dotsc, p_n), (c_1, \dotsc, c_n))$ is then a biproduct of~$X_1, \dotsc, X_n$.
  \end{enumerate}
\end{lemma}


\begin{proof}
  For security reasons we consider the case~$n = 0$ separately:
  The product over the empty family is a final object of~$\Acat$, the coproduct over the empty family is an initial object of~$\Acat$, and the biproduct over the empty family is a zero object of~$\Acat$.
  The statements therefore follow for~$n = 0$ from \cref{inital terminal zero}.
  In the following we consider the case~$n \geq 1$.
  \begin{enumerate}
    \item
      It sufficies by duality to show that~$(X, (c_i)_i)$ is a coproduct for~$X_1, \dotsc, X_n$.
      Let~$(D, (d_i)_i)$ be another pair consisting of an object~$D \in \Ob(\Acat)$ and morphisms~$d_i \colon X_i \to D$.
      We need to show that there exists a unique morphism~$\mu \colon X \to D$ which makes the triangle
      \[
        \begin{tikzcd}[sep = large]
            X_i
            \arrow{r}[above]{c_i}
            \arrow{dr}[below left]{d_i}
          & X
            \arrow[dashed]{d}[right]{\mu}
          \\
            {}
          & D
        \end{tikzcd}
      \]
      commute for every~$i = 1, \dotsc, n$. 
      If such a morphism~$\mu$ exists then
      \[
          \mu
        = \mu \id_X
        = \mu \sum_{i=1}^n c_i p_i
        = \sum_{i=1}^n \mu c_i p_i
        = \sum_{i=1}^n d_i p_i \,,
      \]
      which shows that~$\mu$ is unique.
      If we define on the other hand~$\mu \defined \sum_{i=1}^n d_i p_i$ then
      \[
          \mu c_i
        = \sum_{j=1}^n d_j \underbrace{p_j c_i}_{= \delta_{ij}}
        = d_i \,,
      \]
      which shows the existence of~$\mu$.
    \item
      By the universal property of the product there exists for every~$i = 1, \dotsc, n$ a unique morphism~$c_i \colon X_i \to X$ which makes for all~$j \neq i$ the triangles
      \[
        \begin{tikzcd}[sep = large]
            X_i
            \arrow{dr}[above right]{\id_{X_i}}
            \arrow[dashed]{d}[left]{c_i}
          & {}
          \\
            X
            \arrow{r}[below]{p_i}
          & X_i
        \end{tikzcd}
        \qquad\text{and}\qquad
        \begin{tikzcd}[sep = large]
            X_i
            \arrow{dr}[above right]{0}
            \arrow[dashed]{d}[left]{c_i}
          & {}
          \\
            X
            \arrow{r}[below]{p_j}
          & X_i
        \end{tikzcd}
      \]
      commute.
      This means that~$p_j c_i = \delta_{ij}$ for all~$i, j = 1, \dotsc, n$.
      
      We now show that~$\sum_{i=1}^n c_i p_i = \id_X$.
      Indeed, we find for every~$j = 1, \dotsc, n$ that
      \[
          p_j \circ \sum_{i=1}^n c_i p_i
        = \sum_{i=1}^n \underbrace{ p_j c_i }_{= \delta_{ij}} p_i
        = p_j \,.
      \]
      That shows that for every~$j = 1, \dotsc, n$ the triangle
      \[
        \begin{tikzcd}
            X
            \arrow{dr}[below left]{p_j}
            \arrow[dashed]{rr}[above]{\sum_{i=1}^n c_i p_i}
          & {}
          & X
            \arrow{dl}[below right]{p_j}
          \\
            {}
          & X_j
          &
        \end{tikzcd}
      \]
      commutes.
      But it follows from the uniqueness of products (up to isomorphism) that there exist a \emph{unique} morphism~$X \to X$ which makes this triangle commute.
      The identity~$\id_X \colon X \to X$ also makes the above triangle commute, and so it follows that~$\sum_{i=1}^n c_i p_i = \id_X$.
    \item
      This can be shown dually to part~\ref*{products into biproducts}.
    \qedhere
  \end{enumerate}
\end{proof}


\begin{remark}
  It follows from \cref{product coproduct biproduct} that for a preadditive category~$\Acat$ the following are equivalent:
  \begin{enumerate}
    \item
      $\Acat$ has finite products.
    \item
      $\Acat$ has finite coproducts.
    \item
      $\Acat$ has finite biproducts.
  \end{enumerate}
\end{remark}


\begin{definition}
  A preadditve (or {\preklin}) category~$\Acat$ is \emph{additive}\index{additive!category}\index{category!additive} (resp.~{\klin}\index{k-linear@$\kf$-linear!category}\index{category!k-linear@$\kf$-linear}) if it has finite biproducts (and thus equivalently finite products, and equivalently finite coproducts).
\end{definition}


\begin{remark*}
% TODO: Add this footnote.
% \footnote{This remark was made in the lecture, but unnumbered.}
  Additive (and~{\klin}) categories have zero objects, as these are the biproducts of empty family of objects.
\end{remark*}


\begin{remark*}
  In a preadditive catgory~$\Acat$ one can express morphisms between biproducts as matrices:
  Let~$X_1, \dotsc, X_n$ and~$Y_1, \dotsc, Y_m$ be two families of objects in~$\Acat$ whose biproducts~$X_1 \oplus \dotsb \oplus X_n$ and~$Y_1 \oplus \dotsb \oplus Y_m$ exist, and denote the associated morphisms by
  \begin{align*}
    c_i \colon X_i \to X_1 \oplus \dotsb \oplus X_n
    \quad&\text{and}\quad
    p_i \colon X_1 \oplus \dotsb \oplus X_n \to X_i \,,
  \shortintertext{and}
    d_i \colon Y_i \to Y_1 \oplus \dotsb \oplus Y_m
    \quad&\text{and}\quad
    q_i \colon Y_1 \oplus \dotsb \oplus Y_m \to Y_i \,.
  \end{align*}
  
  Suppose first that we are given a morphism
  \[
            f
    \colon  X_1 \oplus \dotsb \oplus X_n
    \to     Y_1 \oplus \dotsb \oplus Y_m
  \]
  in~$\Acat$.
  It then follows from the calculation
  \begin{align*}
        f
    &=  \id_{Y_1 \oplus \dotsb \oplus Y_m} \circ f \circ \id_{X_1 \oplus \dotsb \oplus X_n} \\
    &=  \left( \sum_{i=1}^n d_i q_i \right) \circ f \circ \left( \sum_{j=1}^m c_j p_j \right)
     =  \sum_{i=1}^n \sum_{j=1}^m d_i (q_i \circ f \circ c_j) p_j \,.
  \end{align*}
  that the morphism~$f$ is unique determined by the compositions~$q_i \circ f \circ c_j$.
  We will refer to the composition
  \[
    [f]_{ij} \defined q_i \circ f \circ c_j
  \]
  as the~\dash{$(i,j)$}{th} component of~$f$.
  The above calculation shows that the morphism~$f$ can be retrieved from its components via the formula
  \[
    f = \sum_{i=1}^n \sum_{j=1}^m d_i [f]_{ij} p_j \,.
  \]
  To better visualize the relation between~$f$ and its components, we may arrange these components in the form of an~\dash{$(m \times n)$}{matrix}
  \[
    \begin{bmatrix}
      f_{11}  & \cdots  & f_{1n}  \\
      \vdots  & \ddots  & \vdots  \\
      f_{m1}  & \cdots  & f_{mn}
    \end{bmatrix} \,.
  \]
  We will refer to this matrix as~$[f]$.
  (Note that~$[f]_{ij}$ is hence the~\dash{$(i,j)$}{th} entry of the matrix~$[f]$.)
  
  Suppose on the other hand that we are given a collection of morphisms~$g_{ij} \colon X_j \to Y_i$ where~$i = 1, \dotsc, m$ and~$j = 1, \dotsc, n$.
  We can then define a morphism~$g \colon X \to Y$ via
  \[
              g
    \defined  \sum_{i=1}^m \sum_{j=1}^n d_i g_{ij} p_j \,.
  \]
  The components~$[g]_{ij}$ of the morphism~$g$ are then for all~$i = 1, \dotsc, m$ and~$j = 1, \dotsc, n$ given by
  \begin{align*}
        [g]_{ij}
     =  q_i \circ g \circ c_j
    &=  q_i \circ \left( \sum_{i'=1}^m \sum_{j'=1}^n d_{i'} g_{i'j'} p_{j'} \right) \circ c_j \\
    &=  \sum_{i=1}^m \sum_{j=1}^n
        \underbrace{q_i d_{i'}}_{= \delta_{i,i'}} g_{i'j'} \underbrace{p_{j'} c_j}_{= \delta_{j',j}}
     =  g_{ij} \,.
  \end{align*}
  The components~$[g]_{ij}$ of~$g$ are hence the morphisms~$g_{ij}$ that we started with.
  
  This shows overall that we have constructed a bijection
  \begin{align*}
      \Acat(X_1 \oplus \dotsb \oplus X_n, Y_1 \oplus \dotsb \oplus Y_m)
    &\longleftrightarrow
      \left\{
        \begin{bsmallmatrix}
          g_{11}  & \cdots  & g_{1n}  \\
          \vdots  & \ddots  & \vdots  \\
          g_{m1}  & \cdots  & g_{mn}
        \end{bsmallmatrix}
      \suchthat*
        g_{ij} \in \Acat(X_j, Y_i)
      \right\}  \,,
    \\
      f
    &\longmapsto
      [f] \,,
    \\
      \sum_{i=1}^m \sum_{j=1}^n q_i g_{ij} c_j
    =
      g
    &\longmapsfrom
      \begin{bsmallmatrix}
          g_{11}  & \cdots  & g_{1n}  \\
          \vdots  & \ddots  & \vdots  \\
          g_{1n}  & \cdots  & g_{mn}
        \end{bsmallmatrix}  \,.
  \end{align*}
  This way of representing morphisms between biproducts as matrices is compatible with both sums and composition of morphisms, and if~$\Acat$ is~{\preklin} then also with scalar multiplication of morphisms.
  \begin{itemize}
    \item
      Let~$f_1, f_2 \colon X_1 \oplus \dotsb \oplus X_n \to Y_1 \oplus \dotsb \oplus Y_m$ be two morphisms in~$\Acat$.
      Then the morphism~$f_1 + f_2$ has for all~$i = 1, \dotsc, m$ and~$j = 1, \dotsc, n$ the components
      \[
          [f_1 + f_2]_{ij}
        = q_i \circ (f_1 + f_2) \circ c_j
        = q_i \circ f_1 \circ c_j + q_i \circ f_2 \circ c_j
        = [f_1]_{ij} + [f_2]_{ij} \,.
      \]
      This shows that indeed
      \[
          [f_1 + f_2]
        = [f_1] + [f_2] \,.
      \]
    \item
      Let~$Z_1, \dotsc, Z_l$ be objects in~$\Acat$ whose biproduct~$Z_1 \oplus \dotsb \oplus Z_l$ exists in~$\Acat$, and let
      \[
        e_i \colon Z_i \to Z_1 \oplus \dotsb \oplus Z_l
      \quad\text{and}\quad
        r_i \colon Z_1 \oplus \dotsb \oplus Z_l \to Z_i \,,
      \]
      be the associated morphisms.
      It then holds for any two composable morphisms
      \[
          X_1 \oplus \dotsb \oplus X_n
        \xlongto{f}
          Y_1 \oplus \dotsb \oplus Y_m
        \xlongto{g}
          Z_1 \oplus \dotsb \oplus Z_l
      \]
      in~$\Acat$ that
      \[
          [g \circ f]
        = [g] \cdot [f]
      \]
      where the product on the right hand side is taken in the naive way.
      Indeed the composition~$g \circ f$ has for all~$i = 1, \dotsc, l$ and~$k = 1, \dotsc, n$ the components
      \begin{align*}
            [g \circ f]_{ik}
        &=  r_i \circ (g \circ f) \circ c_k
         =  r_i \circ g \circ \id_Y \circ f \circ c_k \\
        &=  r_i \circ g \circ \left( \sum_{j=1}^m d_j q_j \right) \circ f \circ c_k \\
        &=  \sum_{j=1}^m (r_i \circ g \circ d_j) \circ (q_j \circ f \circ c_k)
         =  \sum_{j=1}^m [g]_{ij} [f]_{jk} \,.
      \end{align*}
      The resulting term~$\sum_{j=1}^m [g]_{ij} [f]_{jk}$ is precisely the~\dash{$(i,k)$}{th} entry of the matrix product~$[g] \cdot [f]$.
    \item
      If~$\Acat$ also~{\preklin} then let~$f \colon X_1 \oplus \dotsb \oplus X_n \to Y_1 \oplus \dotsb \oplus Y_m$ be a morphism in~$\Acat$ and let~$\lambda \in \kf$ be a scalar.
      Then the morphism~$\lambda f$ has for all~$i = 1, \dotsc, m$ and~$j = 1, \dotsc, n$ the components
      \[
          [\lambda f]_{ij}
        = q_i \circ (\lambda f) \circ c_j
        = \lambda (q_i \circ f \circ c_j)
        = \lambda [f]_{ij} \,.
      \]
      This shows that indeed
      \[
          [\lambda f]
        = \lambda [f] \,.
      \]
  \end{itemize}
  
  In the following we will notationally often not distinguish between the morphism~$f \colon X_1 \oplus \dotsb \oplus X_n \to Y_1 \oplus \dotsb \oplus Y_m$ and its matrix representation.
  So instead of
  \[
      [f]
    = \begin{bmatrix}
        f_{11}  & \cdots  & f_{1n}  \\
        \vdots  & \ddots  & \vdots  \\
        f_{m1}  & \cdots  & f_{mn}
      \end{bmatrix}
  \]
  (where~$f_{ij} = [f]_{ij}$ is the~\dash{$(i,j)$}{th} component of~$f$) we will just write
  \[
      f
    = \begin{bmatrix}
        f_{11}  & \cdots  & f_{1n}  \\
        \vdots  & \ddots  & \vdots  \\
        f_{m1}  & \cdots  & f_{mn}
      \end{bmatrix} \,.
  \]
  If one of the morphisms~$f_{ij}$ is the identity~$\id_Z$ of some object~$Z$ (which is then necessarily given by~$Z = X_j = Y_i$) then we will often just write the corresponding matrix entry as~$1$ instead of~$\id_Z$.
  
  We finish this remark by pointing out that the morphisms~$c_i \colon X_i \to X_1 \oplus \dotsb \oplus X_n$ and~$p_i \colon X_1 \oplus \dotsb \oplus X_n \to X_i$ are by these notional conventions given by the matrices
  \[
      c_i
    = \begin{bsmallmatrix}
        {} \\0 \\ \vdots \\ 0 \\ 1 \\ 0 \\ \vdots \\ 0 \\ {}
      \end{bsmallmatrix}
    \qquad\text{and}\qquad
      p_i
    = \begin{bsmallmatrix}
        0 & \cdots & 0 & 1 & 0 & \cdots & 0
      \end{bsmallmatrix} \,.
  \]
  
  (This whole remark was not present in the lecture.
  The idea of representing a morphism between biproducts as a matrix was explained only for the special case~$n = m = 2$, since is used in the upcoming \cref{sum via category structure}.
  But the author found this ad\nobreakdash-hoc explanation a bit insufficient, and thus decided to add a more detailed explanation for these notes.)
\end{remark*}


\begin{remark}
  \label{sum via category structure}
  Let$~\Acat$ be an additive category.
  For any objects~$X \in \Ob(\Acat)$ we can define the \emph{diagonal \textup(morphism\textup)}\index{diagonal morphism}
  \[
            \diag_X
    \colon  X
    \to     X \oplus X
  \]
  by using the universal property of the product for~$X \oplus X$, as the unique morphism~$X \to X \oplus X$ which make the diagram
  \[
    \begin{tikzcd}[sep = large]
        {}
      & X
        \arrow[dashed]{d}[right]{\diag_X}
        \arrow{dl}[above left]{\id_X}
        \arrow{dr}[above right]{\id_X}
      & {}
      \\
        X
      & X \oplus X
        \arrow{l}[below]{p_1}
        \arrow{r}[below]{p_2}
      & X
    \end{tikzcd}
  \]
  commute.
  This means that
  \[
    p_1 \circ \diag_X = \id_X
    \quad\text{and}\quad
    p_2 \circ \diag_X = \id_X \,,
  \]
  so the morphism~$\diag_X$ can be written is matrix form as
  \[
      \diag_X
    = \begin{bmatrix}
        1 \\ 1
      \end{bmatrix} \,.
  \]
  We can dually define the \emph{codiagonal \textup(morphism\textup)}\index{codiagonal morphism}
  \[
            \codiag_X
    \colon  X \oplus X
    \to     X
  \]
  by using the universal property of the coproduct for~$X \oplus X$, as the unique morphism~$X \oplus X \to X$ which makes the diagram
  \[
    \begin{tikzcd}[sep = large]
        X
        \arrow{dr}[below left]{\id_X}
      & X \oplus X
        \arrow[dashed]{d}[right]{\codiag_X}
        \arrow{l}[above]{c_1}
        \arrow{r}[above]{c_2}
      & X
        \arrow{dl}[below right]{\id_X}
      \\
        {}
      & X
      & {}
    \end{tikzcd}
  \]
  commute.
  This means that
  \[
    \codiag_X \circ c_1 = \id_X
    \quad\text{and}\quad
    \codiag_X \circ c_2 = \id_X \,,
  \]
  so the morphism~$\codiag_X$ can be written in matrix form as
  \[
      \codiag_X
    = \begin{bmatrix}
        1 & 1
      \end{bmatrix} \,.
  \]
  
  (Note that for~$\Acat = \Modl{A}$, where~$A$ is a~{\kalg}, the diagonal~$\diag_X$ is the usual diagonal map~$\diag_X(x) = (x,x)$, and the codiagonal~$\codiag_X$ is the addition~$\codiag_X(x_1, x_2) = x_1 + x_2$.)
  
  We can now describe the sum~$f + g$ of two parallel morphisms~$f, g \colon X \to Y$ in~$\Acat$ as the compositions
  \begin{equation}
    \label{composition for sum}
      X=
    \xlongto{\diag_X}
      X \oplus X
    \xlongto{\begin{bsmallmatrix} f & 0 \\ 0 & g \end{bsmallmatrix}}
      Y \oplus Y
    \xlongto{\codiag_Y}
      Y \,.
  \end{equation}
  Indeed, we find by matrix multiplication that
  \[
      \codiag_Y
      \circ
      \begin{bmatrix}
        f & 0 \\
        0 & g
      \end{bmatrix}
      \circ
      \diag_X
    = \begin{bmatrix}
        1 & 1
      \end{bmatrix}
      \begin{bmatrix}
        f & 0 \\
        0 & g
      \end{bmatrix}
      \begin{bmatrix}
        1 \\
        1
      \end{bmatrix}
    = f + g \,.
  \]
  By using that
  \[
    \begin{bmatrix}
      1 & 1
    \end{bmatrix}
    \begin{bmatrix}
      f & 0 \\
      0 & g
    \end{bmatrix}
    =
    \begin{bmatrix}
      f & g
    \end{bmatrix}
    \quad\text{and}\quad
    \begin{bmatrix}
      f & 0 \\
      0 & g
    \end{bmatrix}
    \begin{bmatrix}
      1 \\
      1
    \end{bmatrix}
    =
    \begin{bmatrix}
      f \\
      g
    \end{bmatrix}
  \]
  we can also rewrite the composition~\eqref{composition for sum} as
  \[
      X
    \xlongto{\diag_X}
      X \oplus X
    \xlongto{\begin{bsmallmatrix} f & g \end{bsmallmatrix}}
      Y
    \qquad\text{or}\qquad
      X
    \xlongto{\begin{bsmallmatrix} f \\ g \end{bsmallmatrix}}
      Y \oplus Y
    \xlongto{\codiag_Y}
      Y \,.
  \]
  
  This shows that the addition of~$\Acat$ can be retrieved from the categorical structure of~$\Acat$.
  It follows that an arbitrary category~$\Acat$ can be made into an additive category in at most one way.
  We can therefore regard \enquote{being additive} as a property of a category.
\end{remark}


\begin{definition}
  Let~$F \colon \Ccat \to \Dcat$ be a functor between arbitrary categories~$\Ccat$ and~$\Dcat$.
  \begin{enumerate}
    \item
      The functor~$F$ \emph{respect \textup(finite\textup) products}\index{functor!respects!products} if it holds for every (finite) family~$(X_i)_{i \in I}$ of objects~$X_i \in \Ob(\Ccat)$ and every product~$(P, (p_i)_{i \in I})$ of this family that the pair~$(F(P), (F(p_i))_{i \in I})$ is a product of the family~$(F(X_i))_{i \in I}$.
    \item
      The functor~$F$ \emph{respect \textup(finite\textup) coproducts}\index{functor!respects!coproducts} if it holds for every (finite) family~$(X_i)_{i \in I}$ of objects~$X_i \in \Ob(\Ccat)$ and every coproduct~$(C, (c_i)_{i \in I})$ of this family that the pair~$(F(C), (F(c_i))_{i \in I})$ is a coproduct of the family~$(F(X_i))_{i \in I}$.
  \end{enumerate}
  Suppose now that~$F \colon \Acat \to \Bcat$ is a functor between preadditive categories~$\Acat$ and~$\Bcat$.
  \begin{enumerate}[resume]
    \item
      The functor~$F$ \emph{respects biproducts}\index{functor!respects!biproducts} if it holds for all objects~$X_1, \dotsc, X_n \in \Ob(\Acat)$ (where~$n \geq 0$) and every biproduct~$(X, (p_1, \dotsc, p_n), (c_1, \dotsc, c_n))$ of these objects that the triple~$(F(X), (F(p_1), \dotsc, F(p_n)), (F(c_1), \dotsc, F(c_n)))$ is a biproduct of the objects~$F(X_1), \dotsc, F(X_n)$.
  \end{enumerate}
\end{definition}


\begin{theorem}
  Let~$F \colon \Acat \to \Bcat$ be a functor between preadditive categories~$\Acat$ and~$\Bcat$.
  \begin{enumerate}
    \item
      \label{additive preserves biproducts}
      If the functor~$F$ is additive then it respects biproducts (and hence also finite products and finite coproducts).
    \item
      If the categories~$\Acat$ and~$\Bcat$ are already additive, then the following conditions on~$F$ are equivalent:
      \begin{enumerate}
        \item
          \label{is additive}
          $F$ is additive.
        \item
          \label{respects biproducts}
          $F$ respects biproducts.
        \item
          \label{respects finite products}
          $F$ respects finite products.
        \item
          \label{respects finite coproducts}
          $F$ respects finite coproducts.
      \end{enumerate}
  \end{enumerate}
\end{theorem}


\begin{proof}
  \leavevmode
  \begin{enumerate}
    \item
      Let~$n \geq 0$, let~$X_1, \dotsc, X_n \in \Ob(\Acat)$ and let~$(X, (p_i)_i, (c_i)_i)$ be a biproduct of~$X_1, \dotsc, X_n$.
      
      Once again we consider the case~$n = 0$ separately:
      The biproduct~$X$ is then a zero object of~$\Acat$.
      It follows that $\id_X = 0_{X,X}$ by \cref{inital terminal zero}, and hence
      \[
          \id_{F(X)}
        = F( \id_X )
        = F( 0_{X,X} )
        = 0_{F(X), F(X)} \,,
      \]
      by the additivity of~$F$.
      This shows that~$F(X)$ is a zero object for~$\Bcat$.
      
      Let now~$n \geq 1$.
      We then calculate that
      \begin{gather*}
          F(p_j)  F(c_i)
        = F(p_j c_i)
        = F
          \left(
              \begin{cases}
                \id_{X_i}     & \text{if~$i = j$} \\
                0_{X_i, X_j}  & \text{if~$i \neq j$}
              \end{cases}
          \right)
        = \begin{cases}
            \id_{F(X_i)}        & \text{if~$i = j$}     \,, \\
            0_{F(X_i), F(X_j)}  & \text{if~$i \neq j$}  \,,
          \end{cases}
      \intertext{and}
          \sum_{i=1}^n F(c_i) F(p_i)
        = \sum_{i=1}^n F(c_i p_i)
        = F\left( \sum_{i=1}^n c_i p_i \right)
        = F( \id_X )
        = \id_{F(X)} \,.
      \end{gather*}
    \item
      \begin{description}
        \item[\ref*{is additive}~$\implies$~\ref*{respects biproducts}]
          This has been shown in part~\ref*{additive preserves biproducts}.
          
        \item[\ref*{respects biproducts}~$\implies$~\ref*{respects finite products}]
          Let~$(X, (p_i)_i)$ be a product of the objects~$X_1, \dotsc, X_n$.
          It follows from \cref{product coproduct biproduct} unique morphism~$c_i \colon X_i \to X$ such that the triple~$(X, (p_i)_i, (c_i)_i)$ is a biproduct for~$X_1, \dotsc, X_n$.
          It follows that the triple~$(F(X), (F(p_i))_i, (F(c_i))_i)$ is a biproduct for the objects~$F(X_1), \dotsc, F(X_n)$ because the functor~$F$ preserves biproducts.
          This entails that the tuple~$(F(X), (F(p_i))_i)$ is a product of these objects.
        
        \item[\ref*{respects finite products}~$\implies$~\ref*{respects biproducts}]
          It follows from~$F$ respecting products that~$F$ respects terminal objects, because a terminal object is the same as an empty product.
          Hence~$F(0) = 0$.
          It follows that~$F(0_{X,Y}) = 0_{F(X), F(Y)}$ for any two objects~$X, Y \in \Ob(\Acat)$.
          Indeed, by applying the fuctor~$F$ to the commutative triangle
          \[
            \begin{tikzcd}[column sep = small]
                X
                \arrow{rr}[above]{0_{X,Y}}
                \arrow{dr}
              & {}
              & Y
              \\
                {}
              & 0
                \arrow{ur}
              & {}
            \end{tikzcd}
          \]
          we get the following commutative triangle:
          \[
            \begin{tikzcd}[column sep = small]
                F(X)
                \arrow{rr}[above]{F(0_{X,Y})}
                \arrow{dr}
              & {}
              & F(Y)
              \\
                {}
              & 0
                \arrow{ur}
              & {}
            \end{tikzcd}
          \]
          The commutativity of this triangle shows that the morphism~$F(0_{X,Y})$ factors through the zero object~$0$, which is precisly what is means for~$F(0_{X,Y})$ to be the zero morphism.
          
          Let now~$(X, (p_i)_i, (c_i)_i)$ be a biproduct of some objects~$X_1, \dotsc, X_n \in \Ob(\Acat)$.
          Then~$(X, (p_i)_i)$ is a product of~$X_1, \dotsc, X_n$, and it follows that~$(F(X), (F(p_i))_i)$ is a product of~$F(X_1), \dotsc, F(X_n)$ because~$F$ preserves finite product.
          We find for the morphisms~$F(c_i) \colon F(X_i) \to F(X)$ that
          \[
              F(p_i) \circ F(c_i)
            = F(p_i \circ c_i)
            = F( \id_{X_i} )
            = \id_{F(X_i)}
          \]
          and we also find for~$j \neq i$ that
          \[
              F(p_j) \circ F(c_i)
            = F(p_j \circ c_i)
            = F(0_{X_i, X_j})
            = 0_{F(X_i), F(X_j)} \,.
          \]
          It follows from \cref{product coproduct biproduct} that the triple~$(F(X), (F(p_i))_i, (F(c_i))_i)$ is a biproduct of~$F(X_1), \dotsc, F(X_n)$.
          
        \item[\ref*{respects biproducts}~$\iff$~\ref*{respects finite coproducts}]
          This can be shown dually to the equivalence of~\ref*{respects biproducts} and~\ref*{respects finite products}.
        
        \item[\ref*{respects biproducts}~$\implies$~\ref*{is additive}]
          We find as in the implication \ref*{respects finite products}~$\implies$~\ref*{respects biproducts} that~$F(0) = 0$ and that consequently~$F(0_{X,Y}) =  0_{F(X), F(Y)}$ for all~$X, Y \in \Ob(\Acat)$.
          It also follows from the already proven implications~\ref*{respects biproducts}~$\implies$~\ref*{respects finite products} and \ref*{respects biproducts}~$\implies$~\ref*{respects finite coproducts} that~$F$ respects products and coproducts.
          We hence find that
          \[
              F(\diag_X)
            = \diag_{F(X)}
            \quad\text{and}\quad
              F(\codiag_Y)
            = \codiag_{F(Y)}
          \]
          for all~$X, Y \in \Ob(\Acat)$.
          
          Let~$f, g \colon X \to Y$ be two parallel morphisms in~$\Acat$.
          We can then describe their sum~$f+g$ as the composition
          \begin{equation}
            \label{sum as composition}
              f+g
            \colon
              X
            \xlongto{\diag_X}
              X \oplus X
            \xlongto{\begin{bsmallmatrix} f & 0 \\ 0 & g \end{bsmallmatrix}}
              Y \oplus Y
            \xlongto{\codiag_Y}
              Y \,.
          \end{equation}
          It follows from~$F$ preserving coproducts and products, as well as identities and zero morphisms, that
          \[
            F
            \left(
              \begin{bmatrix}
                f & 0 \\
                0 & g
              \end{bmatrix}
            \right)
            =
            \begin{bmatrix}
              F(f)  & 0     \\
              0     & F(g)
            \end{bmatrix} \,.
          \]
          By using that $F(\diag_X) = \diag_{F(X)}$ and~$F(\codiag_Y) = \codiag_{F(Y)}$ we altogether find that applying the functor~$F$ to the compositon~$\eqref{sum as composition}$ exhibits the morphism~$F(f+g)$ as the composition
          \[
              F(f+g)
            \colon
              F(X)
            \xlongto{\diag_{F(X)}}
              F(X) \oplus F(X)
            \xlongto{\begin{bsmallmatrix} F(f) & 0 \\ 0 & F(g) \end{bsmallmatrix}}
              F(Y) \oplus F(Y)
            \xlongto{\codiag_{F(Y)}}
              Y \,.
          \]
          This composition is precisely~$F(f) + F(g)$, so~$F(f + g) = F(f) + F(g)$.
        \qedhere
      \end{description}
  \end{enumerate}
\end{proof}





\section{Kernels And Cokernels}


\begin{definition}
  Let~$\Ccat$ be a category that has a zero object, or that is preaddive.
  Let~$f \colon X \to Y$ be a morphism in~$\Ccat$.
  \begin{enumerate}
      \item
        A \emph{kernel}\index{kernel} of~$f$ is a pair~$(K,k)$ consisting of an object~$K \in \Ob(\Ccat)$ and a morphism~$k \colon K \to X$ with~$f \circ k = 0$, such that for every morphism~$\ell \colon L \to X$ in~$\Ccat$ with~$f \circ \ell = 0$ there exist a unique morphism~$\lambda \colon L \to K$ which makes the following diagram commute:
        \[
          \begin{tikzcd}[row sep = small, column sep = large]
              K
              \arrow{dr}[above right]{k}
              \arrow[bend left]{drr}[above right]{0}
            & {}
            & {}
            \\
              {}
            & X
              \arrow{r}[above, near start]{f}
            & Y
            \\
              L
              \arrow{ur}[below right]{\ell}
              \arrow[bend right]{urr}[below right]{0}
              \arrow[dashed]{uu}[left]{\lambda}
            & {}
            & {}
          \end{tikzcd}
        \]
      \item
        A \emph{cokernel}\index{cokernel} of~$f$ is a pair~$(C,c)$ consisting of an object~$C \in \Ob(\Ccat)$ and a morphism~$c \colon Y \to C$ with~$c \circ f = 0$, such that for every morphism~$d \colon Y \to D$ in~$\Ccat$ with~$d \circ f = 0$ there exist a unique morphism~$\mu \colon C \to D$ which makes the following diagram commute:
        \[
          \begin{tikzcd}[row sep = small, column sep = large]
              {}
            & {}
            & C
              \arrow[dashed]{dd}[right]{\mu}
            \\
              X
              \arrow{r}[above]{f}
              \arrow[bend left]{urr}[above left]{0}
              \arrow[bend right]{drr}[below left]{0}
            & Y
              \arrow{ur}[above left]{c}
              \arrow{dr}[below left]{d}
            & {}
            \\
              {}
            & {}
            & D
          \end{tikzcd}
        \]
  \end{enumerate}
\end{definition}


\begin{remark}
  Let~$\Ccat$ be a category that has a zero object, or that is preaddive.
  Let~$f \colon X \to Y$ be a morphism in~$\Ccat$.
  \begin{enumerate}
    \item
      A pair~$(K,k)$ is a kernel of~$f$ in~$\Ccat$ if and only if it is a cokernel of~$f$ in~$\Ccat^\op$.
    \item
      Kernels and cokernels are unique up to unique isomorphism.
    \item
      If every morphism in~$\Ccat$ has a kernel (resp.\ a cokernel) then the category~$\Ccat$ \emph{has kernels} (resp.\ \emph{has cokernels}).
    \item
      The kernel of~$f$ is denoted by~$\ker(f) \to X$, and the cokernel of~$f$ is denoted by~$Y \to \coker(f)$.
  \end{enumerate}
\end{remark}





\lecturend{10}


















