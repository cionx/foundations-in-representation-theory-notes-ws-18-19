\chapter{Abelian Categories}





\section{Monomorphisms and Epimorphisms}


\begin{definition}
  Let~$f \colon X \to Y$ be a morphism in a category~$\Ccat$.
  \begin{enumerate}
    \item
      The morphism~$f$ is a \emph{monomorphism}\index{monomorphism}\index{morphism!mono-} if it follows for every pair of parallel morphisms~$u, v \colon Z \to X$ from~$f \circ u = f \circ v$ that already~$u = v$.
    \item
      The morphism~$f$ is an \emph{epimorphism}\index{epimorphism}\index{morphism!epi-} if it follows for every pair of parallel morphisms~$u, v \colon Y \to Z$ from~$u \circ f = v \circ f$ that already~$u = v$.
  \end{enumerate}
\end{definition}


\begin{remark}
  Let~$f \colon X \to Y$ and~$g \colon Y \to Z$ be composable morphisms in a category~$\Ccat$.
  \begin{enumerate}
    \item
      If~$f$ is an isomorphism then it is both a monomorphism and an epimorphism.
    \item
      If both~$f$ and~$g$ are monomorphisms then their composition~$g \circ f$ is again a monomorphism.
      If both~$f$ and~$g$ are epimorphisms then their composition~$g \circ f$ is again an epimorphism.
    \item
      If the composition~$g \circ f$ is a monomorphism then~$f$ is a monomorphism.
      If the composition~$g \circ f$ is an epimorphism then~$g$ is an epimorphism.
    \item
      The morphism~$f$ is a monomorphism (in~$\Ccat$) if and only if it is an epimorphism in~$\Ccat^\op$.
  \end{enumerate}
\end{remark}


\begin{example}
  We give examples of monomorphisms.
  \begin{enumerate}
    \item
      In the category~$\Set$ the monomorphisms are precisely the injective maps.
      The same holds for the categories~$\Modl{A}$,~$\Group$,~$\Ring$,~$\CommRing$,~$\kAlg$,~$\kCommAlg$,~$\Top$.
    \item
      If~$Q$ is a quiver then in its path category~$\Path(Q)$ every morphism is a monomorphism:
      Let~$p = \alpha_\ell \dotsm \alpha_1$ be a morphism in~$Q$, i.e.\ a path in~$Q$.
      If~$u = u_r \dotsm u_1$ and~$v = v_s \dotsm v_1$ are morphisms in~$\Path(Q)$, i.e.\ paths in~$Q$, with~$s(u) = s(v)$ and~$t(u) = t(v) = s(p)$ then the equality~$p \circ u = p \circ v$ means that
      \[
          \alpha_\ell \dotsm \alpha_1 u_r \dotsm u_1
        = \alpha_\ell \dotsm \alpha_1 v_s \dotsm v_1 \,.
      \]
      It then follows that~$r = s$ and~$u_i = v_i$ for all~$i = 1, \dotsc, r$.
    \item
      Let~$\Conn_*$ be the category of pointed, connected topological spaces:
      The objects of~$\Conn_*$ are pairs~$(X, x_0)$ consisting of a connected topological space~$X$ and a base point~$x_0 \in X$.
      A morphism~$f \colon (X, x_0) \to (Y, y_0)$ is a continuous map~$f \colon X \to Y$ with~$f(x_0) = y_0$.
      The morphism~$f \colon (\Real, 0) \to (S^1, 1)$ with~$f(x) = e^{2 \pi i x}$ is then a monomorphism.
  \end{enumerate}
\end{example}


\begin{example}
  We give examples for epimorphisms.
  \begin{enumerate}
    \item
      In the category~$\Set$ a morphism is an epimorphism if and only if it surjective.
    \item
      If~$Q$ is a quiver then in its path category~$\Path(Q)$ every morphism in an epimorphism.
    \item
      Let~$\Haus$ be the category of Hausdorff topological spaces (where morphisms are just continuous maps).
      A morphism~$f \colon X \to Y$ in~$\Haus$ is an epimorphism if and only if it has dense image.
    \item
      Let~$A$ be a commutative ring and let~$S \subseteq A$ be a multiplicative set.
      Then the canonical map~$f \colon A \to S^{-1} A$,~$a \mapsto a/1$ is an epimorphism:
      If~$u,v \colon S^{-1} A \to B$ are two ring homomorphisms with~$u \circ f = v \circ f$ then~$u(a/1) = v(a/1)$ for every~$a \in A$.
      It then follows for every fraction~$a/s \in S^{-1} A$ that
      \[
          u\left( \frac{a}{s} \right)
        = u\left( \frac{a}{1} \right) u\left( \frac{s}{1} \right)^{-1}
        = v\left( \frac{a}{1} \right) v\left( \frac{s}{1} \right)^{-1}
        = v\left( \frac{a}{s} \right) \,,
      \]
      which then shows that~$u = v$.
  \end{enumerate}
\end{example}





\section{Special Objects}


\begin{definition}
  Let~$X$ be an object in a category~$\Ccat$.
  \begin{enumerate}
    \item
      The object~$X$ is \emph{initial}\index{initial object}\index{object!initial} if there exists  for every object~$Y \in \Ob(\Ccat)$ a unique morphism~$X \to Y$ in~$\Ccat$.
    \item
      The object~$X$ is \emph{terminal}\index{terminal object}\index{object!terminal} or \emph{final}\index{final object}\index{object!final} if there exists for every object~$Y \in \Ob(\Ccat)$ a unique morphism~$Y \to X$ in~$\Ccat$.
    \item
      The object~$X$ is a \emph{zero object}\index{zero!object}\index{object!zero} if it is both initial and terminal.
  \end{enumerate}
\end{definition}


\begin{remark}
  Let~$\Ccat$ be a category.
  \begin{enumerate}
    \item
      An object~$X$ of~$\Ccat$ is initial (in~$\Ccat$) if and only if it is terminal in~$\Ccat^\op$.
    \item
      Initial and terminal objects are unique up to unique isomorphisms (if they exist).
    \item
      If~$\Ccat$ admits a zero object then it is denoted by~$0 = 0_\Ccat$.
  \end{enumerate}
\end{remark}


\begin{example}
  \leavevmode
  \begin{enumerate}
    \item
      In the category~$\Set$, the empty set~$\emptyset$ is the unique initial object, and every \dash{one}{point} set~$\{\ast\}$ is a final object.
    \item
      In the category~$\Modl{A}$ the zero module~$0$ is the zero object.
    \item
      In the category~$\kAlg$ the~{\kalg}~$\kf$ is initial, while the zero algebra~$0$ is final.
  \end{enumerate}
\end{example}


\begin{remarkdefinition}
  Let~$\Ccat$ be a category which admits a zero object~$0$.
  Then there exist for any two objects~$X, Y \in \Ob(\Ccat)$ a unique morphism~$0_{XY} \colon X \to Y$ which factors trough the zero object, i.e.\ which makes the triangle
  \[
    \begin{tikzcd}
        X
        \arrow{dr}
        \arrow{rr}[above]{0_{XY}}
      & {}
      & Y
      \\
        {}
      & 0
        \arrow{ur}
      & {}
    \end{tikzcd}
  \]
  commute.
  The morphism~$0_{XY}$ is the \emph{zero morphism}\index{zero!morphism}\index{morphism!zero} from~$X$ to~$Y$.
\end{remarkdefinition}






\lecturend{8}





\section{Products and Coproducts}

\begin{definition}
  Let~$(X_i)_{i \in I}$ be a family of objects in a category~$\Ccat$.
  \begin{enumerate}
    \item
      A \emph{product}\index{product} of the family of objects~$(X_i)_{i \in I}$ is a pair~$(P, (p_i)_{i \in I})$ consisting of an object~$P \in \Ob(\Ccat)$ and morphisms~$p_i \colon P \to X_i$, such that for every other pair~$(Q, (q_i)_{i \in I})$ consisting of an object~$Q \in \Ob(\Ccat)$ and morphisms~$q_i \colon Q \to X_i$ there exists a unique morphism~$\lambda \colon Q \to P$ which makes the triangle
      \[
        \begin{tikzcd}[sep = large]
            P
            \arrow{d}[left]{p_i}
          & Q
            \arrow{dl}[below right]{q_i}
            \arrow[dashed]{l}[above]{\lambda}
          \\
            X_i
          & {}
        \end{tikzcd}
%         \begin{tikzcd}[sep = large]
%             Q
%             \arrow{dr}[above right]{q_i}
%             \arrow[dashed]{d}[left]{\lambda}
%           & {}
%           \\
%             P
%             \arrow{r}[above]{p_i}
%           & X_i
%         \end{tikzcd}
      \]
      commute for every~$i \in I$.
    \item
      A \emph{coproduct}\index{coproduct} of the family of objects~$(X_i)_{i \in I}$ is a pair~$(C, (c_i)_{i \in I})$ consisting of an object~$C \in \Ob(\Ccat)$ and morphisms~$c_i \colon X_i \to C$, such that for every pair~$(D, (d_i)_{i \in I})$ consisting of an object~$D \in \Ob(\Ccat)$ and morphisms~$d_i \colon X_i \to C$ there exists a unique morphism~$\mu \colon C \to D$ which makes the triangle
      \[
        \begin{tikzcd}[sep = large]
            C
            \arrow[dashed]{r}[above]{\mu}
          & D
          \\
            X_i
            \arrow{u}[left]{c_i}
            \arrow{ur}[below right]{d_i}
          & {}
        \end{tikzcd}
      \]
      commute for every~$i \in I$.
  \end{enumerate}
\end{definition}


\begin{remark}
  Let~$(X_i)_{i \in I}$ be a family of objects~$X_i$ in a category~$\Ccat$.
  \begin{enumerate}
    \item
      A pair~$(P,(p_i)_{i \in I})$ is a product of the family~$(X_i)_{i \in I}$ in~$\Ccat$ if and only if it is a coproduct of this family in~$\Ccat^\op$.
    \item
      Products are unique up to unique isomorphism, i.e.\ if~$(P, (p_i)_i)$ and~$(P', (p'_i)_{i \in I})$ are two products of the family~$(X_i)_{i \in I}$ in~$\Ccat$ then there exist a unique morphism~$\lambda \colon P \to P'$ which makes the triangle
      \[
        \begin{tikzcd}
            P
            \arrow[dashed]{rr}[above]{\lambda}
            \arrow{dr}[below left]{p_i}
          & {}
          & P'
            \arrow{dl}[below right]{p'_i}
          \\
            {}
          & X_i
          & {}
        \end{tikzcd}
      \]
      commute for every~$i \in I$, and the morphism~$\lambda$ is already an isomorphism.
      Similarly, coproducts are unique up to unique isomorphism.
    \item
      The product of the family~$(X_i)_{i \in I}$ is denoted by~$\prod_{i \in I} X_i$, or by~$X_1 \times \dotsb \times X_n$ if~$I = \{1, \dotsc, n\}$.
      The coproduct of the family~$(X_i)_{i \in I}$ is denoted by~$\coprod_{i \in I} X_i$, or by~$X_1 \dcup \dotsb \dcup X_n$ if~$I = \{1, \dotsc, n\}$.%
      \footnote{In the lecture the notations~$\bigsqcap_{i \in I} X_i$, resp.~$\bigsqcup_{i \in I} X_i$ and~$X_1 \sqcup \dotsb \sqcup X_n$ are used instead.}
    \item
      If every family of objects in~$\Ccat$ has a product (resp.\ coproduct) in~$\Ccat$ then we say that that~$\Ccat$ \emph{has products} (resp.\ \emph{has coproducts}).
      If every finite family of objects in~$\Ccat$ has a product (resp.\ coproducts) in~$\Ccat$ then we say that~$\Ccat$ \emph{has finite coproducts} (resp.\ \emph{has finite coproducts}).
  \end{enumerate}
\end{remark}


\begin{example}
  In the following let~$(X_i)_{i \in I}$ be a family objects in the given category~$\Ccat$.
  \begin{enumerate}
    \item
      Let~$\Ccat = \Set$.
      Then the (categorical) product~$\prod_{ \in I} X_i$ is the cartesian product, and the map~$p_i \colon \prod_{j \in I} X_j \to X_i$ is for every~$i \in I$ the usual projections onto the~\dash{$i$}{th} factor.
      The coproduct~$\coprod_{i \in I} X_i$ is the (formal) disjoint union of the sets~$X_i$, and the map~$c_i \colon X_i \to \coprod_{j \in I} X_j$ is for every~$i \in I$ the inclusion into the~\dash{$j$}{th} set.
    \item
      Let~$\Ccat = \Modl{A}$.
      Then the (categorical) product~$\prod_{i \in I} X_i$ is the cartesian product, and the morphism~$p_i \colon \prod_{j \in I} X_j \to X_i$ is for every~$i \in I$ the usual projection onto the~\dash{$i$}{th} factor.
      The coproduct of the family~$(X_i)_{i \in I}$ is the direct sum~$\bigoplus_{i \in I} X_i$, and the morphisms~$c_i \colon X_i \to \bigoplus_{j \in I} X_j$ is for every~$i \in I$ the inclusion into the~\dash{$i$}{th} summand.
    \item
      Let~$\Ccat = \kCommAlg$.
      Then the (categorical) product~$\prod_{i \in I} X_i$ is the cartesian product, and the morphism~$p_i \colon \prod_{j \in I} X_j \to X_i$ is for every~$i \in I$ the usual projection onto the~\dash{$i$}{th} factor.
      The coproduct of finitely many commutative~{\kalg}~$A_1, \dotsc, A_n$ in the category~$\kCommAlg$ is their tensor product~$A_1 \tensor \dotsb \tensor A_n$, and the morphism~$c_i \colon A_i \to A_1 \tensor \dotsb \tensor A_n$ is for every~$i \in I$ the inclusion into the~\dash{$i$}{th} factor, i.e.\ the algebra homomorphism
      \[
                A_i
        \to     A_1 \tensor \dotsb \tensor A_n \,,
        \quad   x
        \mapsto 1 \tensor \dotsb \tensor 1 \tensor x \tensor 1 \tensor \dotsb \tensor 1 \,.
      \]
      (The coproduct of an arbitrary family~$(A_i)_{i \in I}$ of commutative~{\kalgs} in the category~$\kCommAlg$ can be described similarly.)
  \end{enumerate}
\end{example}


\begin{lemma}
  \label{existence of coproducts}
  Let~$(X_i)_{i \in I}$ be a family of objects in a category~$\Ccat$.
  \begin{enumerate}
    \item
      The following are equivalent:
      \begin{enumerate}
        \item
          The product~$\prod_{i \in I} X_i$ exists in~$\Ccat$.
        \item
          The functor~$\Ccat^\op \to \Set$ given by~$Y \mapsto \prod_{i \in I} \Ccat(Y,X_i)$ is representable.
      \end{enumerate}
    \item
      \label{for coproducts}
      The following are equivalent:
      \begin{enumerate}
        \item
          The coproduct~$\coprod_{i \in I} X_i$ exists in~$\Ccat$.
        \item
          The functor~$\Ccat \to \Set$ given by~$Y \mapsto \prod_{i \in I} \Ccat(X_i,Y)$ is representable.
      \end{enumerate}
  \end{enumerate}
\end{lemma}


\begin{proof}
  It sufficies to show part~\ref*{for coproducts}.
  We denote the given functor~$\Ccat \to \Set$ by~$F$.
  
  Suppose first that the coproduct~$\coprod_{i \in I} X_i$ exists, and denote the associated morphisms by~$c_i \colon X_i \to \coprod_{j \in I} X_j$.
  We claim that the functor~$F$ is represented by the coproduct~$\coprod_{i \in I} X_i$;
  we thus need to construct a natural isomorphism~$\eta \colon h^{(\coprod_{i \in I} X_i)} \to F$.
  We define the components of~$\eta$ as
  \begin{align*}
              \eta_Y
     \colon   h^{(\coprod_{i \in I} X_i)}(Y)
     =        \Ccat\left( \coprod_{i \in I} X_i, Y \right)
    &\to      \prod_{i \in I} \Ccat(X_i, Y)
     =        F(Y) \,,
     \\
              f
    &\mapsto  (f \circ c_i)_{i \in I} \,.
  \end{align*}
  Then the family~$\eta \defined (\eta_Y)_{Y \in \Ob(\Ccat)}$ defines a natural transformation~$\eta \colon h^{(\coprod_{i \in I} X_i)} \to F$.
  Indeed, for every morphism~$g \colon Y \to Y'$ in~$\Ccat$ the diagram
  \[
    \begin{tikzcd}
        h^{(\coprod_{i \in I} X_i)}(Y)
        \arrow{rrr}[above]{h^{(\coprod_{i \in I} X_i)}(g)}
        \arrow{ddd}[left]{\eta_Y}
        \arrow[equal]{dr}
      & {}
      & {}
      & h^{(\coprod_{i \in I} X_i)}(Y')
        \arrow{ddd}[right]{\eta_{Y'}}
        \arrow[equal]{dl}
      \\
        {}
      & \Ccat\left( \coprod_{i \in I} X_i, Y \right)
        \arrow{r}[above]{g_*}
        \arrow{d}[left]{\eta_Y}
      & \Ccat\left( \coprod_{i \in I} X_i, Y' \right)
        \arrow{d}[right]{\eta_{Y'}}
      & {}
      \\
        {}
      & \prod_{i \in I} \Ccat(X_i, Y)
        \arrow{r}[below]{\prod_{i \in I} g_*}
      & \prod_{i \in I} \Ccat(X_i, Y')
      & {}
      \\
        F(Y)
        \arrow[equal]{ur}
        \arrow{rrr}[below]{F(g)}
      & {}
      & {}
      & F(Y')
        \arrow[equal]{ul}
    \end{tikzcd}
  \]
  commutes, because the inner square is given on elements by
  \[
    \begin{tikzcd}[column sep = huge, row sep = large]
        f
        \arrow[maps to]{r}[above]{g_*}
        \arrow[maps to]{d}[left]{\eta_Y}
      & g \circ f
        \arrow[maps to]{d}[right]{\eta_{Y'}}
      \\
        (f \circ c_i)_{i \in I}
        \arrow[maps to]{r}[above]{\prod_{i \in I} g_*}
      & (g \circ f \circ c_i)_{i \in I}
    \end{tikzcd}
  \]
  and thus commutes.
  
  The natural transformation~$\eta$ is already a natural isomorphism:
  There exist at every objects~$Y \in \Ob(\Ccat)$ for every family of morphisms~$(h_i)_{i \in I} \in \prod_{i \in I} \Ccat(X_i, Y)$ by the definition of the coproduct~$\coprod_{i \in I} X_i$ a unique morphism~$g \in \Ccat(\coprod_{i \in I} X_i, Y)$ with~$g \circ c_i = h_i$ for every~$i \in I$, i.e.\ with~$\eta_Y(g) = (h_i)_{i \in I}$.
  This means that~$\eta_Y$ is bijective at every~$Y \in \Ob(\Ccat)$.
  
  Suppose now that the functor~$F$ is representable.
  Let~$C$ be a representing object for~$F$ and let~$\eta \colon h^C \to F$ be a natural isomorphism.
  Then~$\eta_C$ is a map
  \[
      \eta_C
    \colon
      h^C(C)
    =
      \Ccat(C,C)
    \to
      F(C)
    =
      \prod_{i \in I} \Ccat(X_i, C) \,.
  \]
  By setting $(c_i)_{i \in I} \defined \eta_C(\id_C)$ we therefore get for every~$i \in I$ a morphism~$c_i \colon X_i \to C$.
  We show that the pair~$(C, (c_i)_{i \in I})$ is a coproduct of the family~$(X_i)_{i \in I}$.
  So let~$(D, (d_i)_{i \in I})$ be another pair consisting of an object~$D \in \Ob(\Dcat)$ and a family~$(d_i)_{i \in I}$ of morphisms~$d_i \colon X_i \to D$.
  Then
  \[
        (d_i)_{i \in I}
    \in \prod_{i \in I} \Ccat(X_i, D)
    =   F(D)
  \]
  and it follows from~$\eta_D \colon h^C(D) \to F(D)$ being a bijection that there exist a unique element~$\lambda \in h^C(D) = \Ccat(C,D)$, i.e.\ morphism~$\lambda \colon C \to D$, with~$(d_i)_{i \in I} = \eta_D(\lambda)$.
  It follows from the naturality of~$\eta$ that the diagram
  \[
    \begin{tikzcd}[row sep = large]
        \Ccat(C,C)
        \arrow[equal]{r}
        \arrow{d}[left]{\lambda_*}
      & h^C(C)
        \arrow{r}[above]{\eta_C}
        \arrow{d}[left]{h^C(\lambda)}
      & F(C)
        \arrow{d}[right]{F(\lambda)}
        \arrow[equal]{r}
      & \prod_{i \in I} \Ccat(X_i, C)
        \arrow{d}[right]{\prod_{i \in I} \lambda_*}
      \\
        \Ccat(C,D)
        \arrow[equal]{r}
      & h^C(D)
        \arrow{r}[below]{\eta_D}
      & F(D)
        \arrow[equal]{r}
      & \prod_{i \in I} \Ccat(X_i, D)
    \end{tikzcd}
  \]
  commutes.
  It therefore follows for the element~$\id_C \in \Ccat(\Ccat, \Ccat) = h^C(C)$ that
  \begin{align*}
        (d_i)_{i \in I}
     =  \eta_D( \lambda )
     =  \eta_D( \lambda_*( \id_C) )
    &=  \eta_D( h^C(\lambda)( \id_C ) )  \\
    &=  F(\lambda)( \eta_C( \id_C) ) )
     =  F(\lambda)( (c_i)_{i \in I} )
     =  (\lambda \circ c_i)_{i \in I} \,,
  \end{align*}
  and hence that~$\lambda \circ c_i = d_i$ for every~$i \in I$.
  This shows the existence of the required morphism~$\lambda \colon C \to D$.
  By reading the above argumentation from the bottom to the top we also find that the morphism~$\lambda$ is unique.
  
  This shows that the object~$C$ together with the morphisms~$c_i \colon X_i \to C$ is a coproduct of the family~$(X_i)_{i \in I}$.
\end{proof}


\begin{remark*}
  The above proof actually shows that an object~$C \in \Ob(\Ccat)$ is a coproduct of the family~$(X_i)_{i \in I}$, with respect to some suitable morphisms~$c_i \colon X_i \to C$, if and only if it represents the functor~$\prod_{i \in I} \Ccat(X_i, -) \colon \Ccat \to \Set$.
  This statement is stronger than the formulation in \cref{existence of coproducts}.
  
  One can also show a slightly stronger version of this:
  It follows for every object~$C \in \Ob(\Ccat)$ from Yoneda’s~lemma that the map
  \begin{align*}
              \{ \text{natural transformations~$\eta \colon h^C \to F$} \}
    &\to      F(C) \,,
    \\
              \eta
    &\mapsto  \eta_C(\id_C)
  \end{align*}
  is a bijection.
  An element of the right hand side is an element of~$F(C)$, i.e.\ a family~$(c_i)_{i \in I}$ of morphisms~$c_i \colon X_i \to C$.
  It then holds that a natural transformation~$\eta \colon h^C \to F$ is an isomorphism if and only if the corresponding family~$(c_i)_{i \in I}$ makes the pair~$(C, (c_i)_{i \in I})$ into a coproduct of the family~$(X_i)_{i \in I}$.
  
  Indeed, that the natural transformation~$\eta$ is a natural isomorphism means that at every object~$D \in \Ob(\Ccat)$ the map
  \[
            \eta_D
    \colon  \Ccat(C,D)
    =       h^C(D)
    \to     F(D)
  \]
  is a bijection.
  The component~$\eta_D$ can be expressed via the element~$(c_i)_{i \in I} \in F(C)$ as
  \[
      \eta_D(\lambda)
    = F(\lambda)( (c_i)_{i \in I} )
    = (\lambda \circ c_i)_{i \in I} \,.
  \]
  The bijectivity of~$\eta_D$ therefore means that for every~$(d_i)_{i \in I} \in F(D) = \prod_{i \in I} \Ccat(X_i, D)$ there exists a unique element~$\lambda \in \Ccat(C, D)$ with~$(\lambda \circ c_i)_{i \in I} = (d_i)_{i \in I}$.
  In other words, there exists for every object~$D \in \Ob(\Ccat)$ and every family~$(d_i)_{i \in I}$ of morphisms~$d_i \colon X_i \to D$ a unique morphism~$\lambda \colon C \to D$ with~$d_i = \lambda \circ c_i$ for every~$i \in I$.
  But this is precisely what it means for the pair~$(C, (c_i)_{i \in I})$ to be a coproduct of the family~$(X_i)_{i \in I}$.
  
  This shows that for every object~$C \in \Ob(\Ccat)$, a family~$(c_i)_{i \in I}$ of morphisms~$c_i \colon X_i \to C$ that makes~$(C, (c_i)_{i \in I})$ into a coproduct of the family~$(X_i)_{i \in I}$ is \enquote{the same} as a natural isomorphism~$\eta \colon h^C \to F$ (via Yoneda’s~lemma).
  It follows in particular that~$C$ is a coproduct of the family~$(X_i)_{i \in I}$, with respect to a suitable choice of morphisms~$c_i \colon X_i \to C$, if and only if there exist a natural isomorphism~$h^C \to F$.
  (This is what was shown in the above proof.)
\end{remark*}





\section{Additive Categories}


\begin{definition}
  A \emph{preadditive~category} is a category~$\Acat$ together with the structure of an abelian group on~$\Acat(X,Y)$ for all~$X, Y \in \Ob(\Acat)$ such that the composition in~$\Ccat$ is~{\Zbilin}, i.e.\ such that
  \[
    k \circ (g + h) = k \circ g + k \circ h
    \quad\text{and}\quad
    (g + h) \circ f = g \circ f + h \circ f
  \]
  for all morphisms~$f \colon W \to X$,~$g, h \colon X \to Y$ and~$k \colon Y \to Z$ in~$\Acat$.
\end{definition}


\begin{remark}
  \leavevmode
  \begin{enumerate}
    \item
      Preddative categories are also known as~\dash{$\Ab$}{categories} (where~$\Ab$ denotes the category of abelian groups).
      (In the language of enriched category theory, an~\dash{$\Ab$}{category} is precisely a category that is enriched over the monoidal category~$(\Ab,{\tensor})$.)
    \item
      If~$\kf$ is a commutative ring then one can similarly define the notation of a {\preklin} category (also known as~\dash{$\Modl{k}$}{category})~$\Ccat$.
      Every~$\Ccat(X,Y)$ is then endowed with the structure of a~{\module{$\kf$}} and the composition is~{\kbilin}.
  \end{enumerate}
\end{remark}


\begin{example}
  \leavevmode
  \begin{enumerate}
    \item
      The category~$\Ab = \Modl{\Integer}$ is preadditive.
    \item
      If~$A$ is a~{\kalg} then the categories~$\Modl{A}$ and~$\Modr{A}$ are {\preklin}.
    \item
      If~$R$ is a ring then we may think about~$R$ as a preadditive category~$\Rcat$ consisting of a single object~$\Ob(\Rcat) = \{ \ast \}$ with~$\Rcat(\ast,\ast) = R$.
      The composition in~$\Rcat$ is given by the multiplication of~$R$, i.e.\ by~$g \circ f = gf$ for all~$f, g \in R$, and the addition of morphisms is the addition in~$R$.
      
      One can similarly regard every~{\kalg}~$A$ as a {\preklin} category~$\Acat$ which consists of a single object~$\Ob(\Acat) = \{\ast\}$ with~$\Acat(\ast,\ast) = A$.
    \item
      Let~$\Ccat$ be any category and let~$\Acat$ be a preadditive category.
      Then the functor category~$\Fun(\Ccat, \Acat)$ is again preadditive:
      For any two natural transformations~$\eta, \zeta \colon F \to G$ between functors~$F, G \in \Ob(\Fun(\Ccat, \Acat))$, their sum~$\eta + \zeta$ is at an object~$X \in \Ob(\Ccat)$ given by
      \[
          (\eta + \zeta)_X
        = \eta_X + \zeta_X \,.
      \]
      This defines again a natural transformation~$\eta + \zeta \colon F \to G$.
      Indeed, for every morphism~$f \colon X \to X'$ in~$\Ccat$ the square
      \[
        \begin{tikzcd}[sep = large]
            F(X)
            \arrow{r}[above]{F(f)}
            \arrow{d}[left]{\eta_X + \zeta_X}
          & F(X')
            \arrow{d}[right]{\eta_{X'} + \zeta_{X'}}
          \\
            G(X)
            \arrow{r}[above]{G(f)}
          & G(X')
        \end{tikzcd}
      \]
      commutes because
      \begin{align*}
            (\eta_{X'} \circ \zeta_{X'}) \circ F(f)
        &=  \eta_{X'} \circ F(f) + \zeta_{X'} \circ F(f)  \\
        &=  G(f) \circ \eta_X + G(f) \circ \zeta_X
         =  G(f) \circ (\eta_X + \zeta_X) \,.
      \end{align*}
      
      We find similarly that for every category~$\Ccat$ and every {\preklin} category~$\Acat$ the functor category~$\Fun(\Ccat, \Acat)$ is again~{\preklin}. 
  \end{enumerate}
\end{example}


\begin{definition}
  \leavevmode
  \begin{enumerate}
    \item
      A functor~$F \colon \Acat \to \Bcat$ between preadditive categories~$\Acat$ and~$\Bcat$ is \emph{additive} if
      \[
          F(f + g)
        = F(f) + F(g)
      \]
      for all morphisms~$f, g \colon X \to X'$ in~$\Acat$, i.e.\ if the map
      \[
                    \Acat(X, Y)
        \xlongto{F} \Bcat(F(X), F(Y))
      \]
      is a group homorphism for all~$X, Y \in \Ob(\Acat)$.
    \item
      A functor~$F \colon \Acat \to \Bcat$ between {\preklin} categories~$\Acat$ and~$\Bcat$ is~\emph{{\klin}} if
      \[
        F(f + g) = F(f) + F(g)
        \quad\text{and}\quad
        F(\lambda f) = \lambda F(f)
      \]
      for all morphisms~$f, g \colon X \to X'$ in~$\Acat$ and scalars~$\lambda \in \kf$, i.e.\ if the map
      \[
                    \Acat(X, Y)
        \xlongto{F} \Bcat(F(X), F(Y))
      \]
      is~{\klin} for all~$X, Y \in \Ob(\Acat)$.%
      \footnote{The notion of a~{\klin} functor was not introduced in the lecture.}
  \end{enumerate}
\end{definition}


\begin{lemma}
  Let~$\Acat$ be a preadditive category.
  \begin{enumerate}
    \item
      An object~$X$ of~$\Acat$ is initial if and only if it is terminal.
    \item
      Suppose that the category~$\Acat~$ has a zero object.
      Then it holds for any two objects~$X, Y \in \Ob(\Acat)$ that~$0_{X,Y} = 0_{\Acat(X,Y)}$.
      (Here~$0_{X,Y}$ denotes the unique morphism~$X \to Y$ which factors through the zero object of~$\Acat$, and~$ _{\Acat(X,Y)}$ denotes the neutral element of the abelian group~$\Acat(X,Y)$.)
  \end{enumerate}
\end{lemma}


\begin{proof}
  This is part~(i) of Exercise~3 of the fifth exercise sheet.
\end{proof}


























