\section{Constructions with Complexes}


\begin{definition}
  \leavevmode
  \begin{enumerate}
    \item
      Let~$\Ccc$ be a chain complex in~$\Acat$.
      Then for every~$p \in \Integer$ the \emph{shifted chain complex}~$\Ccc[p]$\index{shifted!chain complex}\index{chain complex!shifted} is given by
      \[
                  (\Ccc[p])_n
        \defined  C_{n+p}
        \quad\text{and}\quad
                  d^{\,\Ccc[p]}_n
        \defined  (-1)^p d^{\,\Ccc}_{n+p}
      \]
      for every~$n \in \Integer$.
    \item
      Let~$\Cccc$ be a chain complex in~$\Acat$.
      Then for every~$p \in \Integer$ the \emph{shifted cochain complex}~$\Cccc[p]$\index{shifted!cochain complex}\index{cochain complex!shifted} is given by
      \[
                  (\Cccc[p])^n
        \defined  C^{n-p}
        \quad\text{and}\quad
                  d_{\,\Cccc[p]}^n
        \defined  (-1)^p d_{\,\Cccc}^{n-p}
      \]
      for every~$n \in \Integer$.
  \end{enumerate}
\end{definition}


\begin{remark}
  \leavevmode
  \begin{enumerate}
    \item
      Shifting complexes is covariantly functorial (for both chain complexes and cochain complexes):
      
      If~$f = (f_n)_{n \in \Integer} \colon \Ccc \to \Dcc$ is a morphism of chain complexes then for every~$p \in \Integer$ the family~$f[p] \defined (f_{n+p})_{n \in \Integer}$ is a morphism of chain complexes~$f[p] \colon \Ccc[p] \to \Dcc[p]$.
      Dually, if~$f = (f^n)_{n \in \Integer} \colon \Cccc \to \Dccc$ is a morphism of cochain complexes then for every~$p \in \Integer$ the family~$f[p] \defined (f^{n-p})_{n \in \Integer}$ is a morphism of cochain complexes~$f[p] \colon \Cccc[p] \to \Dccc[p]$.
      
      The shift operator~$[1]$ is also denoted by~$\Sigma$ (for both chain complexes and cochain complexes).
    \item
      Let~$p \in \Integer$.
      If~$\Ccc$ is a chain complex in~$\Acat$ then
      \[
          \Hl_n(\Ccc[p])
        = \Hl_{n+p}(\Ccc)
      \]
      for every~$n \in \Integer$, and if~$\Cccc$ is a cochain complex in~$\Acat$ then
      \[
          \Hl^n(\Cccc[p])
        = \Hl^{n-p}(\Cccc)
      \]
      for every~$n \in \Integer$.
  \end{enumerate}
\end{remark}


\begin{remark}
  Let~$\Ccc$ be a chain complex in~$\Acat$.
  We can consider the subcomplexes~$\Zcc$ and~$\Bcc$ of~$\Ccc$ that are given in components by
  \[
      Z_n
    = \Zl_n(\Ccc)
    \quad\text{and}\quad
      B_n
    = \Bl_n(\Ccc)
  \]
  for every~$n \in \Integer$, together with the zero morphisms as differentials.
  We then get a short exact sequence of chain complexes
  \[
    0
    \to
    \Zcc
    \to
    \Ccc
    \to
    \Bcc[-1]
    \to
    0 \,.
  \]
  
  We get dually for every cochain complex~$\Cccc$ in~$\Acat$ a short exact sequence of cochain complexes
  \[
    0
    \to
    \Zccc
    \to
    \Cccc
    \to
    \Bccc[-1]
    \to
    0 \,.
  \]

\end{remark}





\lecturend{16}




\begin{remarkdefinition}
  \leavevmode
  \begin{enumerate}
    \item
      Let~$\Ccc$ be a chain complex in~$\Acat$ and let~$p \in \Integer$.
      \begin{itemize}
        \item
          The differential~$d_{p+1} \colon C_{p+1} \to C_p$ induces for~$\widetilde{C}_p \defined \Zl_p(\Ccc) = \ker(d_p)$ a morphism~$\tilde{d}_{p+1} \colon C_{p+1} \to \widetilde{C}_p$.
          This results in a chain complex
          \[
                      \tau_{\,\geq p}(\Ccc)
            \defined  \left(
                        \dotsb
                        \to
                        C_{p+2}
                        \xlongto{d_{p+2}}
                        C_{p+1}
                        \xlongto{\tilde{d}_{p+1}}
                        \widetilde{C}_p
                        \to
                        0
                        \to
                        0
                        \to
                        \dotsb
                      \right) \,.
          \]
        \item
          The differential~$d_p \colon C_p \to C_{p-1}$ induces for~$\widetilde{C}_p \defined \coker(d_{p+1})$ a morphism~$\tilde{d}_p \colon \tilde{C}_p \to C_{p-1}$.
          This results in a chain complex
          \[
                      \tau_{\,\leq p}(\Ccc)
            \defined  \left(
                        \dotsb
                        \to
                        0
                        \to
                        0
                        \to
                        \widetilde{C}_p
                        \xlongto{\tilde{d}_p}
                        C_{p-1}
                        \xlongto{d_{p+1}}
                        C_{p-2}
                        \to
                        \dotsb
                      \right).
          \]
        \end{itemize}
        The chain complexes~$\tau_{\,\geq p}(\Ccc)$ and~$\tau_{\,\leq p}(\Ccc)$ are the \emph{truncated complexes}\index{truncation}\index{chain complex!truncation} of~$\Ccc$.
      \item
        Dually, let~$\Cccc$ be a cochain complex in~$\Acat$ and let~$p \in \Integer$.
      \begin{itemize}
        \item
          The differential~$d^{p-1} \colon C^{p-1} \to C^p$ induces for~$\widetilde{C}^p \defined \Zl^p(\Cccc) = \ker(d^p)$ a morphism~$\tilde{d}^{p-1} \colon C^{p-1} \to \widetilde{C}^p$.
          This results in a cochain complex
          \[
                      \tau^{\,\leq p}(\Cccc)
            \defined  \left(
                        \dotsb
                        \to
                        C^{p-2}
                        \xlongto{d^{p-2}}
                        C^{p-1}
                        \xlongto{\tilde{d}^{p-1}}
                        \widetilde{C}^p
                        \to
                        0
                        \to
                        0
                        \to
                        \dotsb
                      \right) \,.
          \]
        \item
          The differential~$d^p \colon C^p \to C^{p+1}$ induces for~$\widetilde{C}^p \defined \coker(\tilde{d}^{p+1})$ a morphism~$\tilde{d}^p \colon \widetilde{C}^p \to C^{p+1}$.
          This results in a cochain complex
          \[
                      \tau^{\,\geq p}(\Cccc)
            \defined  \left(
                        \dotsb
                        \to
                        0
                        \to
                        0
                        \to
                        \widetilde{C}^p
                        \xlongto{\tilde{d}^p}
                        C^{p+1}
                        \xlongto{d^{p+1}}
                        C^{p+2}
                        \to
                        \dotsb
                      \right).
          \]
      \end{itemize}
      The chain complexes~$\tau^{\,\leq p}(\Cccc)$ and~$\tau^{\,\geq p}(\Cccc)$ are the \emph{truncated complexes}\index{truncation}\index{cochain complex!truncation} of~$\Cccc$.
  \end{enumerate}
\end{remarkdefinition}


\begin{remark}
  \leavevmode
  \begin{enumerate}
    \item
      Truncation of complexes is functorial (for both chain complexes and cochain complexes):
      
      Let~$f = (f_n)_{n \in \Integer} \colon \Ccc \to \Dcc$ be a morphism of chain complexes and let~$p \in \Integer$.
      Then the morphism~$f_p \colon C_p \to D_p$ restrict to a morphism~$\tilde{f}_p \colon \ker(d^C_p) \to \ker(d^D_p)$. 
      We get from this a morphism of chain complexes
      \[
                \tau_{\,\geq p}(f)
        \colon  \tau_{\,\geq p}(\Ccc)
        \to     \tau_{\,\geq p}(\Dcc)
      \]
      that is given in components given by
      \[
                  \tau_{\,\geq p}(f)_n
        \defined  \begin{cases}
                    f_n         & \text{if~$n > p$} \,, \\
                    \tilde{f}_n & \text{if~$n = p$} \,, \\
                    0           & \text{if~$n < p$} \,.
                  \end{cases}
      \]
      We similarly get an induced morphism~$\tilde{f}_p \colon \coker(d^C_{p+1}) \to \coker(d^D_{p+1})$, from which we get a morphism of chain complexes
      \[
                \tau_{\,\leq p}(f)
        \colon  \tau_{\,\leq p}(\Ccc)
        \to     \tau_{\,\leq p}(\Dcc)
      \]
      that is given in components by
      \[
                  \tau_{\,\leq p}(f)_n
        \defined  \begin{cases}
                    f_n         & \text{if~$n < p$} \,, \\
                    \tilde{f}_n & \text{if~$n = p$} \,, \\
                    0           & \text{if~$n > p$} \,.
                  \end{cases}
      \]
      These induced morphisms are compatible with identity morphisms and composition of morphisms by the \hyperref[functoriality of (co)kernel]{functoriality of (co)kernels}.
      
      For truncation of cochain complexes the analogous construction works.
    \item
      If~$\Cccc$ is a chain complex in~$\Acat$ and~$p \in \Integer$ then
      \begin{align*}
                \Hl_n(\tau_{\,\geq p}(\Ccc))
        &\cong  \begin{cases}
                  \Hl_n(\Ccc) & \text{if~$n \geq p$}  \,, \\
                  0           & \text{if~$n < p$}     \,,
                \end{cases}
      \shortintertext{and}
                \Hl_n(\tau_{\,\leq p}(\Ccc))
        &\cong  \begin{cases}
                  \Hl_n(\Ccc) & \text{if~$n \leq p$}  \,, \\
                  0           & \text{if~$n > p$}     \,.
                \end{cases}
      \end{align*}
      Similarly, if~$\Cccc$ is a cochain complex in~$\Acat$ and~$p \in \Integer$ then
      \begin{align*}
                \Hl^n(\tau^{\,\leq p}(\Cccc))
        &\cong  \begin{cases}
                  \Hl^n(\Cccc)  & \text{if~$n \leq p$}  \,, \\
                  0             & \text{if~$n > p$}     \,,
                \end{cases}
      \shortintertext{and}
                \Hl^n(\tau_{\,\geq p}(\Cccc))
        &\cong  \begin{cases}
                  \Hl^n(\Cccc)  & \text{if~$n \geq p$}  \,, \\
                  0             & \text{if~$n < p$}     \,.
                \end{cases}
      \end{align*}
  \end{enumerate}
\end{remark}


% Make this isomorphism of homologies more precise: It is induced by the canonical morphisms of chain complexes.

\begin{remark*}
  If~$\Ccc$ as a chain complex in~$\Acat$ and~$p \in \Integer$ then one can also define the \enquote{stupid truncations}\index{stupid truncation}\index{truncation!stupid}\index{chain complex!truncation!stupid}
  \begin{align*}
              \sigma_{\,\geq p}(\Ccc)
    &\defined (
                \dotsb
                \to
                C_{p+2}
                \xlongto{d_{p+2}}
                C_{p+1}
                \xlongto{d_{p+1}}
                C_p
                \to
                0
                \to
                0
                \to 
                \dotsb
              )
  \shortintertext{and}
              \sigma_{\,\leq p}(\Ccc)
    &\defined (
                \dotsb
                \to
                0
                \to
                0
                \to
                C_p
                \xlongto{d_p}
                C_{p-1}
                \xlongto{d_{p-1}}
                C_{p-2}
                \to
                \dotsb
              ) \,.
  \end{align*}
  For a cochain complex~$\Cccc$ in~$\Acat$ and~$p \in \Integer$ one can define the \enquote{stupid truncations}\index{stupid truncation}\index{truncation!stupid}\index{cochain complex!truncation!stupid}
  \begin{align*}
              \sigma^{\,\leq p}(\Cccc)
    &\defined (
                \dotsb
                \to
                C^{p-2}
                \xlongto{d^{p-2}}
                C^{p-1}
                \xlongto{d^{p-1}}
                C_p
                \to
                0
                \to
                0
                \to
                \dotsb
              )
  \shortintertext{and}
              \sigma^{\,\geq p}(\Cccc)
    &\defined (
                \dotsb
                \to
                0
                \to
                0
                \to
                C^p
                \xlongto{d^p}
                C^{p+1}
                \xlongto{d^{p+1}}
                C^{p+2}
                \to
                \dotsb
              ) \,.
  \end{align*}
  But then
  \begin{alignat*}{2}
          \Hl_p( \sigma_{\,\geq p}(\Ccc) )
    &\cong \coker(d_{p+1}) \,,
    &\quad&
          \Hl_p( \sigma_{\,\leq p}(\Ccc) )
    =     \Zl_p(\Ccc) \,,
  \shortintertext{and}
          \Hl^p( \sigma^{\,\leq p}(\Cccc) )
    &\cong \coker(d^{p-1}) \,,
    &\quad&
          \Hl^p( \sigma^{\,\geq p}(\Cccc) )
    =     \Zl^p(\Cccc)  \,.
  \end{alignat*}
  Hence the~\dash{$p$}{th} (co)homology is (in general) not preserved by the stupid truncations.
\end{remark*}




