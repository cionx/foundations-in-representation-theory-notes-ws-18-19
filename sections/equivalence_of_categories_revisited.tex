\section{Equivalence of Categories Revisited}


\begin{theorem}
  A functor~$F \colon \Ccat \to \Dcat$ between two categories~$\Ccat$ and~$\Dcat$ is an equivalence if and only if it is both fully faithful and dense.
\end{theorem}


\begin{proof}
  Suppose first that~$F$ is an equivalence.
  Then let~$G \colon \Dcat \to \Ccat$ be a functor with~$G \circ F \cong \Id_\Ccat$ and~$F \circ G \cong \Id_\Dcat$, and let~$\eta \colon G \circ F \to \Id_\Ccat$ and~$\zeta \colon F \circ G \to \Id_\Dcat$ be natural isomorphisms.
  
  The functor~$F$ is dense because it holds for every object~$Y \in \Ob(\Dcat)$ that~$F(X) \cong Y$ for the object~$X \defined G(Y) \in \Ob(\Ccat)$ via the isomorphism~$\zeta_Y \colon F(G(Y)) \to Y$.
  
  The proof that~$f$ is fully faithful which was given in the lecture doesn’t seem correct;
  we%
  \footnote{Either Dr.\ Franzen or the author.}
  will add a proof in the near future.
  
  Suppose on the other hand that the functor~$F$ is both fully faithful and dense.
  For every object~$Y \in \Ob(\Dcat)$ let~$G(Y) \in \Ob(\Ccat)$ be an object with~$FG(Y) \cong Y$;
  we choose an isomorphism~$\varepsilon_Y \colon FG(Y) \to Y$.
  If~$g \colon Y \to Y'$ is a morphism in~$\Dcat$ then there exist for the conjugated morphism~$\varepsilon_{Y'}^{-1} \circ g \circ \varepsilon_Y \colon FG(Y) \to FG(Y')$ a unique morphism~$G(g) \colon G(Y) \to G(Y')$ in~$\Dcat$ with~$FG(g) = \varepsilon_{Y'}^{-1} \circ g \circ \varepsilon_Y$, because~$F$ is fully faithful.
  
  We claim that~$G$ is a functor~$G \colon \Dcat \to \Ccat$ with both~$G \circ F \cong \Id_\Ccat$ and~$F \circ G \cong \Id_\Dcat$.
  
  We first show that~$G$ is a functor:
  If~$Y \in \Ob(\Dcat)$ then
  \[
      \varepsilon_Y^{-1} \circ \id_Y \circ \varepsilon_Y
    = \id_{FG(Y)}
    = F(\id_{G(Y)})
  \]
  and hence~$\id_{G(Y)} = G(\id_Y)$.
  It holds for any two composable morphisms~$g \colon Y \to Y'$ and~$g' \colon Y' \to Y''$ in~$\Dcat$ that
  \begin{align*}
     {}&  \varepsilon_{Y''}^{-1} \circ (g' \circ g) \circ \varepsilon_Y \\
    ={}&  \varepsilon_{Y''}^{-1} \circ g' \circ \varepsilon_{Y'}
          \circ
          \varepsilon_{Y'}^{-1} \circ g \circ \varepsilon_Y \\
    ={}&  FG(g') \circ FG(g)  \\
    ={}&  F( G(g') \circ G(g) ) \,,
  \end{align*}
  which shows that~$G(g' \circ g) = G(g') \circ G(g)$.
  
  To show that~$F \circ G \cong \Id_\Dcat$ we note that~$\varepsilon \defined (\varepsilon_Y)_{Y \in \Ob(\Dcat)}$ is a natural isomorphism~$\varepsilon \colon F \circ G \to \Id_\Dcat$.
  That~$\varepsilon$ is a natural transformation, i.e.\ that the square
  \[
    \begin{tikzcd}[sep = large]
        FG(Y)
        \arrow{r}[above]{FG(g)}
        \arrow{d}[left]{\varepsilon_Y}
      & FG(Y')
        \arrow{d}[right]{\varepsilon_{Y'}}
      \\
        Y
        \arrow{r}[above]{g}
      & Y'
    \end{tikzcd}
  \]
  commutes for every morphism~$g \colon Y \to Y'$ in~$\Dcat$, holds by construction of~$G(g)$.
  That~$\varepsilon_Y$ is an isomorphism for every~$Y \in \Dcat$ holds by choice of~$\varepsilon_Y$.
  
  To show that~$G \circ F \cong \Id_\Ccat$ we construct a natural isomorphism~$\eta \colon G \circ F \to \Id_\Ccat$:
  
  There exist for every object~$X \in \Ccat$ for the morphisms~$\varepsilon_{F(X)} \colon FGF(X) \to F(X)$ a unique morphisms~$\eta_X \colon GF(X) \to X$ with~$\varepsilon_{F(X)} = F(\eta_X)$ because~$F$ is fully faithful.
  We set~$\eta \defined (\eta_X)_{X \in \Ob(\Ccat)}$.
  
  The family~$\eta$ is a natural transformation~$\eta \colon G \circ F \to \Id_\Ccat$:
  Let~$f \colon X \to X'$ be a morphism in~$\Ccat$.
  Then the square
  \[
    \begin{tikzcd}[sep = large]
        FGF(X)
        \arrow{r}[above]{FGF(f)}
        \arrow{d}[left]{\varepsilon_{F(X)}}
      & FGF(X')
        \arrow{d}[right]{\varepsilon_{F(X')}}
      \\
        F(X)
        \arrow{r}[above]{F(f)}
      & F(X')
    \end{tikzcd}
  \]
  commutes because~$\varepsilon \colon FG \to \Id_\Dcat$ is a natural transformation.
  We may rewrite this diagram as
  \[
    \begin{tikzcd}[sep = large]
        FGF(X)
        \arrow{r}[above]{FGF(f)}
        \arrow{d}[left]{F(\eta_X)}
      & FGF(X')
        \arrow{d}[right]{F(\eta_{X'})}
      \\
        F(X)
        \arrow{r}[above]{F(f)}
      & F(X')
    \end{tikzcd}
  \]
  by construction of~$\eta$.
  We thus find that
  \[
      F(f \circ \eta_X)
    = F(f) \circ F(\eta_X)
    = F(\eta_{X'}) \circ FGF(f)
    = F(\eta_{X'} \circ GF(f)) \,.
  \]
  It follows from~$F$ being faithful that already
  \[
      \eta_{X'} \circ GF(f)
    = f \circ \eta_X \,,
  \]
  i.e.\ that the square
  \begin{equation}
    \label{naturality of eta}
    \begin{tikzcd}[sep = large]
        GF(X)
        \arrow{r}[above]{GF(f)}
        \arrow{d}[left]{\eta_X}
      & GF(X')
        \arrow{d}[right]{\eta_{X'}}
      \\
        X
        \arrow{r}[above]{f}
      & X'
    \end{tikzcd}
  \end{equation}
  commutes.
  This shows that~$\eta \colon GF \to \Id_\Ccat$ is indeed a natural transformation.
  
  It follows for every~$X \in \Ob(\Ccat)$ from~$\varepsilon_{F(X)} = F(\eta_X)$ being an isomorphism that~$\eta_X$ is again an isomorphism:
  Indeed, there exists for the inverse~$\varepsilon_{F(X)}^{-1} \colon F(X) \to FGF(X)$ by the fully faithfulness of~$F$ a unique morphism~$\eta'_X \colon X \to GF(X)$ with~$\varepsilon_{F(X)}^{-1} = F(\eta'_X)$.
  Then
  \[
      F(\eta_X \circ \eta'_X)
    = F(\eta_X) \circ F(\eta'_X)
    = \varepsilon_{F(X)} \circ \varepsilon_{F(X)}^{-1}
    = \id_{F(X)}
    = F(\id_X)
  \]
  and hence~$\eta_X \circ \eta'_X = \id_X$ because~$F$ is faithful.
  It can be shown similarly that also~$\eta'_X \circ \eta_X = \id_{GF(X)}$.
  This shows that the morphism~$\eta_X$ is an isomorphism with~$\eta_X^{-1} = \eta'_X$.
  
  This shows altogether the claim that~$\eta$ is a natural isomorphism~$\eta \colon G \circ F \to \Id_\Ccat$.
\end{proof}


\begin{remark*}
  The above proof displays an important property that a faithful functor~$F \colon \Ccat \to \Dcat$ possesses:
  \begin{enumerate}
    \item
      An identity between morphisms in~$\Ccat$ holds if and only if it holds after applying~$F$.
      In particular, a diagram in~$\Ccat$ commutes if and only if it does so after applying~$F$ to it.
      We have used this observation to show the commutativity of the diagram~\eqref{naturality of eta}.
  \end{enumerate}
  If~$F$ is not only faithful but also full, then we can observe the following:
  \begin{enumerate}[resume]
    \item
      A morphism~$f \colon X \to X'$ in~$\Ccat$ is an isomorphism if and only if the morphism~$F(f) \colon F(X) \to F(X')$ in~$\Dcat$ is an isomorphism.
      (This means that the functor~$F$ \emph{reflects}\index{reflects isomorphisms}\index{functor!reflects isomorphism} isomorphisms.)
      We have used this observation to show that~$\eta_X$ is again an isomorphism.
  \end{enumerate}
\end{remark*}




